\documentclass{ximera}
 \usepackage{amsmath}


%\newcommand{\dfn}{\textbf}
\renewcommand{\vec}[1]{{\overset{\boldsymbol{\rightharpoonup}}{\mathbf{#1}}}\hspace{0in}}
%% Simple horiz vectors
\renewcommand{\vector}[1]{\left\langle #1\right\rangle}
\newcommand{\arrowvec}[1]{{\overset{\rightharpoonup}{#1}}}
\newcommand{\R}{\mathbb{R}}
\newcommand{\transpose}{\intercal}
\newcommand{\ro}{\texttt{R}}%% row operation
\newcommand{\dotp}{\bullet}%% dot product

\usetikzlibrary{calc,bending}
\tikzset{>=stealth}


\usepackage{mdframed} % For framing content
%\usepackage{ifthen}   % For conditional statements

% Define the 'concept' environment with an optional header
\newenvironment{concept}[1][]{%
  \begin{mdframed}[linecolor=black, linewidth=2pt, innertopmargin=5pt, innerbottommargin=5pt, skipabove=12pt, skipbelow=12pt]%
    \noindent\large\textbf{#1}\normalsize%
}{%
  \end{mdframed}%
}











%% \colorlet{textColor}{black}
%% \colorlet{background}{white}
%% \colorlet{penColor}{blue!50!black} % Color of a curve in a plot
%% \colorlet{penColor2}{red!50!black}% Color of a curve in a plot
%% \colorlet{penColor3}{red!50!blue} % Color of a curve in a plot
%% \colorlet{penColor4}{green!50!black} % Color of a curve in a plot
%% \colorlet{penColor5}{orange!80!black} % Color of a curve in a plot
%% \colorlet{penColor6}{yellow!70!black} % Color of a curve in a plot
%% \colorlet{fill1}{penColor!20} % Color of fill in a plot
%% \colorlet{fill2}{penColor2!20} % Color of fill in a plot
%% \colorlet{fillp}{fill1} % Color of positive area
%% \colorlet{filln}{penColor2!20} % Color of negative area
%% \colorlet{fill3}{penColor3!20} % Fill
%% \colorlet{fill4}{penColor4!20} % Fill
%% \colorlet{fill5}{penColor5!20} % Fill
%% \colorlet{gridColor}{gray!50} % Color of grid in a plot






\begin{document}
\section*{Inverse of a Matrix}

We have learned how to add, subtract and multiply matrices. Now we discuss how division is performed for matrices. One cannot write $\frac{A}{B}$ for matrices $A$ and $B$ and we need to introduce the inverse of a matrix. For a matrix $A: n\times n$, its inverse, denoted as $A^{-1}$, is an $n\times n$ matrix such that
\begin{equation}\label{inverse}
AA^{-1}= A^{-1}A = I.
\end{equation}
Using this property, if for example, for matrices $A, B, C$, we have $AB = C$ and we wish to solve for matrix $B$, then we can multiply by the inverse of $A$ as follows:
\begin{equation*}
AB= C \hspace{.5cm} \Rightarrow \hspace{.5cm} A^{-1}AB= A^{-1}C \hspace{.5cm} \Rightarrow \hspace{.5cm} B = A^{-1}C.
\end{equation*}

\begin{question} Which of the following is the inverse of
\begin{equation*}
A= \left(\begin{array}{cc}
  2 & 3   \\
  1 &  4
\end{array}\right)
\end{equation*}
\begin{multipleChoice}
\choice{\[ \left(\begin{array}{cc}
  1 & 4  \\
  0 &  -5
\end{array}\right)
\]}

\choice{\[ \left(\begin{array}{cc}
  2 & -3   \\
  -1 &  2
\end{array}\right)
\]}

\choice{\[ \begin{pmatrix}
 -1\\ 2
\end{pmatrix}
\]}


\choice[correct]{ \[ \left(\begin{array}{cc}
  4/5 & -3/5  \\
  -1/5 &  2/5
\end{array}\right)
\]}

\choice{\[ \frac{1}{2} \left(\begin{array}{cc}
  3 & 2  \\
  4 &  1
\end{array}\right)
\]}
\end{multipleChoice}
\end{question}

If a matrix has an inverse then we say it is invertible. Note that there can be only one inverse for any given matrix, making the inverse of a matrix unique.

Let us now discuss techniques on how to find inverses. One may find the inverse of a $2\times 2$ matrix,
 \[A= \left(\begin{array}{cc}
  a & b  \\
  c &  d
\end{array}\right),
\]
by
\begin{equation}\label{informula}
A^{-1}=  \frac{1}{ad-bc} \left(\begin{array}{cc}
  d & -b  \\
  -c &  a
\end{array}\right).
\end{equation}
That is, we find the determinant, $ad-bc$ by cross multiplying and we change the entries inside $A$ by switching entries $a_{11}$ and $a_{22}$ and making $a_{12}$ and $a_{21}$ negative.

\begin{example}
 Find the inverse of
\[ \left(\begin{array}{cc}
  4 & -1 \\
  2 & -3
\end{array}\right)
\]
and then verify that the matrix found is the inverse.

\begin{prompt}
Based on \eqref{informula}, we find,
\[
A^{-1}=  \frac{1}{-12-\answer[given]{-2}} \left(\begin{array}{cc}
  \answer[given]{-3} &  \answer[given]{1} \\
  \answer[given]{-2} &  \answer[given]{4}
\end{array}\right)= \left(\begin{array}{cc}
  3/10 &  \answer[given]{-1/10} \\
  \answer[given]{1/5} &  \answer[given]{-2/5}
\end{array}\right)
\]
Since,
\[
AA^{-1}= \left(\begin{array}{cc}
  \answer[given]{1} &  \answer[given]{0} \\
  \answer[given]{0} &  \answer[given]{1}
\end{array}\right), \hspace{.5cm} \text{and} \hspace{.5cm}
A^{-1}A= \left(\begin{array}{cc}
  \answer[given]{1} &  \answer[given]{0} \\
  \answer[given]{0} &  \answer[given]{1}
\end{array}\right),
\]
then $A^{-1}$ is the inverse of $A$.
\end{prompt}
\end{example}

\begin{example}\label{ex2}
Find the inverse of
\[
A= \left(\begin{array}{cc}
  5 &  1 \\
  -2 &  0
\end{array}\right),
\]
by two methods:\\
 a. applying \eqref{informula},\\
 b. using the definition of inverse given in \eqref{inverse}.

\begin{prompt}
Based on the formula \eqref{informula}, we have
\begin{equation}\label{Amatrix}
A^{-1}= \frac{1}{\answer[given]{2}} \left(\begin{array}{cc}
  \answer[given]{0} &  \answer[given]{-1} \\
  \answer[given]{2} &  \answer[given]{5}
\end{array}\right)= \left(\begin{array}{cc}
  \answer[given]{0} &  \answer[given]{-1/2} \\
  \answer[given]{1} &  \answer[given]{5/2}
\end{array}\right).
\end{equation}
Now by \eqref{inverse} we may find the inverse by the following setup: find entries $x_{1}, x_{2}, x_{3}, x_{4}$ such that,
\begin{equation}\label{setup}
\underbrace{\left(\begin{array}{cc}
  5 &  1 \\
  -2 &  0
\end{array}\right)}_{A}\underbrace{\left(\begin{array}{cc}
  x_{1} &  x_{2} \\
  x_{3} &  x_{4}
\end{array}\right)}_{A^{-1}}= \underbrace{\left(\begin{array}{cc}
  1 &  0 \\
  0 &  1
\end{array}\right)}_{I},
\end{equation}
Hence we need to solve the system,
\begin{eqnarray*}
5x_{1}+x_{3}&=&1\\
5x_{2} + x_{\answer[given]{4}} &=& \answer[given]{0}\\
-2x_{1}&=& \answer[given]{0}\\
-2x_{2} &=& \answer[given]{1}
\end{eqnarray*}
leading to,
\begin{equation}\label{entries}
x_{1} = \answer[given]{0},\hspace{.5cm} x_{2} =\answer[given]{-1/2},\hspace{.5cm} x_{3} = \answer[given]{1},\hspace{.5cm} x_{4} = \answer[given]{5/2},
\end{equation}
which correspond to the entries we obtained in \eqref{entries}. Note that $A^{-1}$ needs to satisfy both multiplications $AA^{-1}=I$ and $A^{-1}A=I$ and our setup in \eqref{setup} only uses $AA^{-1}=I$. Thus, we form $A^{-1}$ based on entries in \eqref{entries} and verify this second condition as follows,
\[A^{-1}A= \left(\begin{array}{cc}
 \answer[given]{0}  & \answer[given]{-1/2} \\
  \answer[given]{1} &  \answer[given]{5/2}
\end{array}\right)
\left(\begin{array}{cc}
 \answer[given]{5}  & \answer[given]{1} \\
  \answer[given]{-2} &  \answer[given]{0}
\end{array}\right) = \left(\begin{array}{cc}
 \answer[given]{1}  & \answer[given]{0} \\
  \answer[given]{0} &  \answer[given]{1}
\end{array}\right).
\]
\end{prompt}
\end{example}


The second method shown in example \ref{ex2} becomes more involved for a $3\times 3$ matrix. For example, for the matrix,
\[
A= \left(\begin{array}{ccc}
  2 &  0 & 1 \\
  0 &  3 & 4 \\
  1 & -1 & -2
\end{array}\right),
\]
 we set up,
 \[
\left(\begin{array}{ccc}
  2 &  0 & 1 \\
  0 &  3 & 4 \\
  1 & -1 & -2
\end{array}\right) \left(\begin{array}{ccc}
  x_{11} &  x_{12} & x_{13} \\
 x_{21} &  x_{22} & x_{23} \\
  x_{31} &  x_{32} & x_{33}
\end{array}\right)
= \left(\begin{array}{ccc}
  1 &  0 & 0 \\
  0 &  1 & 0 \\
  0 & 0 & 1
\end{array}\right),
\]
which produces a system of nine equations to solve. To avoid such long calculations, we use the following more direct method for finding the inverse of an $n\times n$ matrix for $n>2$. Begin by forming the matrix $(A \hspace{.08cm}|\hspace{.08cm} I)$, where the identity matrix, $I$, has the same size as $A$. Then apply row operations to convert $(A \hspace{.08cm}|\hspace{.08cm} I)$ to $(I \hspace{.08cm}|\hspace{.08cm} B)$. That is covert $A$ to the identity matrix and apply each row operation in the process to the entire row including to that of the adjacent identity matrix. Then the matrix $B$ is the inverse of $A$. In short,
\begin{equation}\label{method}
(A \hspace{.08cm}|\hspace{.08cm} I) \hspace{.5cm}\xrightarrow{\text{row operations}}\hspace{.5cm} (I \hspace{.08cm} |\hspace{.08cm} A^{-1}).
\end{equation}

 \begin{example}
Find the inverse of
 \[A= \left(\begin{array}{ccc}
  1 &  -1 & 0 \\
  2 &  0 & 3 \\
  1 & -2 & 0
\end{array}\right)
\]
and verify your answer.

\begin{prompt}
We begin by forming,
\[
(A \hspace{.08cm}|\hspace{.08cm} I)= \left(\begin{array}{ccc|ccc}
  1 &  -1 & 0 & 1 &0 &0 \\
  2 &  0 & 3 & 0 &1 &0\\
  1 & -2 & 0 & 0 &0 &1
\end{array}\right)
\hspace{.2cm}\Rightarrow \hspace{.2cm}
\left(\begin{array}{ccc|ccc}
  1 &  -1 & 0 & 1 &0 &0 \\
  0 &  2 & \answer[given]{3} & \answer[given]{-2} &\answer[given]{1} &\answer[given]{0}\\
  0 & -1 & \answer[given]{0} & \answer[given]{-1} &\answer[given]{0} &\answer[given]{1}
\end{array}\right)\]
\[\hspace{.2cm}\Rightarrow \hspace{.2cm}
\left(\begin{array}{ccc|ccc}
  1 &  0 & \answer[given]{3/2} & \answer[given]{0} &\answer[given]{1/2} &0 \\
  0 &  1 & \answer[given]{3/2} & \answer[given]{-1} &\answer[given]{1/2} &0\\
  0 & 0 & 3/2 & \answer[given]{-2} &\answer[given]{1/2} &\answer[given]{1}
\end{array}\right)
\]
\[
\hspace{.2cm}\Rightarrow \hspace{.2cm}
\left(\begin{array}{ccc|ccc}
  1 &  0 & 0 & \answer[given]{2}  &\answer[given]{0}  &\answer[given]{-1}  \\
  0 &  1 &0& \answer[given]{1}  & \answer[given]{0} &\answer[given]{-1} \\
  0 & 0 & 3/2&\answer[given]{-2}   &\answer[given]{1/2}  &\answer[given]{1}
\end{array}\right)
\]
\[\hspace{.2cm}\Rightarrow \hspace{.2cm}
\left(\begin{array}{ccc|ccc}
  1 &  0 & 0 & \answer[given]{2}  &\answer[given]{0}  &\answer[given]{-1}  \\
  0 &  1 &0& \answer[given]{1}  & \answer[given]{0} &\answer[given]{-1} \\
  0 & 0 & 1&\answer[given]{-4/3}   &\answer[given]{1/3}  &\answer[given]{2/3}
\end{array}\right)
\]
Thus,
\[A^{-1}= \left(\begin{array}{ccc}
  \answer[given]{2}  &\answer[given]{0}  &\answer[given]{-1}  \\
  \answer[given]{1}  & \answer[given]{0} &\answer[given]{-1} \\
 \answer[given]{-4/3}   &\answer[given]{1/3}  &\answer[given]{2/3}
\end{array}\right)\]
To verify that the above is the inverse of $A$, we find,
\[AA^{-1}=\left(\begin{array}{ccc}
   1  &-1  &0 \\
   2  & 0 &3 \\
  1   &-2 &0
\end{array}\right)\left(\begin{array}{ccc}
   \answer[given]{2}  &\answer[given]{0}  &\answer[given]{-1}  \\
   \answer[given]{1}  & \answer[given]{0} &\answer[given]{-1} \\
  \answer[given]{-4/3}   &\answer[given]{1/3}  &\answer[given]{2/3}
\end{array}\right)=\left(\begin{array}{ccc}
   \answer[given]{1}  &\answer[given]{0}  &\answer[given]{0}  \\
   \answer[given]{0}  & \answer[given]{1} &\answer[given]{0} \\
  \answer[given]{0}   &\answer[given]{0}  &\answer[given]{1}
\end{array}\right),
\]
and
 \[A^{-1}A=\left(\begin{array}{ccc}
   \answer[given]{2}  &\answer[given]{0}  &\answer[given]{-1}  \\
   \answer[given]{1}  & \answer[given]{0} &\answer[given]{-1} \\
  \answer[given]{-4/3}   &\answer[given]{1/3}  &\answer[given]{2/3}
\end{array}\right)\left(\begin{array}{ccc}
   1  &-1  &0 \\
   2  & 0 &3 \\
  1   &-2 &0
\end{array}\right)=\left(\begin{array}{ccc}
   \answer[given]{1}  &\answer[given]{0}  &\answer[given]{0}  \\
   \answer[given]{0}  & \answer[given]{1} &\answer[given]{0} \\
  \answer[given]{0}   &\answer[given]{0}  &\answer[given]{1}
\end{array}\right),
\]
\end{prompt}
\end{example}

\section*{Properties of inverses}
Recall that the inverse matrix, $A^{-1}$ is the matrix that satisfies,
\begin{equation*}
AA^{-1}= A^{-1}A= I.
\end{equation*}
Note that for $A$ to be invertible, it must be a square matrix and cannot have a row or column of zeros. For example,
 \[B= \left(\begin{array}{ccc}
  2 &  3 & 0 \\
  4 &  1 & 8
\end{array}\right), \hspace{.5cm} C= \left(\begin{array}{ccc}
  1 &  2 & 5 \\
  2 &  0 & 1 \\
  0 & 0& 0
\end{array}\right), \hspace{.5cm} D = \left(\begin{array}{ccc}
  3 &  0 & 9 \\
  2 &  0 & -1 \\
  0 & 0& 3
\end{array}\right),
\]
cannot have an inverse since there are no matrices $B^{-1}, C^{-1}, D^{-1}$ that may be multiplied by $B,C,D$ to give, $BB^{-1}= CC^{-1}=D^{-1}D=I$. One may also observe that by method given in \eqref{method}, the above matrices cannot be transformed into identity matrices.

\begin{question} Which of the following is invertible?
\begin{multipleChoice}
\choice{\[\left(\begin{array}{cccc}
  0 &  -1 & 0 &0 \\
  -1 &  5 & 3 &0 \\
  0 & -3& 4 &1
\end{array}\right)\]}
\choice{\[\left(\begin{array}{cccc}
  7 &  -1 & 0 &8 \\
  11 &  -3 & 0 &0 \\
  -5 & 0& 0 &2\\
   0& 0&0&1
\end{array}\right)\]}
\choice[correct]{\[\left(\begin{array}{ccc}
  1 &  0& 0  \\
  0 &  1 & 0 \\
  0 & 0& 1
\end{array}\right)\]}
\choice{\[\left(\begin{array}{cc}
8 & 0\\
0 &0
\end{array}\right)\]}
\choice[correct]
{\[\left(\begin{array}{cccc}
  0 &  0& 4&0  \\
  9 &  0 & 0 &0 \\
  0 & 0& 0&8\\
  0&2&0&0
\end{array}\right)\]}
\choice[correct]
{\[\left(\begin{array}{cccc}
  -2 &  -1& 0&0  \\
  0 &  8 & 0 &1 \\
  10 & -3& 9&-4\\
  7&-5&0&3
\end{array}\right)\]}
\end{multipleChoice}
\end{question}

Now we study properties of inverses by going through an example.

\begin{example}\label{example1}
Consider the matrix,
\[A= \left(\begin{array}{cc}
  2 &  -1  \\
   7&  4
\end{array}\right)\]

a. Find $A^{-1}$.
\begin{prompt}
\[A^{-1}= \frac{1}{\answer[given]{8}- \answer[given]{-7}}\left(\begin{array}{cc}
  \answer[given]{4} &  \answer[given]{1}  \\
   \answer[given]{-7}&  \answer[given]{2}
\end{array}\right)= \left(\begin{array}{cc}
  \answer[given]{4/15} &  \answer[given]{1/15}  \\
   \answer[given]{-7/15}&  \answer[given]{2/15}
\end{array}\right)
\]
\end{prompt}
b. Determine the inverse of $A^{-1}$.
\begin{prompt}
\[\left(A^{-1}\right)^{-1}= \frac{1}{\answer[given]{8/225}- \answer[given]{-7/225}}\left(\begin{array}{cc}
  \answer[given]{2/15} &  \answer[given]{-1/15}  \\
   \answer[given]{7/15}&  \answer[given]{4/15}
\end{array}\right)= \left(\begin{array}{cc}
  \answer[given]{2} &  \answer[given]{-1}  \\
   \answer[given]{7}&  \answer[given]{4}
\end{array}\right)
\]
\end{prompt}
c. What is $\left(A^{T}\right)^{-1}$?
\begin{prompt}
\[A^{T}= \left(\begin{array}{cc}
  \answer[given]{2} &  \answer[given]{7}  \\
   \answer[given]{-1}&  \answer[given]{4}
\end{array}\right)
\]
Therefore,
\[\left(A^{T}\right)^{-1}= \frac{1}{\answer[given]{15}}\left(\begin{array}{cc}
  \answer[given]{4} &  \answer[given]{-7}  \\
   \answer[given]{1}&  \answer[given]{2}
\end{array}\right) = \left(\begin{array}{cc}
  \answer[given]{4/15} &  \answer[given]{-7/15}  \\
   \answer[given]{1/15}&  \answer[given]{2/15}
\end{array}\right) \]
\end{prompt}

d. Using your answer in part a, determine $\left(A^{-1}\right)^{T}$.
\begin{prompt}
\[\left(A^{-1}\right)^{T}= \left(\begin{array}{cc}
  \answer[given]{4/15} &  \answer[given]{-7/15}  \\
   \answer[given]{1/15}&  \answer[given]{2/15}
\end{array}\right) \]
\end{prompt}

e. What is $(kA)^{-1}$, where $k$ is any arbitrary number.
\begin{prompt}
\[kA = \left(\begin{array}{cc}
  2k &  \answer[given]{-k}  \\
   \answer[given]{7k}&  \answer[given]{4k}
\end{array}\right) \]
Hence,
\[\left(kA\right)^{-1}= \frac{1}{\answer[given]{8k^{2}}- \answer[given]{-7k^2}} \left(\begin{array}{cc}
  \answer[given]{4k} &  \answer[given]{k}  \\
   \answer[given]{-7k}&  \answer[given]{2k}
\end{array}\right) = \left(\begin{array}{cc}
  \frac{4}{15k} &  \answer[given]{1/(15k)}  \\
   \answer[given]{-7/(15k)}&  \answer[given]{2/(15k)}
\end{array}\right)\]
\[ = \frac{1}{k}\left(\begin{array}{cc}
  \answer[given]{4/15} &  \answer[given]{1/15}  \\
   \answer[given]{-7/15}&  \answer[given]{2/15}
\end{array}\right). \]
\end{prompt}

f. Find $B^{-1}$ if
\[B= \left(\begin{array}{cc}
  1 &  -3  \\
   5&  -2
\end{array}\right).
\]


\begin{prompt}
\[B^{-1}= \left(\begin{array}{cc}
  \answer[given]{-2/13} &  \answer[given]{3/13}  \\
   \answer[given]{-5/13}&  \answer[given]{1/13}
\end{array}\right). \]
\end{prompt}

g. What is $(AB)^{-1}$?

\begin{prompt}
\[AB = \left(\begin{array}{cc}
  \answer[given]{2} &  \answer[given]{-1}  \\
   \answer[given]{7}&  \answer[given]{4}
\end{array}\right)\left(\begin{array}{cc}
  \answer[given]{1} &  \answer[given]{-3}  \\
   \answer[given]{5}&  \answer[given]{-2}
\end{array}\right)= \left(\begin{array}{cc}
  \answer[given]{-3} &  \answer[given]{-4}  \\
   \answer[given]{27}&  \answer[given]{-29}
\end{array}\right)  \]
Thus,
\[ \left(AB\right)^{-1} = \left(\begin{array}{cc}
  \answer[given]{-29/195} &  \answer[given]{4/195}  \\
   \answer[given]{-27/195}&  \answer[given]{-3/195}
\end{array}\right). \]
\end{prompt}

h. What is $A^{-1}B^{-1}$?

\begin{prompt}
Using parts b and f we have,
\[ A^{-1}B^{-1}= \left(\begin{array}{cc}
  \answer[given]{4/15} &  \answer[given]{1/15}  \\
   \answer[given]{-7/15}&  \answer[given]{2/15}
\end{array}\right)\left(\begin{array}{cc}
  \answer[given]{-2/13} &  \answer[given]{3/13}  \\
   \answer[given]{-5/13}&  \answer[given]{1/13}
\end{array}\right)
\]
\[= \left(\begin{array}{cc}
  \answer[given]{-13/195} &  \answer[given]{13/195}  \\
   \answer[given]{4/195}&  \answer[given]{-19/195}
\end{array}\right). \]
\end{prompt}

i. What is $B^{-1}A^{-1}$?

\begin{prompt}
Similar to the answer in part h, we find,
\[ B^{-1}A^{-1}= \left(\begin{array}{cc}
  \answer[given]{-29/195} &  \answer[given]{4/195}  \\
   \answer[given]{-27/195}&  \answer[given]{-3/195}
\end{array}\right).\]
\end{prompt}

j. Based on the results in parts g-i, which of the following is true?
\begin{multipleChoice}
\choice{$\left(AB\right)^{-1}= A^{-1}B^{-1}$}
\choice[correct]{$\left(AB\right)^{-1}= B^{-1}A^{-1}$}
\end{multipleChoice}
\end{example}

The example above illustrates the following facts about inverses. \\
\[
\begin{array}{ccc}
&\left(A^{-1}\right)^{-1}=A &\hspace{1.5cm} \text{see example \ref{example1} a,b}\\ \\
&\left(A^{T}\right)^{-1}= \left(A^{-1}\right)^{T} &\hspace{1.5cm} \text{see example \ref{example1} c,d,} \\ \\
&\left(kA\right)^{-1}= \frac{1}{k}A^{-1} &\hspace{1.5cm} \text{see example \ref{example1} a,e,}\\ \\
&(AB)^{-1}= B^{-1}A^{-1} &\hspace{1.5cm} \text{see example \ref{example1} g-j}.
\end{array}
\]
Note that for three matrices, $A, B, C$, we may write using the last statement above,
\begin{equation*}
(ABC)^{-1} = \left(\left(AB\right)C\right)^{-1} = C^{-1} \left(AB\right)^{-1} = C^{-1}B^{-1}A^{-1}.
\end{equation*}
We may apply this process to any finite number of matrices:
\begin{equation*}
\left(A_{1}A_{2}...A_{k}\right)^{-1}= A_{k}^{-1} A_{k-1}^{-1}...A_{2}^{-1}A_{1}^{-1}.
\end{equation*}

Let us now use the methods introduced in this chapter to solve a system of equations.
\begin{example}
Solve the following system of equations by writing it as $Ax=B$ and then using the inverse of the coefficient matrix, $A$.
\begin{eqnarray*}
5x+2y &=& 4\\
2x-8y&=& 28.
\end{eqnarray*}

\begin{prompt}
The system of equations may be presented as
\begin{equation}
\underbrace{\left(\begin{array}{cc}
  \answer[given]{5} &  \answer[given]{2} \\
  \answer[given]{2} &  \answer[given]{-8}
\end{array}\right)}_{A}\underbrace{\begin{pmatrix}
  x \\
  y
\end{pmatrix}}_{\textbf{x}}= \underbrace{\begin{pmatrix}
  \answer[given]{4} \\
  \answer[given]{28}
\end{pmatrix}}_{B}.
\end{equation}
Thus, $\textbf{x}= A^{-1}B$. We find,
\[
\left(A \hspace{.08cm}|\hspace{.08cm} I\right) = \left(\begin{array}{cc|cc}
  \answer[given]{5} &  2 & \answer[given]{1} & \answer[given]{0}  \\
  \answer[given]{2} &  -8 & \answer[given]{0} & \answer[given]{1}  \\
\end{array}\right)
\hspace{.2cm}\xrightarrow{R_{1}\leftrightarrow R_{2}} \hspace{.2cm}
\left(\begin{array}{cc|cc}
  \answer[given]{2} &  \answer[given]{-8} & \answer[given]{0} & \answer[given]{1}  \\
  \answer[given]{5} &  \answer[given]{2} & \answer[given]{1} & \answer[given]{0}  \\
\end{array}\right)
\]
\[\hspace{.2cm}\Rightarrow \hspace{.2cm}
\left(\begin{array}{cc|cc}
  1 &  -4 & \answer[given]{0} & \answer[given]{1/2}  \\
  0 &  \answer[given]{22} & 1 & \answer[given]{-5/2}  \\
\end{array}\right)
\hspace{.2cm}\Rightarrow \hspace{.2cm}
\left(\begin{array}{cc|cc}
  1 &  0 & \answer[given]{2/11} & \answer[given]{1/22}  \\
  0 &  1 & \answer[given]{1/22} & \answer[given]{-5/44}  \\
\end{array}\right)
\]
Therefore,
\[
A^{-1}= \left(\begin{array}{cc}
 \answer[given]{2/11} & \answer[given]{1/22}  \\
  \answer[given]{1/22} & \answer[given]{-5/44}  \\
\end{array}\right),
\]
and
\[
\textbf{x} = A^{-1}B = \left(\begin{array}{cc}
 \answer[given]{2/11} & \answer[given]{1/22}  \\
  \answer[given]{1/22} & \answer[given]{-5/44}  \\
\end{array}\right) \begin{pmatrix}
\answer[given]{4}\\
\answer[given]{28}
\end{pmatrix} = \begin{pmatrix}
\answer[given]{2}\\
\answer[given]{-3}
\end{pmatrix}
\]
giving $x= \answer[given]{2}, \hspace{.3cm} y= \answer[given]{-3}$. \\

One can check that the result above is indeed the solution to the system of equations. Observe that when solving for $A\textbf{x}=B$ for $\textbf{x}$, we multiply both sides of the equation by $A^{-1}$ from the left hand side. Recall that in the multiplication of matrices order has to be noted and if $A^{-1}$ is multiplied by the right hand side by $B$:
\[
BA^{-1} = \begin{pmatrix} 4\\
28
\end{pmatrix} \left(\begin{array}{cc}
 \answer[given]{2/11} & \answer[given]{1/22}  \\
  \answer[given]{1/22} & \answer[given]{-5/44}  \\
\end{array}\right) \]
we will have a $2 \times 1$ matrix multiplied by a $2\times 2$ matrix, which cannot be completed.
\end{prompt}
\end{example}















\end{document} 