\documentclass{ximera}

\author{Tae Eun Kim}

\newcommand{\dfn}{\textbf}
\renewcommand{\vec}[1]{{\overset{\boldsymbol{\rightharpoonup}}{\mathbf{#1}}}\hspace{0in}}
%% Simple horiz vectors
\renewcommand{\vector}[1]{\left\langle #1\right\rangle}
\newcommand{\arrowvec}[1]{{\overset{\rightharpoonup}{#1}}}
\newcommand{\R}{\mathbb{R}}
\newcommand{\transpose}{\intercal}
\newcommand{\ro}{\texttt{R}}%% row operation
\newcommand{\dotp}{\bullet}%% dot product

\usetikzlibrary{calc,bending}
\tikzset{>=stealth}


\usepackage{mdframed} % For framing content
%\usepackage{ifthen}   % For conditional statements

% Define the 'concept' environment with an optional header
\newenvironment{concept}[1][]{%
  \begin{mdframed}[linecolor=black, linewidth=2pt, innertopmargin=5pt, innerbottommargin=5pt, skipabove=12pt, skipbelow=12pt]%
    \noindent\large\textbf{#1}\normalsize%
}{%
  \end{mdframed}%
}











%% \colorlet{textColor}{black}
%% \colorlet{background}{white}
%% \colorlet{penColor}{blue!50!black} % Color of a curve in a plot
%% \colorlet{penColor2}{red!50!black}% Color of a curve in a plot
%% \colorlet{penColor3}{red!50!blue} % Color of a curve in a plot
%% \colorlet{penColor4}{green!50!black} % Color of a curve in a plot
%% \colorlet{penColor5}{orange!80!black} % Color of a curve in a plot
%% \colorlet{penColor6}{yellow!70!black} % Color of a curve in a plot
%% \colorlet{fill1}{penColor!20} % Color of fill in a plot
%% \colorlet{fill2}{penColor2!20} % Color of fill in a plot
%% \colorlet{fillp}{fill1} % Color of positive area
%% \colorlet{filln}{penColor2!20} % Color of negative area
%% \colorlet{fill3}{penColor3!20} % Fill
%% \colorlet{fill4}{penColor4!20} % Fill
%% \colorlet{fill5}{penColor5!20} % Fill
%% \colorlet{gridColor}{gray!50} % Color of grid in a plot


\begin{document}
\begin{exercise}
  The following column 3-vector $\vec{p}$ represents the \link[official gray
  color]{https://bux.osu.edu/color/primary-colors} of the Ohio State
  University, converted to RGB from the hex code \verb|#a7b1b7|:
  \[
    \vec{p} = \begin{pmatrix}
      167\\ 177\\ 183
    \end{pmatrix}
    \qquad
    \begin{array}{l}
      \text{Red}\\
      \text{Green}\\
      \text{Blue}
    \end{array}
  \]
  In this problem, we will create three additional shades of gray that are
  lighter than the given. Recalling that white is represented by
  \[
    \vec{w} = \begin{pmatrix}
      255\\ 255\\ 255
    \end{pmatrix}
    \qquad
    \begin{array}{l}
      \text{Red}\\
      \text{Green}\\
      \text{Blue}
    \end{array}
  \]
  one strategy is to take the difference of the two RGB vectors, and
  add its various fractions to $\vec{p}$ to generate lighter shades of
  the OSU gray.

  First, compute the differences of RGB intensities between white and the OSU
  gray.
  \[
    \vec{d} = \vec{w} - \vec{p} =
    \begin{pmatrix}
      \answer{88}\\
      \answer{78}\\
      \answer{72}
    \end{pmatrix}
  \]

  Generate a light gray by adding 25\% of $\vec{d}$ onto $\vec{p}$.
  \[
    \vec{p}_1 = \vec{p} + \answer{0.25}\,\vec{d} =
    \begin{pmatrix}
      \answer{189}\\
      \answer{196.5}\\
      \answer{201}
    \end{pmatrix}
  \]

  Generate a lighter gray by adding 50\% of $\vec{d}$ onto $\vec{p}$.
  \[
    \vec{p}_2 = \vec{p} + \answer{0.5}\,\vec{d} =
    \begin{pmatrix}
      \answer{211}\\
      \answer{216}\\
      \answer{219}
    \end{pmatrix}
  \]

  Finally, generate an even lighter gray by adding 75\% of $\vec{d}$
  onto $\vec{p}$.
  \[
    \vec{p}_3 = \vec{p} + \answer{0.75}\,\vec{d} =
    \begin{pmatrix}
      \answer{233}\\
      \answer{235.5}\\
      \answer{237}
    \end{pmatrix}
  \]

  \begin{feedback}[correct]
    Use an online RGB color picker such as
    \link[this]{https://rgbcolorpicker.com} for visual confirmation of
    your work!
  \end{feedback}

%% MATLAB script for this problem
% p = [167 177 183]';
% w = 255*[1 1 1]';
% d = w - p;
% shades = p + d * [.25 .5 .75]; % columns are p_1, p_2, p_3
%
% >> d'
% ans =
%       88    78    72
%
% >> shades
% shades =
%          189    211  233
%          196.5  216  235.5
%          201    219  237

\end{exercise}
\end{document}
