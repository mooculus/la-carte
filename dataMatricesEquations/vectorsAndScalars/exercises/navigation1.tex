\documentclass{ximera}

\author{Parisa Fatheddin \and Bart Snapp}

\newcommand{\dfn}{\textbf}
\renewcommand{\vec}[1]{{\overset{\boldsymbol{\rightharpoonup}}{\mathbf{#1}}}\hspace{0in}}
%% Simple horiz vectors
\renewcommand{\vector}[1]{\left\langle #1\right\rangle}
\newcommand{\arrowvec}[1]{{\overset{\rightharpoonup}{#1}}}
\newcommand{\R}{\mathbb{R}}
\newcommand{\transpose}{\intercal}
\newcommand{\ro}{\texttt{R}}%% row operation
\newcommand{\dotp}{\bullet}%% dot product

\usetikzlibrary{calc,bending}
\tikzset{>=stealth}


\usepackage{mdframed} % For framing content
%\usepackage{ifthen}   % For conditional statements

% Define the 'concept' environment with an optional header
\newenvironment{concept}[1][]{%
  \begin{mdframed}[linecolor=black, linewidth=2pt, innertopmargin=5pt, innerbottommargin=5pt, skipabove=12pt, skipbelow=12pt]%
    \noindent\large\textbf{#1}\normalsize%
}{%
  \end{mdframed}%
}











%% \colorlet{textColor}{black}
%% \colorlet{background}{white}
%% \colorlet{penColor}{blue!50!black} % Color of a curve in a plot
%% \colorlet{penColor2}{red!50!black}% Color of a curve in a plot
%% \colorlet{penColor3}{red!50!blue} % Color of a curve in a plot
%% \colorlet{penColor4}{green!50!black} % Color of a curve in a plot
%% \colorlet{penColor5}{orange!80!black} % Color of a curve in a plot
%% \colorlet{penColor6}{yellow!70!black} % Color of a curve in a plot
%% \colorlet{fill1}{penColor!20} % Color of fill in a plot
%% \colorlet{fill2}{penColor2!20} % Color of fill in a plot
%% \colorlet{fillp}{fill1} % Color of positive area
%% \colorlet{filln}{penColor2!20} % Color of negative area
%% \colorlet{fill3}{penColor3!20} % Fill
%% \colorlet{fill4}{penColor4!20} % Fill
%% \colorlet{fill5}{penColor5!20} % Fill
%% \colorlet{gridColor}{gray!50} % Color of grid in a plot


\begin{document}
\begin{exercise}
If a cruise ship starts at $(0,0)$ at 7:00am and travels with the
true bearing $53^\circ$ at $20$ knots until 5:00pm, when it reaches
its destination, then what are its coordinates at the end of the trip?
Round your answers to two decimal places.
\begin{prompt}
  We start by converting the bearing and speed into cartesian
  coordinates. Since we want to round our final answers to two decimal
  places, we will round this intermediate results to three decimal
  places.  Write with me:
  \begin{align*}
    \vec{c} &=
    \begin{pmatrix}
      \left(\answer{20}\right)\cdot\sin\left(\answer{53}^\circ\right)\\
      \left(\answer{20}\right)\cdot\cos\left(\answer{53}^\circ\right)
    \end{pmatrix}\\
    &=\begin{pmatrix}
      \answer{15.973} \\
    \answer{12.036}
  \end{pmatrix}
  \end{align*}
  The number of hours between 7:00am and 5:00pm is $\answer{10}$ hours.
  Hence we are at the coordiantes (in nautical miles):
  \[
10 \cdot \vec{c} = \begin{pmatrix}
      \answer{159.73} \\
    \answer{120.36}
  \end{pmatrix}
  \]
\end{prompt}
\end{exercise}
\end{document}

%% MATLAB script for confirmation:
% spd = 20;                               % speed in knots
% th = 53;                                % true bearing in degrees
% T = 10;                                 % elapsed time in hours
% vel = spd*[sind(th); cosd(th)];         % velocity in Cart. coord.
% pos = T*vel;                            % final position
%
% >> round(vel, 3) =
%      15.973
%      12.036
% >> round(pos, 2) =
%      159.73
%      120.36
