\documentclass{article}
\usepackage{tikz}
\usetikzlibrary{matrix} %% for fancy matrix stuff
\usepackage{ccicons}
\usepackage{xcolor}
\usepackage[margin=0.5in]{geometry} % Adjust margins to ensure text can go off the page



\usepackage{gillius} % Ensure this package is installed
\renewcommand{\familydefault}{\sfdefault} % Set Gillius as the default font
\usepackage{amsmath} % For bold math symbols


%% For black/white switch
\usepackage{pagecolor}
\definecolor{bkgndcr}{rgb}{0,0,0}
\definecolor{txtcr}{rgb}{1,1,1}

%% \definecolor{bkgndcr}{RGB}{200,10, 10}
%% \definecolor{txtcr}{RGB}{200, 200, 200}


\pagecolor{bkgndcr}


\begin{document}
\boldmath % Makes all math text bold
\bfseries % Makes all text bold
\color{txtcr}
\begin{tikzpicture}[remember picture, overlay,scale=10,every node/.style={scale=10}]
  % Fill the page with the random text
\matrix (m) [transform shape,bkgndcr!80!txtcr,matrix of math nodes,
    column sep=0.2em, row sep=0.1em] at (0,0) {
  \bullet & \bullet & \bullet & \bullet & \bullet & \bullet & \bullet & \bullet  & \bullet & \bullet & \bullet & \bullet & \bullet & \bullet & \bullet \\
  \bullet & \bullet & \bullet & \bullet & \bullet & \bullet & \bullet & \bullet & \bullet & \bullet & \bullet & \bullet & \bullet & \bullet & \bullet \\
  \bullet & \bullet & \bullet & \bullet & \bullet & \bullet & \bullet & \bullet & \bullet & \bullet & \bullet & \bullet & \bullet & \bullet & \bullet \\
  \bullet & \bullet & \bullet & \bullet & \bullet & \bullet & \bullet & \bullet  & \bullet & \bullet & \bullet & \bullet & \bullet & \bullet & \bullet \\
  \bullet & \bullet & \bullet & \bullet & \bullet & \bullet & \bullet & \bullet & \bullet & \bullet & \bullet & \bullet & \bullet & \bullet & \bullet \\
  \bullet & \bullet & \bullet & \bullet & \bullet & \bullet & \bullet & \bullet & \bullet & \bullet & \bullet & \bullet & \bullet & \bullet & \bullet \\
  \bullet & \bullet & \bullet & \bullet & \bullet & \bullet & \bullet & \bullet  & \bullet & \bullet & \bullet & \bullet & \bullet & \bullet & \bullet \\
  \bullet & \bullet & \bullet & \bullet & \bullet & \bullet & \bullet & \bullet & \bullet & \bullet & \bullet & \bullet & \bullet & \bullet & \bullet \\
  \bullet & \bullet & \bullet & \bullet & \bullet & \bullet & \bullet &  0  & \bullet & \bullet & \bullet & \bullet & \bullet & \bullet & \bullet \\
  \bullet & \bullet & \bullet & \bullet & \bullet & \bullet & \bullet & 0  & 0 & \bullet & \bullet & \bullet & \bullet & \bullet & \bullet \\
  \bullet & \bullet & \bullet & \bullet & \bullet & \bullet & \bullet & 0 & 0 & 0 & \bullet & \bullet & \bullet & \bullet & \bullet \\
  \bullet & \bullet & \bullet & \bullet & \bullet & \bullet & \bullet & 0 & 0 & 0 & 0 & \bullet & \bullet & \bullet & \bullet \\
  \bullet & \bullet & \bullet & \bullet & \bullet & \bullet & \bullet & 0  & 0 & 0 & 0 & 0 & \bullet & \bullet & \bullet \\
};
%% \draw[bkgndcr!90!txtcr,thick] (m-8-8.north west) rectangle (m-13-15.south east);
%% \draw[bkgndcr!90!txtcr,thick] (m-9-9.north west) rectangle (m-13-15.south east);
%% \draw[bkgndcr!90!txtcr,thick] (m-10-10.north west) rectangle (m-13-15.south east);
%% \draw[bkgndcr!90!txtcr,thick] (m-11-11.north west) rectangle (m-13-15.south east);
%% \draw[bkgndcr!90!txtcr,thick] (m-12-12.north west) rectangle (m-13-15.south east);
\end{tikzpicture}
\sffamily
\flushleft\scalebox{14}{Matrix}\\[1cm]
\scalebox{14}{Methods}\\[1cm]
\scalebox{4}{\itshape\,Linear Algebra, \`a la carte}

\vfill

\begin{center}
  \scalebox{4}{Fatheddin $\bullet$ Kim $\bullet$ Snapp}
\end{center}
\small This document was typeset on \today. \hfill \Large\textsf{\ccbyncsa}\\
\small\texttt{GIT COMMIT: \input{.git/refs/heads/master}}\\
\small Source: \texttt{https://github.com/mooculus/la-carte}\hfill developed in \textbf{XIMERA}\\



\end{document}
