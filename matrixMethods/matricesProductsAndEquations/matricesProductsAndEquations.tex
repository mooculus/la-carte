\documentclass{ximera}

\newcommand{\dfn}{\textbf}
\renewcommand{\vec}[1]{{\overset{\boldsymbol{\rightharpoonup}}{\mathbf{#1}}}\hspace{0in}}
%% Simple horiz vectors
\renewcommand{\vector}[1]{\left\langle #1\right\rangle}
\newcommand{\arrowvec}[1]{{\overset{\rightharpoonup}{#1}}}
\newcommand{\R}{\mathbb{R}}
\newcommand{\transpose}{\intercal}
\newcommand{\ro}{\texttt{R}}%% row operation
\newcommand{\dotp}{\bullet}%% dot product
\renewcommand{\l}{\ell}
\let\defaultAnswerFormat\answerFormatBoxed
\usetikzlibrary{calc,bending}
\tikzset{>=stealth}


\usepackage{mdframed} % For framing content
%\usepackage{ifthen}   % For conditional statements

% Define the 'concept' environment with an optional header
\newenvironment{concept}[1][]{%
  \begin{mdframed}[linecolor=black, linewidth=2pt, innertopmargin=5pt, innerbottommargin=5pt, skipabove=12pt, skipbelow=12pt]%
    \noindent\large\textbf{#1}\normalsize%
}{%
  \end{mdframed}%
}



%% Define exercise collection. 
\makeatletter
\newcommand{\exerciseCollection}[2]{
\def\input@path{{#1}}
\activity{#1#2}}



\newcommand{\practicestyle}{
  
  \let\exercise\relax
\let\endexercise\relax
\let\c@exercise\relax
\let\problem\relax
\let\endproblem\relax
\let\c@problem\relax 
\newtheoremstyle{problem}
{\topsep}{\topsep}{\rmfamily}{}{\bfseries}{)}{ }{}
\theoremstyle{problem}
\newtheorem{problem}{}
\newtheorem{exercise}[problem]{}
\section{Exercises for Chapter~\thetitlenumber}
\small%\twocolumn
%% \let\exercise\relax
%%   \let\endexercise\relax
%%   \let\c@exercise\relax 
%%   \newtheoremstyle{exercise}
%%                   {\topsep}{\topsep}{\rmfamily}{}{\bfseries}{)}{~}{}
%% \theoremstyle{exercise}
%% \newtheorem{exercise}{}
}
\renewcommand\chapterstyle{%
  \def\activitystyle{activity-chapter}
  \normalsize
  %\onecolumn
  \def\maketitle{%
    \addtocounter{titlenumber}{1}%
                    {\flushleft\small\sffamily\bfseries\@pretitle\par\vspace{-1.5em}}%
                    {\flushleft\LARGE\sffamily\bfseries\thetitlenumber\hspace{1em}\@title \par }%
                    {\vskip .6em\noindent\textit\theabstract\setcounter{problem}{0}\setcounter{section}{0}}%
                    \par\vspace{2em}
                    \phantomsection\addcontentsline{toc}{section}{\textbf{\thetitlenumber\hspace{1em}\@title}}%
}}
\makeatother







%% \colorlet{textColor}{black}
%% \colorlet{background}{white}
%% \colorlet{penColor}{blue!50!black} % Color of a curve in a plot
%% \colorlet{penColor2}{red!50!black}% Color of a curve in a plot
%% \colorlet{penColor3}{red!50!blue} % Color of a curve in a plot
%% \colorlet{penColor4}{green!50!black} % Color of a curve in a plot
%% \colorlet{penColor5}{orange!80!black} % Color of a curve in a plot
%% \colorlet{penColor6}{yellow!70!black} % Color of a curve in a plot
%% \colorlet{fill1}{penColor!20} % Color of fill in a plot
%% \colorlet{fill2}{penColor2!20} % Color of fill in a plot
%% \colorlet{fillp}{fill1} % Color of positive area
%% \colorlet{filln}{penColor2!20} % Color of negative area
%% \colorlet{fill3}{penColor3!20} % Fill
%% \colorlet{fill4}{penColor4!20} % Fill
%% \colorlet{fill5}{penColor5!20} % Fill
%% \colorlet{gridColor}{gray!50} % Color of grid in a plot


\author{Parisa Fatheddin and Bart Snapp}

\title{Matrices, products, and equations}


\begin{document}
\begin{abstract}
  A concrete introduction to matrices and their connections to systems
  of equations.
\end{abstract}
\maketitle

\begin{quote}
  Probably no other area of mathematics has been applied in such
  numerous and diverse contexts as the theory of matrices. In
  mechanics, electromagnetics, statistics, economics, operations
  research, the social sciences, and so on, the list of applications
  seems endless. By and large this is due to the utility of matrix
  structure and methodology in conceptualizing sometimes complicated
  relationships and in the orderly processing of otherwise tedious
  algebraic calculations and numerical manipulations.



  \hfill ---\link[J.\ Cochran]{https://www.amazon.com/Applied-Mathematics-Principles-Applications-Probability/dp/0534980260}
\end{quote}




A \dfn{matrix} is just a large rectangular array of numbers
\[
M =
\underset{\displaystyle\boldsymbol{5}~\textbf{columns}}{\begin{pmatrix}
  a_{1,1} & a_{1,2} & a_{1,3} & a_{1,4} & a_{1,5} \\
  a_{2,1} & a_{2,2} & a_{2,3} & a_{2,4} & a_{2,5} \\
  a_{3,1} & a_{3,2} & a_{3,3} & a_{1,4} & a_{3,5} \\
  a_{4,1} & a_{4,2} & a_{4,3} & a_{4,4} & a_{4,5}
\end{pmatrix}}
\boldsymbol{4}~\textbf {rows}
\]
We give the \dfn{dimensions of a matrix} by stating its number of rows
and number of columns. The number of rows comes first and the number
of columns second, so $M$ above is a $(4\times 5)$-matrix.
\begin{question}
  Which of the following matrices are $3\times 2$ matrices?
  \begin{selectAll}
  \choice{$\begin{pmatrix}
  2 & 4 & 8\\
  5 & 9 & -1\\
  6 & 0 & 1
    \end{pmatrix}$}
  \choice{$\begin{pmatrix}
  0 & 1 &3\\
  0 & 0 & 2
  \end{pmatrix}$}
  \choice[correct]{$\begin{pmatrix}
  2 & 1 \\
  4 & 5 \\
  -1 & 0
  \end{pmatrix}$}
  \choice[correct]{$\begin{pmatrix}
  0 & 0\\
  0 & 0 \\
  0 & 0
  \end{pmatrix}$}
  \end{selectAll}
\end{question}


We can use similar notation to talk about specific entries of a
matrix. Above, $a_{i,j}$ is the $\boldsymbol{(i,j)}${\bf-}\dfn{entry}
of the matrix $M$. Often people write $M_{i,j}$ (or just $M_{ij}$) to
mean the $(i,j)$-entry of the matrix $M$.



\begin{question}
  If
  \[A= \begin{pmatrix}
  -3 & 0 & 1\\
  4 & 5 & -2\\
  0 & 9 & -1
  \end{pmatrix}
  \]
  what are the entries $A_{1,2}$, $A_{2,2}$, $A_{3,2}$?
  \begin{prompt}
    \[
    A_{1,2} = \answer{0}, \quad A_{2,2} = \answer{5}, \quad A_{3,2} = \answer{9}
    \]
  \end{prompt}
\end{question}

Next we'll discuss how matrices arise in different contexts.


\section{Matrices store data}


A matrix can be thought of as a ``mathematical spreadsheet.'' With a
spreadsheet you have rows and columns of data.  The data we're
typically interested in comes in the form of vectors.  You can think
of a matrix as vectors stacked together either horizontally or
vertically.

\begin{example}[Population Counts] %https://worldpopulationreview.com/states/states-by-race
  In a previous example, we encoded the $2023$
  \link[demographics]{https://worldpopulationreview.com/states/states-by-race}~of
  the twelve Midwestern States as $6$-dimensional vectors represented
  as ordered tuples. Represent this data by a $12\times 6$
  matrix. Explain what the entry at position $(7,5)$ represents.
  \begin{explanation}
  Since ordered-tuples are \wordChoice{\choice[correct]{horizontal}\choice{vertical}} it makes sense to
  concatenate this data by stacking it horizontally into a matrix:
  \[
  \begin{pmatrix}
  \vec{p}_{\texttt{IA}} \\
  \vec{p}_{\texttt{IL}} \\
  \vec{p}_{\texttt{IN}} \\
  \vec{p}_{\texttt{KA}} \\
  \vec{p}_{\texttt{MI}} \\
  \vec{p}_{\texttt{MN}} \\
  \vec{p}_{\texttt{MO}} \\
  \vec{p}_{\texttt{ND}} \\
  \vec{p}_{\texttt{NE}} \\
  \vec{p}_{\texttt{OH}} \\
  \vec{p}_{\texttt{SD}} \\
  \vec{p}_{\texttt{WI}}
  \end{pmatrix}
  =
  \begin{pmatrix}
  2806418 & 117035 & 10538 & 79296 & 3941 & 132783\\
  8874067 & 1796660 & 33972 & 709567 & 5196 & 1296702\\
  5510354 & 631923 & 14030 & 158705 & 2205 & 379676\\
  2416165 & 165837 & 22278 & 87093 & 2344 & 218902\\
  7735902 & 1360149 & 50035 & 316844 & 3117 & 507860\\
  4572149 & 359817 & 54558 & 275242 & 2201 & 336199\\
  4978046 & 698043 & 24274 & 123810 & 8887 & 291100\\
  651470 & 23959 & 39165 & 11979 & 1004 & 32817\\
  1641256 & 91896 & 16875 & 47944 & 1235 & 124620\\
  9394878 & 1442655 & 20442 & 268527 & 3907 & 544866\\
  735228 & 18836 & 74975 & 12413 & 544 & 37340\\
  4895065 & 367889 & 48674 & 163396 & 2672 & 329279
  \end{pmatrix}
  \]
  Note that the above is a $\answer[given]{12}\times \answer[given]{6}$ matrix. The $(7, 5)$- entry is
  $\answer[given]{8887}$ which is the number of Hawaiians in Missouri (MO).
  \end{explanation}
\end{example}



\begin{example}[RGB Color Space]
  In a previous example we mentioned an LED table lamp whose shades
  vary linearly through the RGB color space every minute in this
  order:
      \[
    \underset{\text{red}}{\begin{pmatrix}255\\0\\0\end{pmatrix}},
    \underset{\text{yellow}}{\begin{pmatrix}255\\255\\0\end{pmatrix}},
    \underset{\text{green}}{\begin{pmatrix}0\\255\\0\end{pmatrix}},
    \underset{\text{cyan}}{\begin{pmatrix}0\\255\\255\end{pmatrix}},
    \underset{\text{blue}}{\begin{pmatrix}0\\0\\255\end{pmatrix}},
    \underset{\text{magenta}}{\begin{pmatrix}255\\0\\255\end{pmatrix}}
    \]
  Represent this sequence of RGB colors by a $3 \times 6$
  matrix. Interpret the meaning of the $(2,3)$-entry of the matrix.
  \begin{explanation}
    Since we are using column vectors, we'll concatenate this data by
    placing the vectors side-by-side:
    \[
      \begin{pmatrix}
        255 & 255 & 0 & 0 & 0 & 255\\
        0 & 255& 255 & 255 & 0 & 0 \\
        0 & 0 & 0 & 255 & 255 & 255
      \end{pmatrix}
    \]
    The $(2,3)$-entry is if the matrix is $\answer[given]{255}$. This
    represents a that
    \wordChoice{\choice{red}\choice[correct]{green}\choice{blue}}
    is at maximum intensity in that color at time $2$ minutes.
  \end{explanation}
\end{example}


\begin{example}[Grayscale Image]
  Here is a very low resolution image of the number $6$:
  \begin{center}  
\newcommand{\matrixData}{{
    {0, 0.01, 0, 0, 0, 0, 0},
    {0, 0, 0.06, 0.18, 0.18, 0.01, 0},
    {0, 0.38, 0.64, 0.21, 0.37, 0.79, 0.06},
    {0.12, 0.99, 0.12, 0.01, 0, 0.11, 0.02},
    {0.34, 0.94, 0.32, 0.41, 0.64, 0.55, 0.02},
    {0.35, 0.97, 0.09, 0, 0.05, 0.98, 0.43}, {0.14, 0.98, 0.13, 0, 0.01, 0.94, 
  0.39}, {0, 0.34, 0.56, 0.29, 0.47, 0.51, 0.01}, {0, 0, 0, 0.07, 
  0.03, 0, 0}, {0, 0.01, 0, 0, 0, 0.02, 0}
}}
    \begin{tikzpicture}[scale=.5]
      \foreach \y in {1, ..., 8} {
        \foreach \x in {0, 1, ..., 6} {
          \pgfmathsetmacro\val{\matrixData[\y][\x]}
          \fill[black, opacity=\val] (\x, -\y) rectangle (\x+1, -\y-1);
          \draw (\x, -\y) rectangle (\x+1, -\y-1);
        }
      }
    \end{tikzpicture}
  \end{center}
  We can represent this data as a grayscale image where $0$ represents ``black'' and $1$ represents ``white.''
  \[
  \begin{pmatrix}
     1 & 1 & 0.94 & 0.82 & 0.82 & 0.99 & 1 \\
     1 & 0.62 & 0.36 & 0.79 & 0.63 & 0.21 & 0.94 \\
     0.88 & 0.01 & 0.88 & 0.99 & 1 & 0.89 & 0.98 \\
     0.66 & 0.06 & 0.68 & 0.59 & 0.36 & 0.45 & 0.98 \\
     0.65 & 0.03 & 0.91 & 1 & 0.95 & 0.02 & 0.57 \\
     0.86 & 0.02 & 0.87 & 1 & 0.99 & 0.06 & 0.61 \\
     1 & 0.66 & 0.44 & 0.71 & 0.53 & 0.49 & 0.99 \\
     1 & 1 & 1 & 0.93 & 0.97 & 1 & 1 \\
  \end{pmatrix}
  \]
\end{example}













\section{Multiplying vectors and matrices}

Recall the dot product of two vectors:
\begin{align*}
  \begin{pmatrix}
    a_1\\
    a_2\\
    \vdots\\
    a_n
  \end{pmatrix}
  \dotp
  \begin{pmatrix}
    b_1\\
    b_2\\
    \vdots\\
    b_n
  \end{pmatrix}
  &=\begin{pmatrix} a_1 & a_2 & \cdots & a_n\end{pmatrix}\dotp\begin{pmatrix} b_1 & b_2 & \cdots & b_n\end{pmatrix}\\
  &= \sum_{i=1}^n a_ib_i\\
  &= a_1b_1 + a_2b_2 +\dots+a_nb_n
\end{align*}
We can extend this idea to multiplying matrices by vectors and to
multiplying vectors by matrices.

\paragraph{Multiplying a matrix by a column vector}

If I have a matrix $m\times n$ matrix $M$, and $\vec{a}$ is an
$n$-vector expressed as a column,
\[
M \vec{a} =
\begin{pmatrix}
  \row_1(M)^\transpose \dotp \vec{a} \\
  \row_2(M)^\transpose \dotp \vec{a} \\
  \vdots \\
  \row_m(M)^\transpose \dotp \vec{a}
\end{pmatrix}
\]
Where $\row_i(M)$ represents the $i$th row of $M$.

\paragraph{Multiplying a row vector by a matrix}
If on the other hand, $M$ is still a $m\times n$ matrix, but now
$\vec{b}$ is an $m$-vector expressed as a row,
\[
\vec{b} M =
\begin{pmatrix}
  \vec{b}\dotp \col_1(M)^\transpose \\
  \vec{b}\dotp \col_2(M)^\transpose \\
  \vdots \\
  \vec{b}\dotp \col_n(M)^\transpose
\end{pmatrix}
\]
Where $\col_i(M)$ represents the $i$th column of $M$.


\begin{question}
If
\[
A=\begin{pmatrix}
2 & 0 & -1\\
5 & 1 & -2\\
1 & 0 & 4
\end{pmatrix} \quad\text{and}\quad  \vec{v} = \begin{pmatrix}
1 \\
0\\
2
\end{pmatrix}
\]
compute the following, if possible.
\begin{prompt} Write ``NA'' for each entry if the multiplication is not possible.\end{prompt}
\begin{enumerate}
\item $\vec{v} A$ \begin{prompt}
  \[=\begin{pmatrix}
  \answer[given,format=string]{NA}\\ \answer[given,format=string]{NA}\\ \answer[given,format=string]{NA}
\end{pmatrix}\]
\end{prompt}
\item $A\vec{v}$ \begin{prompt} \[= \begin{pmatrix}
\answer[given]{0}\\
\answer[given]{-4}\\
\answer[given]{9}
\end{pmatrix}\]\end{prompt}
\item $\vec{v}^{\transpose} A$ \begin{prompt} \[= \begin{pmatrix}
\answer[given]{2} & \answer[given]{0} & \answer[given]{7}
  \end{pmatrix}
  \]
\end{prompt}
\item $A\vec{v}^{\transpose}$ \begin{prompt} \[= \left(\begin{array}{ccc}
\answer[given,format= string]{NA} & \answer[given,format= string]{NA} & \answer[given,format= string]{NA}
\end{array}\right)\]
\end{prompt}
\end{enumerate}
\end{question}









\section{Systems of equations}



One important use of matrices is to solve systems of equations. By matrix multiplication, one can check that
\begin{eqnarray*}
4x+2y+z &=& 5\\
2x+3y &=& 6\\
5y -z &=& 8
\end{eqnarray*}
can be written as
\[
\left(\begin{array}{ccc}
4 & 2 & 1 \\
2 & 3 & 0\\
0 & 5 & -1
\end{array}\right) \begin{pmatrix}
x \\
y\\
z
\end{pmatrix}
= \begin{pmatrix}
5 \\
6\\
8
\end{pmatrix}
\]
Multiplying the left-hand side above we have,
\[
\begin{pmatrix}
4x+2y+z\\
2x+3y\\
5y-z
\end{pmatrix} = \begin{pmatrix}
5\\
6\\
8
\end{pmatrix}
\]
giving the system of equations. Note that for two matrices $A$ and $B$, we have $A=B$ if both matrices are of the same size and each entry in $A$ matches to that of $B$.
\begin{question}
Write the system of equations below in matrix form:
\begin{eqnarray*}
3x-7y+8w &=& 1\\
2z -10w &=& 9\\
5x-2z+3y &=& 4\\
x-y+z &=& 8
\end{eqnarray*}
\begin{prompt}
\[
\left(\begin{array}{cccc}
\answer[given]{3} & \answer[given]{-7} & \answer[given]{0} & \answer[given]{8}\\
\answer[given]{0} & \answer[given]{0} & \answer[given]{2} & \answer[given]{-10}\\
\answer[given]{5} & \answer[given]{3} & \answer[given]{-2} & \answer[given]{0}\\
\answer[given]{1} & \answer[given]{-1} & \answer[given]{1} & \answer[given]{0}
\end{array}\right) \begin{pmatrix}
x\\
y\\
z\\
w
\end{pmatrix} = \begin{pmatrix}
\answer[given]{1}\\
\answer[given]{9}\\
\answer[given]{4}\\
\answer[given]{8}
\end{pmatrix}
\]
\end{prompt}
\end{question}

\begin{question}
Solve for $x, y, z, w$:
\[\left(\begin{array}{cc}
2x+4 & 5y \\
z^2-8 & 8
\end{array}\right) = \left(\begin{array}{cc}
6x & 10\\
8 & 5w
\end{array}\right)
\]
\begin{prompt}
\[
x= \answer[given]{1}, \hspace{.4cm} y = \answer[given]{2}, \hspace{.4cm} z = \answer[given]{8}, \hspace{.4cm} w = \answer[given]{8/5}
\]
\end{prompt}
\end{question}

To avoid writing the variables and to make the calculations easier to carry we write the condensed matrix called augmented matrix as follows, where the constants on the right hand side of the equations are separated by a line:
\[
\left(\begin{array}{ccc|c}
4 & 2 & 1 & 5\\
2 & 3 & 0 & 6\\
0 & 5 & -1 & 8
\end{array}\right)\]

We often use this form to solve the system of equations. We will discuss this procedure in the next chapter.
\begin{question}
  Write the augmented matrix for the following system:
  \begin{eqnarray*}
  2x-5y+8z &=& 1\\
  3y -4z &=& 9\\
  6x -7z &=& 10
  \end{eqnarray*}
  \begin{prompt}
  \[\left(\begin{array}{ccc|c}
  \answer[given]{2} &  \answer[given]{-5} &  \answer[given]{8} & \answer[given]{1}\\
   \answer[given]{0} &  \answer[given]{3} &  \answer[given]{-4} &  \answer[given]{9}\\
   \answer[given]{6} &  \answer[given]{0} &  \answer[given]{-7} &  \answer[given]{10}
  \end{array}\right)
  \]
  \end{prompt}
  \end{question}



\begin{example}[Networks]
In electrical engineering, networks are often used to analyze a system. A network is composed of junctions and branches and each junction can be described by an equation by equating the branches going in to branches going out. That is input = output. For example,\\

......\\

gives:
\begin{eqnarray*}
20 &=& x_{1} + 2x_{2} + 30\\
x_{1} + 40 &=& x_{3}\\
2x_{2} + x_{3} + 30 &=& 40
\end{eqnarray*}
leading to:
\begin{eqnarray*}
x_{1}+2x_{2} &=& -10\\
x_{1} -x_{3}&=& -40\\
2x_{2} + x_{3}  &=& 10
\end{eqnarray*}
with augmented matrix:
\[\left(\begin{array}{ccc|c}
1 & 2 & 0 & -10\\
1 & 0 & -1 & -40\\
1 & 2 & 1 & 10
\end{array}\right)\]
\end{example}





\begin{example}[Population Counts]
Extract a new data set from a matrix using basis vectors
\end{example}


\begin{example}[RGB Color Space]
\end{example}





\section{Multiplying matrices}

Matrix multiplication may seem strange at first. In fact, only certain
matrices can be multipled togther. As a general rule, we can only
multiply a $m \times n$ matrix by a $k \times \l$ matrix matrix if
$n=k$. That is if \textbf{the middle two numbers match}:
\[
\begin{matrix}
\begin{pmatrix}
    \bullet & \bullet & \bullet & \bullet & \bullet\\
    \bullet & \bullet & \bullet & \bullet & \bullet\\
\end{pmatrix}
&
\begin{pmatrix}
    \bullet & \bullet &\bullet \\\bullet & \bullet & \bullet \\  \bullet & \bullet & \bullet \\\bullet & \bullet & \bullet \\ \bullet & \bullet & \bullet \end{pmatrix} &= &
\begin{pmatrix}
  \bullet & \bullet &\bullet \\\bullet & \bullet &  \bullet
\end{pmatrix}\\
\begin{pmatrix}
  m\times n\\
  \text{matrix}
\end{pmatrix} &
\begin{pmatrix}
  n\times \l\\
  \text{matrix}
\end{pmatrix} 
&= & \begin{pmatrix}
  m\times \l\\
  \text{matrix}
\end{pmatrix}
\end{matrix}
\]
The product then turns out to be a matrix of size $m\times \ell$, \textbf{the
outside numbers}.

\begin{question}
  Given three matrices 
  \[
  A =,\quad B =, \quad C =\quad 
  \]
  which of the following make sense?
  \begin{selectAll}
    \choice{$AB$}
    \choice{$AC$}
  \end{selectAll}
  What are the dimensions of the product?
\end{question}


We'll give several different explanations and interpretations of this
computation.

\paragraph{As an extension of the dot product}
We can extend the concept of the dot product to explain how to
multiply a matrix by another matrix. Here is an example:
\begin{align*}AB &= \begin{pmatrix}
a_{1,1} & a_{1,2}\\
a_{2,1} & a_{2,2}
\end{pmatrix}
\begin{pmatrix}
b_{1,1} & b_{1,2}\\
b_{2,1} & b_{2,2}
\end{pmatrix}\\
&= \begin{pmatrix}
\row_1(A)^\transpose \dotp\col_1(B) & \row_1(A)^\transpose\dotp\col_2(B)\\
\row_2(A)^\transpose \dotp\col_1(B) & \row_2(A)^\transpose\dotp\col_2(B)
\end{pmatrix}\\
&= \begin{pmatrix}
a_{1,1}b_{1,1} + a_{1,2}b_{2,1} & a_{1,1}b_{1,2}+ a_{1,2}b_{2,2}\\
a_{2,1}b_{1,1} + a_{2,2}b_{2,1} & a_{2,1}b_{1,2} + a_{2,2}b_{2,2}
\end{pmatrix}
\end{align*}
Quite generally, if $A$ is an $m\times n$ matrix and $B$ is a $n\times
\l$ matrix, and $C = AB$ is a $m\times \l$ matrix, then
\[
C_{i,j} = \row_i(A)^\transpose\dotp \col_j(B).
\]

\paragraph{Columns of the product}
\begin{align*}AB &= \begin{pmatrix}
a_{1,1} & a_{1,2}\\
a_{2,1} & a_{2,2}
\end{pmatrix}
\begin{pmatrix}
b_{1,1} & b_{1,2}\\
b_{2,1} & b_{2,2}
\end{pmatrix}\\
&= \begin{pmatrix}
\col_1(A)b_{1,1} + \col_2(A) b_{2,1} & \col_1(A)b_{1,2} + \col_2(A) b_{2,2}
\end{pmatrix}\\
&= \begin{pmatrix}
\begin{pmatrix}a_{1,1} \\ a_{2,1}\end{pmatrix}b_{1,1} + \begin{pmatrix}a_{1,2} \\ a_{2,2}\end{pmatrix}b_{2,1} & \begin{pmatrix}a_{1,1} \\ a_{2,1}\end{pmatrix}b_{1,2} + \begin{pmatrix}a_{1,2} \\ a_{2,2}\end{pmatrix}b_{2,2}
\end{pmatrix}\\
&= \begin{pmatrix}
a_{1,1}b_{1,1} + a_{1,2}b_{2,1} & a_{1,1}b_{1,2}+ a_{1,2}b_{2,2}\\
a_{2,1}b_{1,1} + a_{2,2}b_{2,1} & a_{2,1}b_{1,2} + a_{2,2}b_{2,2}
\end{pmatrix}
\end{align*}

Quite generally, if $A$ is an $m\times n$ matrix and $B$ is a $n\times
\l$ matrix, and $C = AB$ is a $m\times \l$ matrix, then
\[
\col_i(C) = \sum_{k=1}^n \col_k(A)b_{k,i}
\]
that is,
\begin{align*}
  C &= \begin{pmatrix} \col_1(C) & \col_2 (C) & \cdots & \col_{\l}(C) \end{pmatrix}\\
  &=  \begin{pmatrix} \sum_{k=1}^n \col_k(A)b_{k,1} & \sum_{k=1}^n \col_k(A)b_{k,2} & \cdots & \sum_{k=1}^n \col_k(A)b_{k,\l}\end{pmatrix}
\end{align*}


\paragraph{Rows of the product}


\begin{align*}AB &= \begin{pmatrix}
a_{1,1} & a_{1,2}\\
a_{2,1} & a_{2,2}
\end{pmatrix}
\begin{pmatrix}
b_{1,1} & b_{1,2}\\
b_{2,1} & b_{2,2}
\end{pmatrix}\\
&= \begin{pmatrix}
a_{1,1}\row_1(B) + a_{1,2}\row_2(B) \\ a_{2,1}\row_1(B) +  a_{2,2}\row_2(B)
\end{pmatrix}\\
&= \begin{pmatrix}
a_{1,1}\begin{pmatrix} b_{1,1} & b_{1,2}\end{pmatrix} + a_{1,2}\begin{pmatrix} b_{2,1} & b_{2,2}\end{pmatrix} \\ a_{2,1}\begin{pmatrix} b_{1,1} & b_{1,2}\end{pmatrix} +  a_{2,2}\begin{pmatrix} b_{2,1} & b_{2,2}\end{pmatrix}
\end{pmatrix}\\
&= \begin{pmatrix}
a_{1,1}b_{1,1} + a_{1,2}b_{2,1} & a_{1,1}b_{1,2}+ a_{1,2}b_{2,2}\\
a_{2,1}b_{1,1} + a_{2,2}b_{2,1} & a_{2,1}b_{1,2} + a_{2,2}b_{2,2}
\end{pmatrix}
\end{align*}

Quite generally, if $A$ is an $m\times n$ matrix and $B$ is a $n\times
\l$ matrix, and $C = AB$ is a $m\times \l$ matrix, then
\[
\row_i(C) = \sum_{k=1}^n a_{i,k}\row_k(B)
\]
that is,
\begin{align*}
  C &= \begin{pmatrix} \row_1(C) \\ \row_2 (C) \\ \vdots \\ \row_{m}(C) \end{pmatrix}\\
  &=  \begin{pmatrix} \sum_{k=1}^n a_{1,k}\row_k(B) \\ \sum_{k=1}^n a_{2,k}\row_k(B) \\ \vdots \\ \sum_{k=1}^n a_{m,k}\row_k(B)\end{pmatrix}
\end{align*}















\begin{question}
For the following matrices, find $AB$ and $BA$.
\[
A= \left(\begin{array}{cc}
5 & 0 \\
2 & 1
\end{array}\right), \hspace{.5cm} B = \left(\begin{array}{cc}
3 & -1 \\
2 & 0
\end{array}\right)
\]
\begin{prompt}
\[
AB = \left(\begin{array}{cc}
\answer[given]{15} & \answer[given]{-5}\\
\answer[given]{8} & \answer[given]{-2}
\end{array}\right)\]

\[BA = \left(\begin{array}{cc}
\answer[given]{13} & \answer[given]{-1}\\
\answer[given]{10} & \answer[given]{0}
\end{array}\right)\]
\end{prompt}
Do the two multiplications give the same result? \wordChoice{\choice{yes} \choice[correct]{no}}
\end{question}











\section{Matrices transform data}


\begin{example}[Financial Transactions]
  Recall that a bookstore sells a variety of books with each type of
  book sold in one month stored as a row vector
  \[
  \vec{s} = \begin{pmatrix}141 & 304 & 249 & 199 & 251 \end{pmatrix}
  \]
  with the entries representing the categories: Science Fiction,
  Fantasy, Mystery, Romance, Historical in that order.  Suppose last
  month the book store bought $200$ Science Fiction, $400$ Fantasy, $300$
  Mystery, $250$ Romance, $300$ Historical books. What percentage of each
  category was sold this month?
  \begin{explanation}
    Let
    \[
    T =
    \begin{pmatrix}
      100/200 & 0 &    0   &   0    &   0 \\
      0 & 100/400 &    0   &   0    &   0 \\
      0 &   0   &  100/300 &   0    &   0 \\
      0 &   0   &    0   & 100/250  &   0 \\
      0 &   0   &    0   &   0    & 100/300
    \end{pmatrix}
    \]
    Each entry of the matrix $T$ is of the form:
    \[
    100 \cdot \frac{1}{(\text{Stock})}
    \]
    where $(\text{Stock})$ represents the number of books the
    bookstore bought to sell.  We can solve our problem by computing
    \[
    \vec{s} T =
    \]
    So we see
  \end{explanation}

\end{example}




\begin{example}[Navigation]
  Scale/normalize fincinal data etc.
\end{example}
























































\section{Types of Matrices}
In Chapter One we introduced the transpose of a vector:
\[
\vec{u} = \left(\begin{array}{ccc}
1 & -1 & 0
\end{array}\right)
\hspace{.4cm}\Rightarrow\hspace{.4cm}
\vec{u}^{\transpose} = \begin{pmatrix}
1\\
-1\\
0
\end{pmatrix}
\]
So transpose switched a $\left(1\times 3\right)$- matrix to a $\left(3\times 1\right)$- matrix. The same concept applies to any larger matrix. The transpose operation switches all the rows to columns:
\[
A = \left(\begin{array}{ccc}
2 & -1 & 0\\
-3 & 4 & 5\\
1 & -2 & 0
\end{array}\right)
\hspace{.4cm}\Rightarrow\hspace{.4cm}
A^{\transpose} = \left(\begin{array}{ccc}
2 & -3 & 1\\
-1 & 4 & -2\\
0& 5 & 0
\end{array}\right)
\]
Thus, the first row of $A$ becomes the first column of $A^{\transpose}$ and the second row of $A$ becomes the second column of $A^{\transpose}$ and so forth.
\begin{question}
If
\[A= \left(\begin{array}{ccc}
-3 & 0 & 1\\
5 & 8 & 10
\end{array}\right), \hspace{.5cm} B = \left(\begin{array}{ccc}
7 & -2 & -1 \\
0 & 1 & 9 \\
5 & 4 & 2
\end{array}\right)\]
Find $A^{\transpose}$ and $B^{\transpose}$.

\begin{prompt}
\[A^{T} = \left(\begin{array}{cc}
\answer[given]{-3} & \answer[given]{5}\\
\answer[given]{0} & \answer[given]{8}\\
\answer[given]{1} & \answer[given]{10}
\end{array}\right), \hspace{.5cm} B^{\transpose}= \left(\begin{array}{ccc}
\answer[given]{7} & \answer[given]{0} & \answer[given]{5}\\
\answer[given]{-2} & \answer[given]{1} & \answer[given]{4}\\
\answer[given]{-1} & \answer[given]{9} & \answer[given]{2}
\end{array}\right)\]
\end{prompt}
\end{question}

\begin{question}
If
\[
A= \left(\begin{array}{ccc}
-3 & 0 & 1\\
4 & 0 & 2\\
-1 & 5 & 0
\end{array}\right), \hspace{.5cm}
B = \left(\begin{array}{ccc}
1 & -1 & 0\\
0 & 2 & 1\\
-3 & 4 & 0
\end{array}\right)\]
Compute the following. \\

a. $2A^{\transpose}$ \begin{prompt} \[= \left(\begin{array}{ccc}
\answer[given]{-6} & \answer[given]{8}& \answer[given]{-2}\\
\answer[given]{0} & \answer[given]{0}& \answer[given]{10}\\
\answer[given]{2} & \answer[given]{4}& \answer[given]{0}
\end{array}\right)\]\end{prompt}\\

b. $AB^{\transpose}$ \begin{prompt} \[= \left(\begin{array}{ccc}
\answer[given]{-3} & \answer[given]{1}& \answer[given]{9}\\
\answer[given]{4} & \answer[given]{2}& \answer[given]{-12}\\
\answer[given]{-6} & \answer[given]{10}& \answer[given]{23}
\end{array}\right)\]\end{prompt}\\

c. $A^{\transpose} + AB$ \begin{prompt} \[=\left(\begin{array}{ccc}
\answer[given]{-9} & \answer[given]{11}& \answer[given]{-1}\\
\answer[given]{-2} & \answer[given]{4}& \answer[given]{5}\\
\answer[given]{0} & \answer[given]{13}& \answer[given]{5}
\end{array}\right)\]\end{prompt}
\end{question}

As in vectors, for any matrices $A$ and $B$, $\left(A^{\transpose}\right)^{\transpose} = A$ and $\left(A+ B\right)^{\transpose} = A^{\transpose} + B^{\transpose}$. Another property of transpose operation which is often used is
\[
\left(AB\right)^{\transpose} = B^{\transpose} A^{\transpose}
\]
Thus, their order changes after taking the transpose of each.
\begin{question}
Let
\[
A= \left(\begin{array}{cc}
2 & -1 \\
3 & 4
\end{array}\right), \hspace{.5cm} B = \left(\begin{array}{cc}
7 & 5 \\
0 & 1
\end{array}\right)
\]
Find $\left(AB\right)^{\transpose}$ and compare it with $B^{\transpose}A^{\transpose}$.

\begin{prompt}
\[
AB = \left(\begin{array}{cc}
\answer[given]{14} & \answer[given]{9}\\
\answer[given]{21} & \answer[given]{19}
\end{array}\right)
\hspace{.4cm} \Rightarrow \hspace{.4cm}
\left(AB\right)^{\transpose} = \left(\begin{array}{cc}
\answer[given]{14} & \answer[given]{21}\\
\answer[given]{9} & \answer[given]{19}
\end{array}\right)
\]
\[
A^{\transpose}= \left(\begin{array}{cc}
\answer[given]{2} & \answer[given]{3}\\
\answer[given]{-1} & \answer[given]{4}
\end{array}\right), \hspace{.5cm}
B^{\transpose} = \left(\begin{array}{cc}
\answer[given]{7} & \answer[given]{0}\\
\answer[given]{5} & \answer[given]{1}
\end{array}\right)
\hspace{.4cm} \Rightarrow \hspace{.4cm}
B^{\transpose} A^{\transpose}  = \left(\begin{array}{cc}
\answer[given]{14} & \answer[given]{21}\\
\answer[given]{9} & \answer[given]{19}
\end{array}\right)
\]
\end{prompt}
\end{question}
Let us explore some types of matrices that usually are used in linear algebra. The zero matrix, usually denoted as $\textbf{O}_{n\times m}$ is an $n\times m$ matrix with each of its entries being zero. For example,
\[
\textbf{O}_{2\times 3} = \left(\begin{array}{ccc}
0 & 0 &0 \\
0 & 0 & 0
\end{array}\right), \hspace{.5cm} \textbf{O}_{4\times 2} = \left(\begin{array}{cc}
0 & 0 \\
0 & 0 \\
0 & 0\\
0 & 0
\end{array}\right)\]
It is not hard to see that the zero matrix multiplied by a matrix gives a zero matrix:
\[
\left(\begin{array}{ccc}
2 & -1 & 0 \\
3 & 4 & 3\\
1 & -2 & 7
\end{array}\right) \left(\begin{array}{ccc}
0 &0 &0 \\
0 &0 &0 \\
0 &0 &0
\end{array}\right) = \left(\begin{array}{ccc}
0 &0 &0 \\
0 &0 &0 \\
0 &0 &0
\end{array}\right)
\]
and $\textbf{O}_{3\times 3} A = \textbf{O}_{3 \times 3}$. Another important type of matrix is the identity matrix, denoted as $I_{n\times m}$. It is a matrix whose diagonal entries are $1$ and its every other entry is zero. For example,
\[ I_{2\times 2}=
\left(\begin{array}{cc}
1 & 0 \\
0 & 1
\end{array}\right), \hspace{.5cm}  I_{4\times 4} = \left(\begin{array}{cccc}
1 & 0 & 0 & 0 \\
0 & 1 & 0& 0 \\
0& 0 & 1 & 0\\
0 & 0 & 0 & 1
\end{array}\right)
\]
Note that the identity matrix is a square matrix ( has the same number of rows as columns). The fundamental property of the identity matrix is that is acts as $1$ as in regular multiplication. That is, $AI = A$ and $IA= A$. For example, one can check that
\[
\left(\begin{array}{cc}
a_{1} & a_{2}\\
a_{3} & a_{4}
\end{array}\right) \left(\begin{array}{cc}
1 & 0\\
0 &1
\end{array}\right) = \left(\begin{array}{cc}
a_{1} & a_{2}\\
a_{3} & a_{4}
\end{array}\right)
\]
and
\[\left(\begin{array}{cc}
1 & 0\\
0 &1
\end{array}\right)\left(\begin{array}{cc}
a_{1} & a_{2}\\
a_{3} & a_{4}
\end{array}\right)= \left(\begin{array}{cc}
a_{1} & a_{2}\\
a_{3} & a_{4}
\end{array}\right) \]

\begin{question}
Write a $3 \times 3$ identity matrix and show that $AI = IA = A$ for
\[A =  \left(\begin{array}{ccc}
2 & 3 & 4\\
0  & 1 & -1\\
5 & 7 & 9
\end{array}\right)
\]

\begin{prompt}
\[
I_{3\times 3} = \left(\begin{array}{ccc}
\answer[given]{1} & \answer[given]{0} & \answer[given]{0}\\
\answer[given]{0} & \answer[given]{1} & \answer[given]{0}\\
\answer[given]{0} & \answer[given]{0} & \answer[given]{1}
\end{array}\right)
\]
Then,
\[
AI = \left(\begin{array}{ccc}
2 & 3 & 4\\
0  & 1 & -1\\
5 & 7 & 9
\end{array}\right)\left(\begin{array}{ccc}
\answer[given]{1} & \answer[given]{0} & \answer[given]{0}\\
\answer[given]{0} & \answer[given]{1} & \answer[given]{0}\\
\answer[given]{0} & \answer[given]{0} & \answer[given]{1}
\end{array}\right)= \left(\begin{array}{ccc}
\answer[given]{2} & \answer[given]{3} & \answer[given]{4}\\
\answer[given]{0} & \answer[given]{1} & \answer[given]{-1}\\
\answer[given]{5} & \answer[given]{7} & \answer[given]{9}
\end{array}\right)\]
and
\[
IA = \left(\begin{array}{ccc}
\answer[given]{1} & \answer[given]{0} & \answer[given]{0}\\
\answer[given]{0} & \answer[given]{1} & \answer[given]{0}\\
\answer[given]{0} & \answer[given]{0} & \answer[given]{1}
\end{array}\right)\left(\begin{array}{ccc}
2 & 3 & 4\\
0  & 1 & -1\\
5 & 7 & 9
\end{array}\right)= \left(\begin{array}{ccc}
\answer[given]{2} & \answer[given]{3} & \answer[given]{4}\\
\answer[given]{0} & \answer[given]{1} & \answer[given]{-1}\\
\answer[given]{5} & \answer[given]{7} & \answer[given]{9}
\end{array}\right)\]
\end{prompt}
\end{question}













For some interesting extra reading check out:
\begin{itemize}
\item \link[\textit{Earliest Uses of Symbols for Matrices and Vectors},  MacTutor History of Mathematics, (University of St Andrews, Scotland, February 2000.]{https://mathshistory.st-andrews.ac.uk/Miller/mathsym/matrices/}
\end{itemize}



\end{document}
