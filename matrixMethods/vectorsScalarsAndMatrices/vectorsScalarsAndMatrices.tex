\documentclass{ximera}

\newcommand{\dfn}{\textbf}
\renewcommand{\vec}[1]{{\overset{\boldsymbol{\rightharpoonup}}{\mathbf{#1}}}\hspace{0in}}
%% Simple horiz vectors
\renewcommand{\vector}[1]{\left\langle #1\right\rangle}
\newcommand{\arrowvec}[1]{{\overset{\rightharpoonup}{#1}}}
\newcommand{\R}{\mathbb{R}}
\newcommand{\transpose}{\intercal}
\newcommand{\ro}{\texttt{R}}%% row operation
\newcommand{\dotp}{\bullet}%% dot product
\renewcommand{\l}{\ell}
\let\defaultAnswerFormat\answerFormatBoxed
\usetikzlibrary{calc,bending}
\tikzset{>=stealth}


\usepackage{mdframed} % For framing content
%\usepackage{ifthen}   % For conditional statements

% Define the 'concept' environment with an optional header
\newenvironment{concept}[1][]{%
  \begin{mdframed}[linecolor=black, linewidth=2pt, innertopmargin=5pt, innerbottommargin=5pt, skipabove=12pt, skipbelow=12pt]%
    \noindent\large\textbf{#1}\normalsize%
}{%
  \end{mdframed}%
}



%% Define exercise collection. 
\makeatletter
\newcommand{\exerciseCollection}[2]{
\def\input@path{{#1}}
\activity{#1#2}}



\newcommand{\practicestyle}{
  
  \let\exercise\relax
\let\endexercise\relax
\let\c@exercise\relax
\let\problem\relax
\let\endproblem\relax
\let\c@problem\relax 
\newtheoremstyle{problem}
{\topsep}{\topsep}{\rmfamily}{}{\bfseries}{)}{ }{}
\theoremstyle{problem}
\newtheorem{problem}{}
\newtheorem{exercise}[problem]{}
\section{Exercises for Chapter~\thetitlenumber}
\small%\twocolumn
%% \let\exercise\relax
%%   \let\endexercise\relax
%%   \let\c@exercise\relax 
%%   \newtheoremstyle{exercise}
%%                   {\topsep}{\topsep}{\rmfamily}{}{\bfseries}{)}{~}{}
%% \theoremstyle{exercise}
%% \newtheorem{exercise}{}
}
\renewcommand\chapterstyle{%
  \def\activitystyle{activity-chapter}
  \normalsize
  %\onecolumn
  \def\maketitle{%
    \addtocounter{titlenumber}{1}%
                    {\flushleft\small\sffamily\bfseries\@pretitle\par\vspace{-1.5em}}%
                    {\flushleft\LARGE\sffamily\bfseries\thetitlenumber\hspace{1em}\@title \par }%
                    {\vskip .6em\noindent\textit\theabstract\setcounter{problem}{0}\setcounter{section}{0}}%
                    \par\vspace{2em}
                    \phantomsection\addcontentsline{toc}{section}{\textbf{\thetitlenumber\hspace{1em}\@title}}%
}}
\makeatother







%% \colorlet{textColor}{black}
%% \colorlet{background}{white}
%% \colorlet{penColor}{blue!50!black} % Color of a curve in a plot
%% \colorlet{penColor2}{red!50!black}% Color of a curve in a plot
%% \colorlet{penColor3}{red!50!blue} % Color of a curve in a plot
%% \colorlet{penColor4}{green!50!black} % Color of a curve in a plot
%% \colorlet{penColor5}{orange!80!black} % Color of a curve in a plot
%% \colorlet{penColor6}{yellow!70!black} % Color of a curve in a plot
%% \colorlet{fill1}{penColor!20} % Color of fill in a plot
%% \colorlet{fill2}{penColor2!20} % Color of fill in a plot
%% \colorlet{fillp}{fill1} % Color of positive area
%% \colorlet{filln}{penColor2!20} % Color of negative area
%% \colorlet{fill3}{penColor3!20} % Fill
%% \colorlet{fill4}{penColor4!20} % Fill
%% \colorlet{fill5}{penColor5!20} % Fill
%% \colorlet{gridColor}{gray!50} % Color of grid in a plot


\author{Bart Snapp}
%%OpenAI. (2023). ChatGPT (Mar 14 version) [Large language model]. https://chat.openai.com/chat


\title{Vectors, scalars, and matrices}


\begin{document}
\begin{abstract}
  A concrete introduction to vectors and matrices, with an informal
  view towards vector spaces and linear transformations.
\end{abstract}
\maketitle


\section{Vectors and scalars}

In mathematics, we study (among other things) numbers. In many cases,
these numbers represent real-world data. Sometimes it ``makes sense''
to add data and sometimes it doesn't.
\paragraph{When it does not make sense to add data}
\begin{description}
\item[Personal Identification Numbers] Summing things like social
  security numbers, phone numbers, or zip codes doesn't provide
  meaningful information.
\item[Temperatures] While you can mathematically add temperatures
  together, doing so usually doesn't provide useful information.
\item[Time] Adding specific points in time, like dates or hours of the
  day, usually doesn't yield meaningful results. 
\item[Geographic Coordinates] Adding the latitude and longitude of two
  locations doesn't give you a location that has any real-world
  significance.
\end{description}
For some of the categories above, the \textit{average} can be
meaningful; or perhaps if each quantity is thought of as
`displacement' or `duration,' summing might be meaningful. However,
without additional (pun intended!) stipulations, summing the types of
quantities above is not meaningful. On the other hand, there's lots of
times that it makes sense to add data:
\paragraph{When it makes sense to add data}
\begin{description}
\item[Population Counts] Adding the populations of different regions,
  cities, or countries to find the total population of a larger
  area. Answers to questions like these help us understand our
  society and are important to many people.
\item[Financial Transactions] Summing daily sales to find total
  monthly sales, or adding up all expenses to find total costs. These
  are real numbers that represent actual amounts of money, and their
  total gives meaningful information about financial status or
  performance.
\item[Time Spent on Tasks] Adding the duration of time spent on
  various activities on day gives a total time spent. We all have busy
  lives, and time is a very precious commodity. Hence, it is good to
  understand how we spend our time.
\item[Distances] (BADBAD needs to be rewritten in light of examples below)If you walk $3$ miles in the morning and another $2$
  miles in the evening, adding these distances gives a meaningful
  total of $5$ miles walked in a day.
\end{description}
Often when it make sense to \textit{add} quantities, it make sense to
\textit{scale} them as well. For example with population, you could
ask for the population of a city with $3$ times the population, or
half the population. Similar \textit{scaling} examples exist for all
the examples above where it make sense to add data. Regular old
numbers used to scale vectors are called \textit{scalars}.


\begin{definition}
  We say data represented by \dfn{vectors} $\vec{v}$ and $\vec{w}$ if
  \[
  s\vec{v}\quad\text{and}\quad s\vec{w}
  \]
  are also vectors (encode `meaningful' data in the same way that
  $\vec{v}$ and $\vec{w}$ do) and for all \textbf{scalars} $s$ (where $s$ is
  usually a real number),
  \[
  s(\vec{v} + \vec{w}) = s\vec{v}+ s\vec{w}.
  \]
\end{definition}
This might all sound very abstract. Let's give some very concrete
examples, based on the examples above. Before we start, we need to
give some notation for vectors.

\paragraph{Ways data can be stored as a vector}
\begin{description}
\item[As an ordered tuple] An ordered pair is just a pair of numbers
  $(a,b)$ delineated by parenthesis with the entries separated by a
  comma. It's called ``ordered'' because $(a,b) \ne (b,a)$. An \textit{ordered
  tuple} is just a list of an arbitrary, but fixed, number of elements
  that is ordered like an ordered pair. As an example, we could
  represent the \link[demographic information]{https://worldpopulationreview.com/states/states-by-race} for Ohio as an vector
  represented by an ordered tuple:
  \[
  \vec{p}_{\texttt{OH}} = (\underset{\text{White}}{9394878},\underset{\text{Black}}{1442655},\underset{\text{American Indian}}{20442},\underset{\text{Asian}}{268527},\underset{\text{Hawaiian}}{3907},\underset{\text{Other}}{544866})
  \]
\item[As a row vector] A row vector is a lot like an ordered tuple,
  except we do not use a comma to separate entries. Instead, we use a
  space.  Suppose a bookstore sells a variety of books in a month, say
  199 Science Fiction, 249 Fantasy, 141 Mystery, 304 Romance, 251
  Historical. We can express this data as a row vector:
  \[
  \vec{s} = \begin{pmatrix}199 & 249 & 141 & 304 & 251 \end{pmatrix} 
  \]
\item[As a column vector] A column vector is just like a row vector,
  except is it written vertically:
  \[
  \vec{w}_{\texttt{TR}} = \begin{pmatrix}
    0.5\\ 4 \\ 0 \\ 1\end{pmatrix}
    \qquad
    \begin{array}{l}
    \text{Emails}\\
    \text{Classes}\\
    \text{Project}\\
    \text{Exercising}
  \end{array}
  \]
  Above, we could suppose that $\vec{w}_{\texttt{TR}}$ represents the
  fact that on Tuesdays and Thursdays, someone spends $0.5$ hours on
  emails, $4$ hours in class, $0$ hours on projects, and $1$ hour
  exercising.
\end{description}


\begin{definition}
  To switch a row vector into a column vector and vice versa, we use the \dfn{transpose} operation:
  \[
  \begin{pmatrix} 1 &  2 & 3 \end{pmatrix}^\transpose =
  \begin{pmatrix} 1 \\ 2 \\ 3 \end{pmatrix}
  \quad\text{and}\quad
  \begin{pmatrix} 1 \\ 2 \\ 3 \end{pmatrix}^\transpose =
  \begin{pmatrix} 1 &  2 & 3 \end{pmatrix}
  \]
\end{definition}

Once we have data represented as vectors encoded as ordered tuples,
row vectors, or column vectors, we can describe some general
information about the vectors. In particular, we can think about their
\textit{dimension} and their \textit{components}.

\begin{definition}
The \dfn{dimension} of a vector is the number of entries. Each
individual entry of a vector is called a \dfn{component}.
\end{definition}

\begin{question}
  What is the dimension of the vector $\vec{p}_{\texttt{OH}}$?
  \begin{prompt}
  \[
  \text{Dimension} = \answer{6}
  \]
  \end{prompt}
\end{question}


\begin{question} %% REWRITE IN ONE OF THE CONTEXTS ABOVE
  What is the dimension of the vector 
  \[
  \vector{3,4,1,-4}?
  \]
  \begin{prompt}
  \[
  \text{Dimension} = \answer{4}
  \]
  \end{prompt}
  \begin{question}
    What are the components of the vector $\vector{1,4,-3}$?
    \begin{prompt}
      \begin{itemize}
      \item The $x$-component is $\answer{1}$.
      \item The $y$-component is $\answer{4}$.
      \item The $z$-component is $\answer{-3}$.
      \end{itemize}
    \end{prompt}
  \end{question}
\end{question}

When we express vectors as ordered tuples, row vectors, or column
vectors we add them in a componentwise fashion.
\begin{align*}
  (1,2,3) + (4,5,6) &= (5,7,9)\\
  \begin{pmatrix} 1 & 2 & 3   \end{pmatrix} + \begin{pmatrix} 4 & 5 & 6   \end{pmatrix} &= \begin{pmatrix} 5 & 7 & 9   \end{pmatrix}\\
  \begin{pmatrix} 1\\ 2\\ 3   \end{pmatrix} + \begin{pmatrix} 4\\ 5\\ 6   \end{pmatrix} &= \begin{pmatrix} 5\\ 7\\ 9   \end{pmatrix}
\end{align*}

We can also multiply vectors by a \dfn{scalar} (a number), by
multiplying each component by the scalar.
\begin{align*}
  5\cdot   (1,2,3) &= (5,10,15)\\
  5\cdot \begin{pmatrix} 1 & 2 & 3   \end{pmatrix}  &= \begin{pmatrix} 5 & 10 & 15   \end{pmatrix}\\
  5\cdot  \begin{pmatrix} 1\\ 2\\ 3   \end{pmatrix} &= \begin{pmatrix} 5\\ 10\\ 15   \end{pmatrix}
\end{align*}
Now that we know the basics, onto the examples!

\begin{example}[Population Counts] %https://worldpopulationreview.com/states/states-by-race
  The Midwest of the United States consists of $12$ states. We can
  express the $2023$
  \link[demographics]{https://worldpopulationreview.com/states/states-by-race}
  of each state as a vector represented by an ordered tuple. The
  ordered tuple for Ohio looks like:
  \[
  \vec{p}_{\texttt{OH}} = (\underset{\text{White}}{9394878},\underset{\text{Black}}{1442655},\underset{\text{American Indian}}{20442},\underset{\text{Asian}}{268527},\underset{\text{Hawaiian}}{3907},\underset{\text{Other}}{544866}).
  \]
  The ordered tuples for each state in the Midwest looks like:
\begin{align*}
  \vec{p}_{\texttt{IA}} &= (2806418,117035,10538,79296,3941,132783)\\
  \vec{p}_{\texttt{IL}} &= (8874067,1796660,33972,709567,5196,1296702)\\
  \vec{p}_{\texttt{IN}} &= (5510354,631923,14030,158705,2205,379676)\\
  \vec{p}_{\texttt{KA}} &= (2416165,165837,22278,87093,2344,218902)\\
  \vec{p}_{\texttt{MI}} &= (7735902,1360149,50035,316844,3117,507860)\\
  \vec{p}_{\texttt{MN}} &= (4572149,359817,54558,275242,2201,336199)\\
  \vec{p}_{\texttt{MO}} &= (4978046,698043,24274,123810,8887,291100)\\
  \vec{p}_{\texttt{ND}} &= (651470,23959,39165,11979,1004,32817)\\
  \vec{p}_{\texttt{NE}} &= (1641256,91896,16875,47944,1235,124620)\\
  \vec{p}_{\texttt{OH}} &= (9394878,1442655,20442,268527,3907,544866)\\
  \vec{p}_{\texttt{SD}} &= (735228,18836,74975,12413,544,37340)\\
  \vec{p}_{\texttt{WI}} &= (4895065,367889,48674,163396,2672,329279)
\end{align*}
\begin{enumerate}
\item What are the combined demographics of the states Michigan, Ohio,
  and Indiana?
\item Suppose that the annual percentage growth rate of Ohio is
  currently $0.1\%$. Assuming this is even across all demographics,
  what might the population data look like for Ohio in 2025?
\end{enumerate}
\begin{solution}
  We'll use the properties of vectors to solve this problem.
  \begin{enumerate}
  \item To find the combined demographics of Michigan, Ohio, and
    Indiana we compute
    \[
    \vec{p}_{\texttt{MI}} + \vec{p}_{\texttt{OH}} + \vec{p}_{\texttt{IN}}
    \]
  \item To find the population growth of each demographic, we use the scalar, $1.001$ to find
    \[
    1.001 \vec{p}_{\texttt{OH}} = 
    \]
  \end{enumerate}
\end{solution}
\end{example}



\begin{example}[Financial Transactions]
  A bookstore sells a variety of books. We'll express the number of
  each type of book sold in one month as a row vector
  \[
  \vec{s} = \begin{pmatrix}199 & 249 & 141 & 304 & 251 \end{pmatrix} 
  \]
  with the entries representing the categories: Science Fiction,
  Fantasy, Mystery, Romance, Historical in that order.  If each book
  costs the bookstore $6$ dollars from the wholesaler, and each book sells
  for $9$ dollars, what is the profit for each category?
  \begin{solution}
    Using the properties of vectors, we wish to compute
    \[
    9\vec{s}-6\vec{s} = 
    \]
  \end{solution}
  
\end{example}



\begin{example}[Time Spent on Tasks]
  On Mondays, Wednesdays, and Fridays, suppose you spend you spend $1$
  hour on emails, $2$ hours in class, and $3$ hours working on a
  project. On Tuesdays and Thursdays, you spend $1/2$ hour on emails,
  $4$ hours in class, and $1$ hour exercising. We can express each day
  as a column vector:
  \[
  \vec{w}_{\texttt{MWF}} =
  \begin{pmatrix}
    1 \\
    2 \\
    3 \\
    0
  \end{pmatrix}
  \qquad
  \vec{w}_{\texttt{TR}} =
  \begin{pmatrix}
    0.5\\
    4\\
    0\\
    1
  \end{pmatrix} \qquad
  \begin{array}{l}
    \text{Emails}\\
    \text{Classes}\\
    \text{Project}\\
    \text{Exercising}
  \end{array}
  \]
\end{example}




\begin{example}[Velocities]
  Airplanes can use true bearings for navigation. Here's the idea, the direction is given by degrees from North:
  
\begin{center}
\upshape
% Define a few constants for easy configuration
\def\radius{2cm}
\def\onedegrad{1.8cm}
\def\fivedegrad{1.75cm}
\def\tendegrad{1.7cm}
\def\labelrad{1.6cm}

\begin{tikzpicture}[scale=1.3] %% https://texample.net/tikz/examples/degree-wheel/
  \draw[ultra thick] (0,0) circle (\radius);
  % labels and longer lines at every 10 degrees
  \node at (0,.65*\radius) {\bf N};
  \node at (0,-.65*\radius) {\bf S};
  \node at (-.65*\radius,0) {\bf W};
  \node at (.65*\radius,0) {\bf E};
  \foreach \x in {0,10,...,350}
  {
    \node[scale=.5, rotate=\x*-1] at (360-\x+90:\labelrad) {\x};
    \draw[thick] (\x:\tendegrad) -- (\x:\radius);
  };

  % lines at every 5 degrees
  \foreach \x in {0,5,...,355}  \draw (\x:\fivedegrad) -- (\x:\radius);

\end{tikzpicture}
\end{center}

So traveling North would be a (true) bearing of $0^\circ$, while
traveling West would be a bearing of $270^\circ$. Now suppose a plane
is flying at a bearing of $330^\circ$ at $125$
\link[knots]{https://en.wikipedia.org/wiki/Knot_(unit)}.
\begin{enumerate}
\item If the plane starts at $(0,0)$, what are the coordinates of the
  plane after it has traveled for $3$ hours?
\item 
\item
\end{enumerate}

\begin{solution}
While it makes sense (separately) to add speeds and angles, we cannot
directly make a vector from this information because there is no
obvious way to ``add'' ordered tuples of speed and angle. However, we
can convert a true bearing of $\beta$ along with a speed $v$ to
cartesian coordinates by writing:
\[
\begin{pmatrix}v\cdot \sin(\beta)\\ v \cdot \cos(\beta)\end{pmatrix}
\]
Armed with this, we are ready to solve the problem.
\begin{enumerate}
\item Converting the planes bearing and speed to a vector in cartesian coordinates, we have
  \[
  \vec{p} = \begin{pmatrix}125\cdot \sin(330^\circ)\\ 125 \cdot \cos(330^\circ)\end{pmatrix}
  \]
To find the the coordinates of the plane after it has traveled for $3$
hours, we use $3$ as a scalar to write
\begin{align*}
  3\vec{p} &= \begin{pmatrix}\answer[given]{375}\cdot \sin(330^\circ)\\ \answer[given]{375} \cdot \cos(330^\circ)\end{pmatrix}\\
  &= \begin{pmatrix}\answer[given]{-187.5}\\ \answer[given]{324.76} \end{pmatrix}\\
\end{align*}
  
\begin{image}
  \begin{tikzpicture}
    \begin{axis}[
        xmin=-1,xmax=5,ymin=-1,ymax=4,
            axis lines=center,
            %ticks=none,
            unit vector ratio*=1 1 1,
            xlabel=$x$, ylabel=$y$,
            ytick={-2,-1,...,7},
	    %yticklabels={$0.5$,$1$,$1.5$,$2$},
	    xtick={-2,-1,...,10},
	    %xticklabels={$0.5$,$1$,$1.5$,$2$},
	    grid = major,
            every axis y label/.style={at=(current axis.above origin),anchor=south},
            every axis x label/.style={at=(current axis.right of origin),anchor=west},
          ]
          \addplot[very thick,penColor,->] plot coordinates {(1,2) (4,3)};
          \addplot[very thick,penColor2,->] plot coordinates {(0,0) (1,2)};
          \addplot[ultra thick,penColor3,->] plot coordinates {(0,0) (4,3)};

           \node[left] at (axis cs:.5, 1) [penColor2] {$\vec{v}$};
           \node[above] at (axis cs:2.5, 2.5 ) [penColor] {$\vec{w}$};
           \node[below right] at (axis cs:2, 1.5 ) [penColor3] {$\vec{v}+\vec{w}$};
    \end{axis}
\end{tikzpicture}
  \end{image}
\end{enumerate}

\end{solution}
\end{example}



\section{Matrices store and transform data}


\subsection{Matrices store data}

Extract a new data set from a matrix using basis vectors

\subsection{Matrices transform data}

Scale/normalize population etc.


Geometry


\end{document}
