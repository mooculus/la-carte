\documentclass{ximera}

\newcommand{\dfn}{\textbf}
\renewcommand{\vec}[1]{{\overset{\boldsymbol{\rightharpoonup}}{\mathbf{#1}}}\hspace{0in}}
%% Simple horiz vectors
\renewcommand{\vector}[1]{\left\langle #1\right\rangle}
\newcommand{\arrowvec}[1]{{\overset{\rightharpoonup}{#1}}}
\newcommand{\R}{\mathbb{R}}
\newcommand{\transpose}{\intercal}
\newcommand{\ro}{\texttt{R}}%% row operation
\newcommand{\dotp}{\bullet}%% dot product

\usetikzlibrary{calc,bending}
\tikzset{>=stealth}


\usepackage{mdframed} % For framing content
%\usepackage{ifthen}   % For conditional statements

% Define the 'concept' environment with an optional header
\newenvironment{concept}[1][]{%
  \begin{mdframed}[linecolor=black, linewidth=2pt, innertopmargin=5pt, innerbottommargin=5pt, skipabove=12pt, skipbelow=12pt]%
    \noindent\large\textbf{#1}\normalsize%
}{%
  \end{mdframed}%
}











%% \colorlet{textColor}{black}
%% \colorlet{background}{white}
%% \colorlet{penColor}{blue!50!black} % Color of a curve in a plot
%% \colorlet{penColor2}{red!50!black}% Color of a curve in a plot
%% \colorlet{penColor3}{red!50!blue} % Color of a curve in a plot
%% \colorlet{penColor4}{green!50!black} % Color of a curve in a plot
%% \colorlet{penColor5}{orange!80!black} % Color of a curve in a plot
%% \colorlet{penColor6}{yellow!70!black} % Color of a curve in a plot
%% \colorlet{fill1}{penColor!20} % Color of fill in a plot
%% \colorlet{fill2}{penColor2!20} % Color of fill in a plot
%% \colorlet{fillp}{fill1} % Color of positive area
%% \colorlet{filln}{penColor2!20} % Color of negative area
%% \colorlet{fill3}{penColor3!20} % Fill
%% \colorlet{fill4}{penColor4!20} % Fill
%% \colorlet{fill5}{penColor5!20} % Fill
%% \colorlet{gridColor}{gray!50} % Color of grid in a plot



\author{Bart Snapp}



\title{Vector spaces}


\begin{document}
\begin{abstract}
  We describe properties of vectors and matrices and discuss their
  implications.
\end{abstract}
\maketitle

\begin{quote}
  We shall not cease from exploration\\
And the end of all our exploring\\
Will be to arrive where we started\\
And know the place for the first time.


  \hfill ---T.S.\ Eliot
\end{quote}


Data and their associated systems of equations are complicated. It is
not enough to simply solve these equations, we must understand the
solutions in a more abstract way. In this chapter we will introduce
various sets (or in this case, vector spaces) of vectors in the
context of solving equations. Then we will revist these ideas from
`higher ground,' and introduce the notion of a vector space.



\section{Homogeneous equations and the null space}

Suppose we have a system of equations:
\begin{align*}
  a_1 x + b_1 y + c_1 z &= d_1\\
  a_2 x + b_2 y + c_2 z &= d_2\\
  a_3 x + b_3 y + c_3 z &= d_3
\end{align*}
Where each $a_i$, $b_i$, $c_i$, and $d_i$ are real numbers. We can
learn about the solutions to this system of equations if we
\textit{change the problem} to the following:
\begin{align*}
  a_1 x + b_1 y + c_1 z &= 0  \\
  a_2 x + b_2 y + c_2 z &= 0  \\
  a_3 x + b_3 y + c_3 z &= 0
\end{align*}
Equations of this form are called \dfn{homogeneous equations}. We call
them \textit{homogeneous} because all the (nonzero) terms are
multiplied by the same \textbf{powers of variables}.
%% TK: I am not sure if I follow this explanation about why they are
%% called homogeneous, because I am not sure whether we can consider
%% the nonlinear system
%%
%%   a_i x^2 + b_i y^2 + c_i z^2 = 0, i = 1, 2, 3
%%
%% as homogeneous. The homogeneity in the case of linear systems
%% are due to 1) linearity and 2) zero constant terms. In
%% particular, because of this, it follows that if x is a solution
%% of Ax = 0, then any scalar multiple of x is also a solution. As a
%% consequence, ker(A) is a subspace.

%% Mayby, it was meant to be written as ``because all the
%% (nonconstant) terms are multiplied by ...''?

\begin{question}
  Which of the following equations are homogeneous?
  \begin{selectAll}
    \choice{$2x^3 + y^2 - 3z = 0$}
    \choice[correct]{$2x^3 + xy^2 - 3y^2z = 0$} % nonlinear, homogeneous
    \choice{$2x + y - 3 = 0$}
    \choice[correct]{$2x + y - 3z = 0$} % linear, homogeneous
  \end{selectAll}
\end{question}



\begin{definition}
  The \dfn{null space} (or \dfn{kernel}) of an $m \times n$ matrix $M$ is
  the set of all solutions to $M\vec{x} = \vec{0}$. We can express
  this as
  \[
  \Null(M) = \{ \vec{x}\in\R^n: M\vec{x} = \vec{0}\}.
  \]
\end{definition}



\begin{example}
  Consider the following homogenous system of equations.

  Explain why the null space has 0

  The sum

  scalar multiples

\end{example}





\section{Column space}

Given a matrix (not necessarily square)
\[
\begin{pmatrix}
  a_1 &  b_1  &  c_1 \\
  a_2 & b_2 & c_2 \\
  a_3 & b_3 & c_3
\end{pmatrix}
\]

\[
\begin{pmatrix} d_1 \\ d_2 \\ d_3 \end{pmatrix}=\begin{pmatrix} a_1 \\ a_2 \\ a_3 \end{pmatrix} x +
\begin{pmatrix} b_1 \\ b_2 \\ b_3 \end{pmatrix} y +
\begin{pmatrix} c_1 \\ c_2 \\ c_3 \end{pmatrix} z
\]
It's called the \textit{column space} because it's equal to every
possible vector produced by the columns of the augmented matrix
associated to the system of equations:
Given a system of equations
\begin{align*}
  a_1 x + b_1 y + c_1 z &= d_1\\
  a_2 x + b_2 y + c_2 z &= d_2\\
  a_3 x + b_3 y + c_3 z &= d_3
\end{align*}
the \dfn{column space} is the set of all possible constants, expressed
as a single vector:





\section{Linear combinatrions of vectors}

\begin{definition}\index{K-vector space@$K$-vector space}\index{vector space@$K$-vector space}
  A \textbf{$\boldsymbol{K}$-vector space} is an Abelian group $(V,+)$
  with identity $\vec{0}$, along with a field $K$ such that we may
  multiply group elements by field elements, meaning that there is a
  binary operation $-\cdot-: K\times V \to V$ such that if $\nu,\mu\in
  V$ and $a,b,\in K$ we have:
\begin{description}
\item[Compatibility with scalars] $(ab)\cdot \nu = a\cdot (b\cdot \nu)$.
\item[Vectors distribute over scalars] $(a+b)\cdot \nu =
  a\cdot\nu + b\cdot \nu$.
\item[Scalars distribute over vectors] $a\cdot (\nu+\mu) =
  a\cdot \nu + a\cdot \mu$.
\item[Identity is respected] $1_K\cdot \nu = \nu$.
\end{description}
In this case, elements of the group $V$ are called \dfn{vectors} and
elements of the field $K$ are called \dfn{scalars}.
\end{definition}

\begin{lemma}[Subspace criterion]\index{subspace criterion}
  Let $V$ be a $K$-vector space. $W\subset V$ is a subspace of $V$ if
  and only if
  \begin{enumerate}
  \item $W\ne \emptyset$.
  \item $W$ is closed under multiplication by scalars.
  \item $W$ is closed under vector addition.
  \end{enumerate}
\end{lemma}

\subsection{Span of vectors}

\begin{definition}
  Given a set of vectors $S$, in a $K$-vector space, $V$, the
  \dfn{span} of the vectors in $S$ is
  \[
  \Span(S) = \left\{\sum_{i=1}^n a_i\sigma_i:\text{$n\in \N$,
    $\sigma_i\in S$, and $a_i\in K$}\right\}.
  \]
  If $\Span(S) = V$, then we say $S$ is a \dfn{spanning set}.
\end{definition}

\subsection{Linear indendence of vectors}


\begin{definition}
  Given a $K$-vector space $V$, a finite set of vectors
  \[
  \{\lambda_1,\dots,\lambda_n\}
  \]
  is said to be \dfn{linearly independent} if
  \[
  a_1\lambda_1 + a_2\lambda_2 +\cdots + a_n\lambda_n = 0\quad \Rightarrow \quad a_1= \cdots =a_n = 0.
  \]
  A finite set of vectors is set to be \dfn{linearly dependent} if
  they are not linearly independent.
\end{definition}



\section{Row space}

\section{Column space and range of a matrix}

\subsection{Singular matrices}

\end{document}
