\documentclass{ximera}

\newcommand{\dfn}{\textbf}
\renewcommand{\vec}[1]{{\overset{\boldsymbol{\rightharpoonup}}{\mathbf{#1}}}\hspace{0in}}
%% Simple horiz vectors
\renewcommand{\vector}[1]{\left\langle #1\right\rangle}
\newcommand{\arrowvec}[1]{{\overset{\rightharpoonup}{#1}}}
\newcommand{\R}{\mathbb{R}}
\newcommand{\transpose}{\intercal}
\newcommand{\ro}{\texttt{R}}%% row operation
\newcommand{\dotp}{\bullet}%% dot product

\usetikzlibrary{calc,bending}
\tikzset{>=stealth}


\usepackage{mdframed} % For framing content
%\usepackage{ifthen}   % For conditional statements

% Define the 'concept' environment with an optional header
\newenvironment{concept}[1][]{%
  \begin{mdframed}[linecolor=black, linewidth=2pt, innertopmargin=5pt, innerbottommargin=5pt, skipabove=12pt, skipbelow=12pt]%
    \noindent\large\textbf{#1}\normalsize%
}{%
  \end{mdframed}%
}











%% \colorlet{textColor}{black}
%% \colorlet{background}{white}
%% \colorlet{penColor}{blue!50!black} % Color of a curve in a plot
%% \colorlet{penColor2}{red!50!black}% Color of a curve in a plot
%% \colorlet{penColor3}{red!50!blue} % Color of a curve in a plot
%% \colorlet{penColor4}{green!50!black} % Color of a curve in a plot
%% \colorlet{penColor5}{orange!80!black} % Color of a curve in a plot
%% \colorlet{penColor6}{yellow!70!black} % Color of a curve in a plot
%% \colorlet{fill1}{penColor!20} % Color of fill in a plot
%% \colorlet{fill2}{penColor2!20} % Color of fill in a plot
%% \colorlet{fillp}{fill1} % Color of positive area
%% \colorlet{filln}{penColor2!20} % Color of negative area
%% \colorlet{fill3}{penColor3!20} % Fill
%% \colorlet{fill4}{penColor4!20} % Fill
%% \colorlet{fill5}{penColor5!20} % Fill
%% \colorlet{gridColor}{gray!50} % Color of grid in a plot



\author{Bart Snapp}



\title{Applications of systems of linear equations}


\begin{document}
\begin{abstract}
  Concrete applications that use systems of linear equations. 
\end{abstract}
\maketitle

One power of mathematics is that it allows you to see disparate problems as being essentially the same. This is especially true of linear algebra. 
%% The examples in this section should build on the examples in "Vectors, scalars, and matrices"
%% Moreover, if necessary, the examples in the previous section should be changed for alignment. 

\begin{example}[Population Counts] %https://worldpopulationreview.com/states/states-by-race
  The Midwest of the United States consists of $12$ states. We can
  express the $2023$
  \link[demographics]{https://worldpopulationreview.com/states/states-by-race}
  of each state as a vector represented by an ordered tuple. The
  ordered tuple for Ohio looks like:
  \[
  \vec{p}_{\texttt{OH}} = (\underset{\text{White}}{9394878},\underset{\text{Black}}{1442655},\underset{\text{American Indian}}{20442},\underset{\text{Asian}}{268527},\underset{\text{Hawaiian}}{3907},\underset{\text{Other}}{544866}).
  \]
  The ordered tuples for each state in the Midwest looks like:
\begin{align*}
  \vec{p}_{\texttt{IA}} &= (2806418,117035,10538,79296,3941,132783)\\
  \vec{p}_{\texttt{IL}} &= (8874067,1796660,33972,709567,5196,1296702)\\
  \vec{p}_{\texttt{IN}} &= (5510354,631923,14030,158705,2205,379676)\\
  \vec{p}_{\texttt{KA}} &= (2416165,165837,22278,87093,2344,218902)\\
  \vec{p}_{\texttt{MI}} &= (7735902,1360149,50035,316844,3117,507860)\\
  \vec{p}_{\texttt{MN}} &= (4572149,359817,54558,275242,2201,336199)\\
  \vec{p}_{\texttt{MO}} &= (4978046,698043,24274,123810,8887,291100)\\
  \vec{p}_{\texttt{ND}} &= (651470,23959,39165,11979,1004,32817)\\
  \vec{p}_{\texttt{NE}} &= (1641256,91896,16875,47944,1235,124620)\\
  \vec{p}_{\texttt{OH}} &= (9394878,1442655,20442,268527,3907,544866)\\
  \vec{p}_{\texttt{SD}} &= (735228,18836,74975,12413,544,37340)\\
  \vec{p}_{\texttt{WI}} &= (4895065,367889,48674,163396,2672,329279)
\end{align*}
\end{example}


\begin{example}[Financial Transactions]
\end{example}

\begin{example}[Time Spent on Tasks]
  
\end{example}




\begin{example}[Navigation]
\end{example}




\begin{example}{Enigma} %% see: https://www.math.utah.edu/~gustafso/s2016/2270/published-projects-2016/adamsTyler-moodyDavid-choiHaysun-CryptographyTheEnigmaMachine.pdf
\end{example}


\end{document}
