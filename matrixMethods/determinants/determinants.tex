\documentclass{ximera}

\newcommand{\dfn}{\textbf}
\renewcommand{\vec}[1]{{\overset{\boldsymbol{\rightharpoonup}}{\mathbf{#1}}}\hspace{0in}}
%% Simple horiz vectors
\renewcommand{\vector}[1]{\left\langle #1\right\rangle}
\newcommand{\arrowvec}[1]{{\overset{\rightharpoonup}{#1}}}
\newcommand{\R}{\mathbb{R}}
\newcommand{\transpose}{\intercal}
\newcommand{\ro}{\texttt{R}}%% row operation
\newcommand{\dotp}{\bullet}%% dot product
\renewcommand{\l}{\ell}
\let\defaultAnswerFormat\answerFormatBoxed
\usetikzlibrary{calc,bending}
\tikzset{>=stealth}


\usepackage{mdframed} % For framing content
%\usepackage{ifthen}   % For conditional statements

% Define the 'concept' environment with an optional header
\newenvironment{concept}[1][]{%
  \begin{mdframed}[linecolor=black, linewidth=2pt, innertopmargin=5pt, innerbottommargin=5pt, skipabove=12pt, skipbelow=12pt]%
    \noindent\large\textbf{#1}\normalsize%
}{%
  \end{mdframed}%
}



%% Define exercise collection. 
\makeatletter
\newcommand{\exerciseCollection}[2]{
\def\input@path{{#1}}
\activity{#1#2}}



\newcommand{\practicestyle}{
  
  \let\exercise\relax
\let\endexercise\relax
\let\c@exercise\relax
\let\problem\relax
\let\endproblem\relax
\let\c@problem\relax 
\newtheoremstyle{problem}
{\topsep}{\topsep}{\rmfamily}{}{\bfseries}{)}{ }{}
\theoremstyle{problem}
\newtheorem{problem}{}
\newtheorem{exercise}[problem]{}
\section{Exercises for Chapter~\thetitlenumber}
\small%\twocolumn
%% \let\exercise\relax
%%   \let\endexercise\relax
%%   \let\c@exercise\relax 
%%   \newtheoremstyle{exercise}
%%                   {\topsep}{\topsep}{\rmfamily}{}{\bfseries}{)}{~}{}
%% \theoremstyle{exercise}
%% \newtheorem{exercise}{}
}
\renewcommand\chapterstyle{%
  \def\activitystyle{activity-chapter}
  \normalsize
  %\onecolumn
  \def\maketitle{%
    \addtocounter{titlenumber}{1}%
                    {\flushleft\small\sffamily\bfseries\@pretitle\par\vspace{-1.5em}}%
                    {\flushleft\LARGE\sffamily\bfseries\thetitlenumber\hspace{1em}\@title \par }%
                    {\vskip .6em\noindent\textit\theabstract\setcounter{problem}{0}\setcounter{section}{0}}%
                    \par\vspace{2em}
                    \phantomsection\addcontentsline{toc}{section}{\textbf{\thetitlenumber\hspace{1em}\@title}}%
}}
\makeatother







%% \colorlet{textColor}{black}
%% \colorlet{background}{white}
%% \colorlet{penColor}{blue!50!black} % Color of a curve in a plot
%% \colorlet{penColor2}{red!50!black}% Color of a curve in a plot
%% \colorlet{penColor3}{red!50!blue} % Color of a curve in a plot
%% \colorlet{penColor4}{green!50!black} % Color of a curve in a plot
%% \colorlet{penColor5}{orange!80!black} % Color of a curve in a plot
%% \colorlet{penColor6}{yellow!70!black} % Color of a curve in a plot
%% \colorlet{fill1}{penColor!20} % Color of fill in a plot
%% \colorlet{fill2}{penColor2!20} % Color of fill in a plot
%% \colorlet{fillp}{fill1} % Color of positive area
%% \colorlet{filln}{penColor2!20} % Color of negative area
%% \colorlet{fill3}{penColor3!20} % Fill
%% \colorlet{fill4}{penColor4!20} % Fill
%% \colorlet{fill5}{penColor5!20} % Fill
%% \colorlet{gridColor}{gray!50} % Color of grid in a plot

\newcommand{\rvx}[1]{\rule[0.5ex]{1em}{.05pt}\,#1\,\rule[0.5ex]{1em}{.05pt}}

\author{Bart Snapp \and Tae Eun Kim}



\title{Determinants}


\begin{document}
\begin{abstract}
  Determinants are numbers that give information about matrices.
\end{abstract}
\maketitle

The determinant is a function of square matrices in the sense that it
takes a square matrix as input and yields a number as output. Viewing
a square matrix as a stack of row vectors (or a concatenation of
column vectors), the determinant may also be viewed as a function of
several input variables producing a scalar output. Regardless of the
perspective adopted, the determinant is a tool which extracts a
representative number that conveys a special kind of information about a
given matrix or the set of its constituent vectors.

The chapter begins mildly with the determinant of the simplest
$2 \times 2$ matrices and study its various properties both in algebraic
and geometric terms. These properties are then shown to extend to
higher dimensions. Different methods of calculation the determinant of
general $n \times n$ matrices are introduced. The connection between the
row operations studied in a previous chapter is presented
afterward. The chapter concludes by showing how to interpret the
determinant of a matrix, both when it is viewed as a data and as a
transformation.

\section{The determinant of $2 \times 2$ matrices}

\begin{definition}\label{defn:det-2by2}
  The \dfn{determinant} of a $2 \times 2$ matrix $\begin{pmatrix} a & b \\
    c & d \end{pmatrix}$ is given by
  \[
    \det
    \begin{pmatrix}
      a & c\\
      b & d
    \end{pmatrix}
    =
    \begin{vmatrix}
      a & b\\
      c & d
    \end{vmatrix}
    = ad - bc.
  \]
\end{definition}

% Typically, when one computes the determinant of a $2\times 2$ matrix,
% we think of the terms as follows.
In words, the determinant of a $2\times 2$ matrix is the difference between
the product of diagonal terms and the product of anti-diagonal terms.
\begin{image}[0.75in]
  \begin{tikzpicture}
    \matrix (mtrx)  [matrix of math nodes,ampersand replacement=\&,%column sep=.1em,
    nodes={text height=1ex,text width=2ex}]
    {
      a \& b \\
      c \& d \\
    };
    \draw[thick] (mtrx-1-1.north) -| (mtrx-2-1.south west)
    -- (mtrx-2-1.south);
    \draw[thick] (mtrx-1-2.north) -| (mtrx-2-2.south east)
    -- (mtrx-2-2.south);
    \draw[ultra thick,red!20!white,->]
    (mtrx-1-2.center) -- (mtrx-2-1.center);
    \draw[draw,ultra thick,blue!20!white,->]
    (mtrx-1-1.center)  --  (mtrx-2-2.center);
    \matrix (mtrx2)  [matrix of math nodes,ampersand replacement=\&,nodes={text height=1ex,text width=2ex}]
    {
      a \& b\\
      c \& d\\
    };
    \node at (0,-.7) {${\color{blue!50!black}ad}-{\color{red!50!black}bc}$};
  \end{tikzpicture}
\end{image}

% \begin{image}[1in]
%   \begin{tikzpicture}
%     \matrix (mtrx)  [matrix of math nodes,ampersand replacement=\&,%column sep=.1em,
%     nodes={text height=1ex,text width=2ex}]
%     {
%       a \& c \\
%       b \& d \\
%     };
%     \draw[thick] (mtrx-1-1.north) -| (mtrx-2-1.south west)
%     -- (mtrx-2-1.south);
%     \draw[thick] (mtrx-1-2.north) -| (mtrx-2-2.south east)
%     -- (mtrx-2-2.south);
%     \draw[ultra thick,red!20!white,->]
%     (mtrx-1-2.center) -- (mtrx-2-1.center);
%     \draw[draw,ultra thick,blue!20!white,->]
%     (mtrx-1-1.center)  --  (mtrx-2-2.center);
%     \matrix (mtrx2)  [matrix of math nodes,ampersand replacement=\&,nodes={text height=1ex,text width=2ex}]
%     {
%       a \& c\\
%       b \& d\\
%     };
%     \node at (0,-.7) {${\color{blue!50!black}ad}-{\color{red!50!black}bc}$};
%   \end{tikzpicture}
% \end{image}

% Why would anyone ever be interested in this? Well to start, given nonzero vectors

\subsection{Geometric interpretation}
To gain a geometric insight behind this definition, consider the
parallelogram spanned by two row vectors
\[
  \vec{v} =
  \begin{pmatrix}
    a & b
  \end{pmatrix}
  \quad\text{and}\quad
  \vec{w} =
  \begin{pmatrix}
    c & d
  \end{pmatrix},
\]
which are the two rows of the matrix in the definition; see the figure
below.

\begin{image}
\begin{tikzpicture}
  \draw[fill=fill1!50!white,draw=none]
  (0, 0)         % starting point
  -- ++(5,1)     % move along this vector
  -- ++(2,3)     % then along this vector
  -- ++(-5,-1)   % then back along that vector
  -- cycle;      % and back to where you started
  \draw[->,ultra thick,penColor] (0,0) -- (5,1);
  \draw[->,ultra thick,penColor2] (0,0) -- (2,3);
  \draw[->,ultra thick,dashed,penColor2] (5,1) -- (7,4);
  \draw[->,ultra thick,dashed,penColor] (2,3) -- (7,4);
  \node[above,penColor] at (2.5,.5) {$\vec{v}$}; %% <a,b>
  \node[below right,penColor2] at (1,1.5) {$\vec{w}$}; %% <c,d>
  \draw[->] (.5,.1) arc[radius=.5cm,start angle=11.3,end angle=56.3];
  \node[above right] at (.4,.2) {$\theta$};
\end{tikzpicture}
\end{image}

The determinant gives the \textit{signed} area of this parallelogram,
where the sign of the area is determined by the orientation of the two
vectors $\vec{v}$ and $\vec{w}$. If the second vector $\vec{w}$ lies
counterclockwise from the first vector $\vec{v}$ as illustrated above,
then the sign is positive.

To understand how the determinant is related to the area of the
parallelogram, make a rectangle around the parallelogram as below.

\begin{image}
\begin{tikzpicture}
  \draw[fill=fill1!50!white,draw=none]
  (0, 0)         % starting point
  -- ++(5,1)     % move along this vector
  -- ++(2,3)     % then along this vector
  -- ++(-5,-1)   % then back along that vector
  -- cycle;      % and back to where you started

  \draw[fill=fill5] (5, 0) -- (7,0) -- (7,1) -- (5,1) -- cycle;
  \draw[fill=fill5] (0, 3) -- (2,3) -- (2,4) -- (0,4) -- cycle;

  \draw[fill=fill3] (0, 0) -- (5,0) -- (5,1) -- cycle;
  \draw[fill=fill3] (2,3) -- (7,4) -- (2,4) -- cycle;

  \draw[fill=fill4] (0, 0) -- (2,3) -- (0,3) -- cycle;
  \draw[fill=fill4] (5, 1) -- (7,1) -- (7,4) -- cycle;

  \draw[->,ultra thick,penColor] (0,0) -- (5,1);
  \draw[->,ultra thick,penColor2] (0,0) -- (2,3);
  \draw[->,ultra thick,dashed,penColor2] (5,1) -- (7,4);
  \draw[->,ultra thick,dashed,penColor] (2,3) -- (7,4);

  \draw[decoration={brace,mirror,raise=.1cm},decorate,thin] (0,0)--(5,0);
  \node[below] at (2.5,-.2) {$a$};

  \draw[decoration={brace,mirror,raise=.1cm},decorate,thin] (7,4)--(2,4);
  \node[above] at (4.5,4.2) {$a$};

  \draw[decoration={brace,mirror,raise=.1cm},decorate,thin] (7,0)--(7,1);
  \node[right] at (7.2,.5) {$b$};

  \draw[decoration={brace,mirror,raise=.1cm},decorate,thin] (0,4)--(0,3);
  \node[left] at (-.2,3.5) {$b$};

  \draw[decoration={brace,mirror,raise=.1cm},decorate,thin] (7,1)--(7,4);
  \node[right] at (7.2,2.5) {$d$};

  \draw[decoration={brace,mirror,raise=.1cm},decorate,thin] (0,3)--(0,0);
  \node[left] at (-.2,1.5) {$d$};

  \draw[decoration={brace,mirror,raise=.1cm},decorate,thin] (0,3)--(0,0);
  \node[left] at (-.2,1.5) {$d$};

  \draw[decoration={brace,raise=.1cm},decorate,thin] (0,4)--(2,4);
  \node[above] at (1,4.2) {$c$};

  \draw[decoration={brace,mirror,raise=.1cm},decorate,thin] (5,0)--(7,0);
  \node[below] at (6,-.2) {$c$};

  \node[above,penColor] at (2.5,.5) {$\vec{v}$}; %% <a,b>
  \node[below right,penColor2] at (1,1.5) {$\vec{w}$}; %% <c,d>
\end{tikzpicture}
\end{image}

Now, we see that the area of the parallelogram is
\[
\text{Area of rectangle} - \text{Area of other regions}
\]
and this is
\begin{align*}
  (a+c)(b+d) - \left(cd + ab + 2bc \right)&= ab + ad + bc + cd - cd - ab-2bc\\
  &=  ad - bc\\
  &= \det
  \begin{pmatrix}
    a & c\\
    b & d
  \end{pmatrix}.
\end{align*}

\subsection{Fundamental properties of the determinant}
Here we present some fundamental properties of the determinant of
$2 \times 2$ matrices. These properties are described as
\textit{fundamental} because they hold true for the determinant of any
square matrices. For this, the properties will be stated in terms of
arbitrarily sized square matrices. Rigorous proofs will not be
presented, but both algebraic and geometric confirmation will be
provided using $2 \times 2$ matrices.

In what follows, $n$ denotes a positive integer.

\begin{definition}[Identity matrix]\label{defn:id}
  The $n \times n$ \dfn{identity matrix}, denoted $I_n$, is the
  $n \times n$ matrix with ones on the main diagonal and zeros elsewhere:
  \[
    I_n =
    \begin{pmatrix}
      1 & 0 & 0 & \cdots & 0 \\
      0 & 1 & 0 & \cdots & 0 \\
      0 & 0 & 1 & \cdots & 0 \\
      \vdots & \vdots & \vdots & \ddots & \vdots \\
      0 & 0 & 0 & \cdots & 1
    \end{pmatrix}.
  \]
\end{definition}
When clear from context, the subscript $n$ is often dropped from
$I_n$, and it is simply written as $I$.

\begin{proposition}[Determinant of identity]
  The determinant of the $n \times n$ identity matrix is 1, that is,
  \[
    \det I_n = 1.
  \]
\end{proposition}

It is readily confirmed for $I_2$ using the formula given in
Definition \ref{defn:det-2by2}:
\[
  \det
  \begin{pmatrix}
    1 & 0 \\ 0 & 1
  \end{pmatrix}
  = 1 \cdot 1 - 0 \cdot 0 = 1.
\]

Geometrically, $\det I_2$ is the signed area of the parallelogram
spanned by $\vec{e}_1^\transpose = (1, 0)$ and
$\vec{e}_2^\transpose = (0, 1)$, which in fact is the unit square with
area 1. Note that the second vector $\vec{e}_2^\transpose$ (facing due
north) lies counterclockwise from the first vector
$\vec{e}_1^\transpose$ (facing due east), so the area is assigned the
positive sign.

\begin{image}[2in]
  \begin{tikzpicture}
    \draw[fill=fill1!50!white,draw=none]
    (0, 0)                      % starting point
    -- ++(5,0)                   % move along this vector
    -- ++(0,5)                   % then along this vector
    -- ++(-5,0)                  % then back along that vector
    -- cycle;                    % and back to where you started
    \draw[->,ultra thick,penColor] (0,0) -- (5,0);
    \draw[->,ultra thick,penColor2] (0,0) -- (0,5);
    \draw[->,ultra thick,dashed,penColor2] (5,0) -- (5,5);
    \draw[->,ultra thick,dashed,penColor] (0,5) -- (5,5);
    \node[above,penColor] at (2.5,-.75) {$\vec{e}_1^\transpose$};
    \node[below right,penColor2] at (-.75,2.5) {$\vec{e}_2^\transpose$};
    \draw[->,thick] (.5,0) -- (.5,.5) -- (0,.5);
    \node[above right] at (.5,.5) {$\theta = \pi/2$};
    \node at (2.5,2.5) {$\det\begin{pmatrix} 1 & 0 \\ 0 & 1 \end{pmatrix} = 1$};
\end{tikzpicture}
\end{image}



\begin{proposition}[Sign reversal]
  The determinant changes signs when two rows are interchanged.
\end{proposition}

Checking this for the $2 \times 2$ case:
\[
  \det
  \begin{pmatrix}
    c & d \\
    a & b
  \end{pmatrix}
  = bc - ad = -(ad - bc) =
  - \det
  \begin{pmatrix}
    a & b \\
    c & d
  \end{pmatrix}.
\]

Geometrically speaking, interchanging two rows of a $2 \times 2$ matrix
boils down to reversing the orientation of the parallelogram spanned
by it two row vectors, which has the very effect of changing the sign
of the area, while keeping the magnitude unchanged.

\begin{image}[.98\textwidth]
  \begin{tikzpicture}
    \draw[fill=fill1!50!white,draw=none]
    (0, 0)         % starting point
    -- ++(5,1)     % move along this vector
    -- ++(2,3)     % then along this vector
    -- ++(-5,-1)   % then back along that vector
    -- cycle;      % and back to where you started
    \draw[->,ultra thick,penColor] (0,0) -- (5,1);
    \draw[->,ultra thick,penColor2] (0,0) -- (2,3);
    \draw[->,ultra thick,dashed,penColor2] (5,1) -- (7,4);
    \draw[->,ultra thick,dashed,penColor] (2,3) -- (7,4);
    \node[below,penColor] at (4,0.5) {$\vec{v}$}; %% <a,b>
    \node[left,penColor2] at (1.2,2.4) {$\vec{w}$}; %% <c,d>
    \draw[->,thick] (.98,0.2) arc[radius=1cm,start angle=11.3,end angle=56.3];
    \node[above right] at (.4,.2) {$\theta$};
    \node at (3.5, 2) {$\det\begin{pmatrix} \vec{v} \\
      \vec{w} \end{pmatrix} > 0$};
    \node at (3.5, -1) {counterclockwise orientation};
  \end{tikzpicture}
  \hfill
  \begin{tikzpicture}
    \draw[fill=fill1!50!white,draw=none]
    (0, 0)         % starting point
    -- ++(5,1)     % move along this vector
    -- ++(2,3)     % then along this vector
    -- ++(-5,-1)   % then back along that vector
    -- cycle;      % and back to where you started
    \draw[->,ultra thick,penColor2] (0,0) -- (5,1);
    \draw[->,ultra thick,penColor] (0,0) -- (2,3);
    \draw[->,ultra thick,dashed,penColor] (5,1) -- (7,4);
    \draw[->,ultra thick,dashed,penColor2] (2,3) -- (7,4);
    \node[below,penColor2] at (4,0.5) {$\vec{v}$}; %% <a,b>
    \node[left,penColor] at (1.2,2.4) {$\vec{w}$}; %% <c,d>
    \draw[->] (.55,.83) arc[radius=1cm,end angle=11.3,start angle=56.3];
    \node[above right] at (.4,.2) {$\theta$};
    \node at (3.5, 2) {$\det\begin{pmatrix} \vec{w} \\
      \vec{v} \end{pmatrix} < 0$};
    \node at (3.5, -1) {clockwise orientation};
  \end{tikzpicture}
\end{image}

\begin{proposition}[Multilinearity]
  The determinant is a linear function of each row while other rows
  are held fixed.
\end{proposition}

We illustrate the last property in the $2 \times 2$ case. In particular, we
show that the determinant is linear in the first row while the second
row is held fixed. This entails showing:
\begin{itemize}
\item multiplying the first row by a scalar $k$, the determinant is
  multiplied by $k$;
\item adding a row vector to the first row, the two determinants are added.
\end{itemize}

Both can be shown using the $2 \times 2$ determinant formula:
\[
  \det
  \begin{pmatrix}
    k a & k b \\
    c & d
  \end{pmatrix}
  = (k a)d - (k b)c
  = k (ad - bc)
  = k \det
  \begin{pmatrix}
    a & b \\
    c & d
  \end{pmatrix}
\]
and
\begin{align*}
  \det
  \begin{pmatrix}
    a_1 + a_2 & b_1 + b_2 \\
    c & d
  \end{pmatrix}
  & = (a_1 + a_2) d - (b_1 + b_2) c \\
  & = (a_1 d - b_1 c) + (a_2 d - b_2 c) \\
  & = \det
  \begin{pmatrix}
    a_1 & b_1 \\
    c & d
  \end{pmatrix}
  + \det
  \begin{pmatrix}
    a_2 & b_2 \\
    c & d
  \end{pmatrix}.
\end{align*}

The geometric illustration of this last property can be quite
satisfying. Below illustrates the \textit{scaling property} for the concrete
case where $k = 3$:
\begin{image}[0.75\textwidth]
  \begin{tikzpicture}
    % shade
    \draw[fill=fill1!50!white,draw=none] % det(3u,v)
    (0, 0) -- ++(6,0) -- ++(2,3) -- ++(-6,0) -- cycle;
    \draw[fill=fill1!80!white,draw=none] % det(u,v)
    (0, 0) -- ++(2,0) -- ++(2,3) -- ++(-2,0) -- cycle;

    % vectors (arrows)
    \draw[->,ultra thick,penColor] (0,0) -- (2,0);
    \draw[->,ultra thick,penColor] (2,0) -- (4,0);
    \draw[->,ultra thick,penColor] (4,0) -- (6,0);
    \draw[->,ultra thick,penColor2] (0,0) -- (2,3);

    % parallel mirrors
    \draw[->,ultra thick,dashed,penColor2] (6,0) -- (8,3);
    \draw[->,ultra thick,dashed,penColor] (2,3) -- (8,3);

    % auxilliary lines
    \draw[-,dashed,penColor2] (2,0) -- (4,3);
    \draw[-,dashed,penColor2] (4,0) -- (6,3);

    % labels
    \node[below,penColor] at (1,-0.1) {$\vec{v}$};
    \node[below,penColor] at (3,-0.1) {$\vec{v}$};
    \node[below,penColor] at (5,-0.1) {$\vec{v}$};
    \node[left,penColor2] at (1.2,2.5) {$\vec{w}$};
  \end{tikzpicture}
\end{image}

One the one hand, the area of the large parallelogram equals
\[
  \det
  \begin{pmatrix}
    3 \vec{v} \\ \vec{w}
  \end{pmatrix}.
\]
On the other hand, the picture suggests that this area is three times
the area of the small parallelogram with darker shade which, in turn,
can be written in terms of the determinant as well:
\[
  3 \times \text{(Area of the smaller parallelogram)}
  = 3 \det
  \begin{pmatrix}
    \vec{v} \\ \vec{w}
  \end{pmatrix}.
\]
Equating the two expression, we arrive at the desired equality:
\[
  \det
  \begin{pmatrix}
    3 \vec{v} \\ \vec{w}
  \end{pmatrix}
  =
  3 \det
  \begin{pmatrix}
    \vec{v} \\ \vec{w}
  \end{pmatrix}.
\]

We leave to the readers the illustration of the other property, namely,
\[
  \det
  \begin{pmatrix}
    \vec{v}_1 + \vec{v}_2 \\ \vec{w}
  \end{pmatrix}
  = \det
  \begin{pmatrix}
    \vec{v}_1 \\ \vec{w}
  \end{pmatrix}
  + \det
  \begin{pmatrix}
    \vec{v}_2 \\ \vec{w}
  \end{pmatrix}.
\]


%% COME BACK HERE AND FINISH --------------------
Moving on to the next dimension:
\begin{definition}
  The \dfn{determinant} of a $3 \times 3$ matrix is given by
  \begin{align*}
    \det\begin{pmatrix}
      a_1 &  b_1 & c_1 \\
      a_2 &  b_2 & c_2 \\
      a_3 &  b_3 & c_3
    \end{pmatrix}
    & =
      a_1 \det
      \begin{pmatrix}
        b_2 & c_2 \\
        b_3 & c_3
      \end{pmatrix}
      - a_2 \det
      \begin{pmatrix}
        b_1 & c_1 \\
        b_3 & c_3
      \end{pmatrix}
      + a_3 \det
      \begin{pmatrix}
        b_1 & c_1 \\
        b_2 & c_2
      \end{pmatrix} \\
          & = a_1(b_2c_3 - b_3c_2) - a_2(b_1c_3 - b_3c_1) + a_3(b_1c_2 - b_2c_1).
  \end{align*}
\end{definition}


One thing to note is that the determinant of a $3 \times 3$ matrix is
defined in terms of determinants of smaller $2 \times 2$ matrices. Note
that each of the three $2 \times 2$ matrices is obtained from the original
matrix by removing the first column and a row at a time. Such matrices
are called \textit{minor matrices} (of the given matrix).

Note also the alternation of signs from term to term. The determinants
of minor matrices with alternating signs are called the \textit{cofactors} (of
the given matrix).

Because of these, the formula for the determinant given above is often
called the \textit{cofactor expansion}, where the coefficients are the
elements of the first column. Incidentally, the determinant can also
be computed using the entries of the first row as coefficients. See
below for the illustration of this alternate method.

  \begin{center}
    \begin{tikzpicture}
      \node at (0,0) {$\det
        \begin{pmatrix}
          a_1 & a_2 & a_3\\
          b_1 & b_2 & b_3\\
          c_1 & c_2 & c_3
        \end{pmatrix}
        = a_1
        \begin{vmatrix}
          b_2 & b_3\\
          c_2 & c_3
        \end{vmatrix}
        -a_2
        \begin{vmatrix}
          b_1 & b_3\\
          c_1 & c_3
        \end{vmatrix}
        +a_3
        \begin{vmatrix}
          b_1 & b_2\\
          c_1 & c_2
        \end{vmatrix}$};

      \draw[ultra thick,->,gray] (-.27,-1.3) -- (-.27,-.7);
      \draw[ultra thick,->,gray] (1.76,-1.3) -- (1.76,-.7);
      \draw[ultra thick,->,gray] (3.8,-1.3) -- (3.8,-.7);

      \node at (-.27,-2) {\scalebox{.7}{
          $\begin{pmatrix}
            a_1 & \bullet & \bullet\\
            \bullet & b_2 & b_3\\
            \bullet & c_2 & c_3
          \end{pmatrix}$}};

      \node at (1.76,-2) {\scalebox{.7}{
          $\begin{pmatrix}
            \bullet & a_2 & \bullet\\
            b_1 & \bullet & b_3\\
            c_1 & \bullet & c_3
          \end{pmatrix}$}};

      \node at (3.8,-2) {\scalebox{.7}{
          $\begin{pmatrix}
            \bullet & \bullet & a_3\\
            b_1 & b_2 & \bullet\\
            c_1 & c_2 & \bullet
          \end{pmatrix}$}};
    \end{tikzpicture}
  \end{center}

  The idea of cofactor expansion nicely extends to handle the
  computation of the determinant of a general $n \times n$ matrix, for any
  positive integer $n$.

% \begin{definition}
%   Given a $3\times 3$ matrix, the \dfn{determinant} is denoted by the following.
%   \[
%   \det\begin{pmatrix}
%   a_1 &  a_2 & a_3 \\
%   b_1 &  b_2 & b_3 \\
%   c_1 &  c_2 & c_3
%   \end{pmatrix}
%   =
%   \begin{vmatrix}
%     a_1 &  a_2 & a_3 \\
%     b_1 &  b_2 & b_3 \\
%     c_1 &  c_2 & c_3
%   \end{vmatrix}
%   \]
%   The determinant can be computed using the following pattern.
%   %\begin{image}[3in]
%   \begin{center}
% \begin{tikzpicture}
% \node at (0,0) {$\det
% \begin{pmatrix}
%   a_1 & a_2 & a_3\\
%   b_1 & b_2 & b_3\\
%   c_1 & c_2 & c_3
% \end{pmatrix}
% = a_1
% \begin{vmatrix}
%   b_2 & b_3\\
%   c_2 & c_3
% \end{vmatrix}
% -a_2
% \begin{vmatrix}
%   b_1 & b_3\\
%   c_1 & c_3
% \end{vmatrix}
% +a_3
% \begin{vmatrix}
%   b_1 & b_2\\
%   c_1 & c_2
% \end{vmatrix}$};

% \draw[ultra thick,->,gray] (-.27,-1.3) -- (-.27,-.7);
% \draw[ultra thick,->,gray] (1.76,-1.3) -- (1.76,-.7);
% \draw[ultra thick,->,gray] (3.8,-1.3) -- (3.8,-.7);

% \node at (-.27,-2) {\scalebox{.7}{
%     $\begin{pmatrix}
%       a_1 & \bullet & \bullet\\
%       \bullet & b_2 & b_3\\
%       \bullet & c_2 & c_3
%     \end{pmatrix}$}};

% \node at (1.76,-2) {\scalebox{.7}{
%     $\begin{pmatrix}
%       \bullet & a_2 & \bullet\\
%       b_1 & \bullet & b_3\\
%       c_1 & \bullet & c_3
%     \end{pmatrix}$}};

% \node at (3.8,-2) {\scalebox{.7}{
%     $\begin{pmatrix}
%       \bullet & \bullet & a_3\\
%       b_1 & b_2 & \bullet\\
%       c_1 & c_2 & \bullet
%     \end{pmatrix}$}};
% \end{tikzpicture}
% \end{center}
% %\end{image}
% This equals
% \[
%  a_1(b_2c_3-b_3c_2)- a_2(b_1c_3-b_3c_1)+ a_3(b_1c_2-b_2c_1).
% \]
% \end{definition}

Another way to conceptualize this computation is the following:
%% Commented because this is not a particularly useful method in this course.
%% %% Credit for the following image goes to
%% %% http://tex.stackexchange.com/a/257063
\begin{image}
  \begin{tikzpicture}
    \matrix (mtrx)  [matrix of math nodes,ampersand replacement=\&,column sep=1em, nodes={text height=1ex,text width=2ex}]
            {
              a_1 \& a_2 \& a_3 \& a_1 \& a_2         \\
              b_1 \& b_2 \& b_3 \& b_1 \& b_2         \\
              c_1 \& c_2 \& c_3 \& c_1 \& c_2         \\
            };
            \matrix (mtrx3)  [matrix of math nodes,ampersand replacement=\&,column sep=1em, nodes={text height=1ex,text width=2ex}] at (5,0)
            {
              a_1 \& a_2 \& a_3 \& a_1 \& a_2         \\
              b_1 \& b_2 \& b_3 \& b_1 \& b_2         \\
              c_1 \& c_2 \& c_3 \& c_1 \& c_2         \\
            };
            \draw[thick] (mtrx-1-1.north) -| (mtrx-3-1.south west)
            -- (mtrx-3-1.south);
            \draw[thick] (mtrx-1-3.north) -| (mtrx-3-3.south east)
            -- (mtrx-3-3.south);
            \draw[ultra thick,red!20!white,->]
            (mtrx3-3-1.center) edge (mtrx3-1-3.center)
            (mtrx3-3-2.center) edge (mtrx3-1-4.center)
            (mtrx3-3-3.center)  --  (mtrx3-1-5.center);
            \draw[draw,ultra thick,blue!20!white,->]
            (mtrx-1-1.center) edge (mtrx-3-3.center)
            (mtrx-1-2.center) edge (mtrx-3-4.center)
            (mtrx-1-3.center)  --  (mtrx-3-5.center);
            \matrix (mtrx2)  [matrix of math nodes,ampersand replacement=\&,column sep=1em, nodes={text height=1ex,text width=2ex}]
            {
              a_1 \& a_2 \& a_3 \& a_1 \& a_2         \\
              b_1 \& b_2 \& b_3 \& b_1 \& b_2         \\
              c_1 \& c_2 \& c_3 \& c_1 \& c_2         \\
            };
            \matrix (mtrx4)  [matrix of math nodes,ampersand replacement=\&,column sep=1em, nodes={text height=1ex,text width=2ex}] at (5,0)
            {
              a_1 \& a_2 \& a_3 \& a_1 \& a_2         \\
              b_1 \& b_2 \& b_3 \& b_1 \& b_2         \\
              c_1 \& c_2 \& c_3 \& c_1 \& c_2         \\
            };
            \draw[thick] (mtrx4-1-1.north) -| (mtrx4-3-1.south west)
            -- (mtrx4-3-1.south);
            \draw[thick] (mtrx4-1-3.north) -| (mtrx4-3-3.south east)
            -- (mtrx4-3-3.south);
            \node at (2.5,-1) {${\color{blue!50!black}a_1b_2c_3}+{\color{blue!50!black}a_2b_3c_1}+{\color{blue!50!black}a_3b_1c_2}-{\color{red!50!black}a_3b_2c_1}-{\color{red!50!black}a_1b_3c_2}-{\color{red!50!black}a_2b_1c_3}$};
            \node at (0,1) {{\color{blue!50!black}positive terms}};
            \node at (5,1) {{\color{red!50!black}negative terms}};
  \end{tikzpicture}
\end{image}

\begin{question}
  Compute.
  \[
  \det
  \begin{pmatrix}
    1 & 4 & 7\\
    2 & 5 & 8\\
    3 & 6 & 9\\
  \end{pmatrix}
  \begin{prompt}
    = \answer{0}
  \end{prompt}
  \]
\end{question}



\section{Connections to row operations and row echelon form}

    %% Swapping Rows: If you swap two rows of a matrix, the determinant of the matrix changes sign. That is, if you swap rows ii and jj in a matrix AA, the determinant of the new matrix A′A′ is det⁡(A′)=−det⁡(A)det(A′)=−det(A).

    %% Multiplying a Row by a Scalar: If you multiply a row by a non-zero scalar cc, the determinant of the matrix is multiplied by cc. That is, if you multiply row ii by cc in a matrix AA, the determinant of the new matrix A′A′ is det⁡(A′)=c⋅det⁡(A)det(A′)=c⋅det(A).

    %% Adding a Multiple of One Row to Another: If you add a multiple of
    %% one row to another row, the determinant of the matrix remains
    %% unchanged. That is, if you replace row ii with
    %% %rowi+c⋅rowjrowi​+c\cdot rowj​
    %% in a matrix $A$, the determinant of the new
    %% matrix $A'$ is $\det⁡(A')=det⁡(A)det(A')=det(A)$.



\section{When matrices store data}



\section{When matrices transform data}





\end{document}
