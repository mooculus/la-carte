\documentclass{ximera}

\author{Tae Eun Kim}

\newcommand{\dfn}{\textbf}
\renewcommand{\vec}[1]{{\overset{\boldsymbol{\rightharpoonup}}{\mathbf{#1}}}\hspace{0in}}
%% Simple horiz vectors
\renewcommand{\vector}[1]{\left\langle #1\right\rangle}
\newcommand{\arrowvec}[1]{{\overset{\rightharpoonup}{#1}}}
\newcommand{\R}{\mathbb{R}}
\newcommand{\transpose}{\intercal}
\newcommand{\ro}{\texttt{R}}%% row operation
\newcommand{\dotp}{\bullet}%% dot product

\usetikzlibrary{calc,bending}
\tikzset{>=stealth}


\usepackage{mdframed} % For framing content
%\usepackage{ifthen}   % For conditional statements

% Define the 'concept' environment with an optional header
\newenvironment{concept}[1][]{%
  \begin{mdframed}[linecolor=black, linewidth=2pt, innertopmargin=5pt, innerbottommargin=5pt, skipabove=12pt, skipbelow=12pt]%
    \noindent\large\textbf{#1}\normalsize%
}{%
  \end{mdframed}%
}











%% \colorlet{textColor}{black}
%% \colorlet{background}{white}
%% \colorlet{penColor}{blue!50!black} % Color of a curve in a plot
%% \colorlet{penColor2}{red!50!black}% Color of a curve in a plot
%% \colorlet{penColor3}{red!50!blue} % Color of a curve in a plot
%% \colorlet{penColor4}{green!50!black} % Color of a curve in a plot
%% \colorlet{penColor5}{orange!80!black} % Color of a curve in a plot
%% \colorlet{penColor6}{yellow!70!black} % Color of a curve in a plot
%% \colorlet{fill1}{penColor!20} % Color of fill in a plot
%% \colorlet{fill2}{penColor2!20} % Color of fill in a plot
%% \colorlet{fillp}{fill1} % Color of positive area
%% \colorlet{filln}{penColor2!20} % Color of negative area
%% \colorlet{fill3}{penColor3!20} % Fill
%% \colorlet{fill4}{penColor4!20} % Fill
%% \colorlet{fill5}{penColor5!20} % Fill
%% \colorlet{gridColor}{gray!50} % Color of grid in a plot

% row-vector-extension: draw horizontal lines on both sides of a row
% vector when constructing a matrix by stacking multiple rows
\newcommand{\rvx}[1]{\rule[0.5ex]{1em}{.05pt}\,#1\,\rule[0.5ex]{1em}{.05pt}}

\begin{document}
\begin{exercise}
  Let $\vec{e}_1, \vec{e}_2, \vec{e}_3, \vec{e}_4$ be defined by
  \[
    \vec{e}_1 =
    \begin{pmatrix}
      1\\ 0\\ 0\\ 0
    \end{pmatrix},\;
    \vec{e}_2 =
    \begin{pmatrix}
      0\\ 1\\ 0\\ 0
    \end{pmatrix},\;
    \vec{e}_3 =
    \begin{pmatrix}
      0\\ 0\\ 1\\ 0
    \end{pmatrix},\;
    \vec{e}_4 =
    \begin{pmatrix}
      0\\ 0\\ 0\\ 1
    \end{pmatrix}.
  \]
  In words, $\vec{e}_j$ is the column vector whose $j$-th component is
  one and the rest are all zeros. These vectors $\vec{e}_j$'s are often
  called the \textit{standard unit basis vectors}.

  Let $\vec{v}$ be a column 4-vector:
  \[
    \vec{v} =
    \begin{pmatrix}
      v_1 \\ v_2 \\ v_3 \\ v_4
    \end{pmatrix}.
  \]

  What is the result of $\vec{e}_2^\transpose \vec{v}$?
  \begin{multipleChoice}
    \choice{$v_1$}
    \choice[correct]{$v_2$}
    \choice{$v_3$}
    \choice{$v_4$}
  \end{multipleChoice}
  \begin{feedback}[correct]
    In general, the operation $\vec{e}_j^\transpose \vec{v}$ extracts
    the $j$-th component of $\vec{v}$.
  \end{feedback}

  Stacking transposed unit basis vectors, we form a matrix which
  consists only of 0's and 1's, with only one 1 appearing on each row. For
  example, the matrix $S$ defined by
  \[
    S =
    \begin{pmatrix}
    \rvx{\vec{e}_1^\transpose}\\
      \rvx{\vec{e}_2^\transpose}\\
      \rvx{\vec{e}_3^\transpose}\\
      \rvx{\vec{e}_4^\transpose}
    \end{pmatrix}
  \]
  is a $3 \times 4$ matrix. Write out the elements of $S$.
  \[
    S =
    \begin{pmatrix}
      \answer{0} & \answer{1} & \answer{0} & \answer{0} \\
      \answer{0} & \answer{0} & \answer{1} & \answer{0} \\
      \answer{0} & \answer{0} & \answer{0} & \answer{1}
    \end{pmatrix}.
  \]

  Now, which of the following is the result of $S \vec{v}$?
  \begin{multipleChoice}
    \choice{$\begin{pmatrix} v_1 \\ v_2 \\ v_3 \end{pmatrix}$}
    \choice{$\begin{pmatrix} 0 \\ v_1 \\ v_2 \\ v_3 \end{pmatrix}$}
    \choice[correct]{$\begin{pmatrix} v_2 \\ v_3 \\ v_4 \end{pmatrix}$}
    \choice{$\begin{pmatrix} 0 \\ v_2 \\ v_3 \\ v_4 \end{pmatrix}$}
  \end{multipleChoice}

  \begin{feedback}[correct]
    In general, the matrix-vector multiplication $S \vec{v}$
    rearranges the elements of $\vec{v}$ according to the arrangement
    of the unit basis vectors forming $S$.
  \end{feedback}
\end{exercise}
\end{document}
