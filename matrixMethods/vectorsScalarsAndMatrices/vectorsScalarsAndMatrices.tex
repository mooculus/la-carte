\documentclass{ximera}

\author{Bart Snapp}
%%OpenAI. (2023). ChatGPT (Mar 14 version) [Large language model]. https://chat.openai.com/chat


\title{Vectors, scalars, and matrices}


\begin{document}
\begin{abstract}
\end{abstract}
\maketitle


\section{Vectors and scalars}

In mathematics, we study (amoung other things) numbers. In many cases,
these numbers represent real-world data. Sometimes it ``makes sense''
to add data and sometimes it doesn't.
\begin{description}
\item[Personal Identification Numbers] Summing things like social
  security numbers, phone numbers, or zip codes doesn't provide
  meaningful information.
\item[Temperatures] While you can mathematically add temperatures
  together, doing so usually doesn't provide useful information.
\item[Time] Adding specific points in time, like dates or hours of the
  day, usually doesn't yield meaningful results. 
\item[Geographic Coordinates] Adding the latitude and longitude of two
  locations doesn't give you a location that has any real-world
  significance.
\end{description}
For some of the categories above, the \textit{average} can be
meaningful; or perhaps if each quantity is thought of as
`displacement' or `duration,' summing might be meaningful. However,
without additional (pun intended!) stipulations, summing the types of
quantites above is not meaningful.


On the other hand, there's lots of times that it makes sense to add data:

\begin{description}
\item[Financial Transactions] Adding up all transactions made by a
  company or an individual over a certain period. For example, summing
  daily sales to find total monthly sales, or adding up all expenses
  to find total costs. These are real numbers that represent actual
  amounts of money, and their total gives meaningful information about
  financial status or performance.
\item[Distances] If you walk 3 miles in the morning and another 2
  miles in the evening, adding these distances gives a meaningful
  total of 5 miles walked in a day. This applies to any physical
  measurements that are additive, such as lengths, widths, and
  heights, where combined measurements are relevant.
\item[Quantities of Items] Similar to your apple example, if you're
  counting objects, adding makes sense. If a warehouse has 200 units
  of product A, 300 units of product B, and 500 units of product C,
  the total inventory is 1000 units when these figures are added
  together.
\item[Population Counts] Adding the populations of different regions,
  cities, or countries to find the total population of a larger
  area. For example, adding the populations of all the states in a
  country to get the total national population.
\item[Energy Consumption] Adding up energy used (in kilowatt-hours,
  for example) over different periods (days, weeks) or in different
  locations (home, office) to get a total energy consumption
  figure. This can be important for understanding overall energy usage
  and for planning energy needs.
\item[Time Spent on Tasks] Adding the amount of time spent on various
  activities throughout a day to get a total time spent. For example,
  if you spend 1 hour on emails, 2 hours in meetings, and 3 hours
  working on a project, your total work time for the day is 6 hours.
\end{description}



\section{Matrices}

\end{document}
