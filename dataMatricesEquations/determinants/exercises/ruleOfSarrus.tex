\documentclass{ximera}

\author{Tae Eun Kim}

\newcommand{\dfn}{\textbf}
\renewcommand{\vec}[1]{{\overset{\boldsymbol{\rightharpoonup}}{\mathbf{#1}}}\hspace{0in}}
%% Simple horiz vectors
\renewcommand{\vector}[1]{\left\langle #1\right\rangle}
\newcommand{\arrowvec}[1]{{\overset{\rightharpoonup}{#1}}}
\newcommand{\R}{\mathbb{R}}
\newcommand{\transpose}{\intercal}
\newcommand{\ro}{\texttt{R}}%% row operation
\newcommand{\dotp}{\bullet}%% dot product
\renewcommand{\l}{\ell}
\let\defaultAnswerFormat\answerFormatBoxed
\usetikzlibrary{calc,bending}
\tikzset{>=stealth}


\usepackage{mdframed} % For framing content
%\usepackage{ifthen}   % For conditional statements

% Define the 'concept' environment with an optional header
\newenvironment{concept}[1][]{%
  \begin{mdframed}[linecolor=black, linewidth=2pt, innertopmargin=5pt, innerbottommargin=5pt, skipabove=12pt, skipbelow=12pt]%
    \noindent\large\textbf{#1}\normalsize%
}{%
  \end{mdframed}%
}



%% Define exercise collection. 
\makeatletter
\newcommand{\exerciseCollection}[2]{
\def\input@path{{#1}}
\activity{#1#2}}



\newcommand{\practicestyle}{
  
  \let\exercise\relax
\let\endexercise\relax
\let\c@exercise\relax
\let\problem\relax
\let\endproblem\relax
\let\c@problem\relax 
\newtheoremstyle{problem}
{\topsep}{\topsep}{\rmfamily}{}{\bfseries}{)}{ }{}
\theoremstyle{problem}
\newtheorem{problem}{}
\newtheorem{exercise}[problem]{}
\section{Exercises for Chapter~\thetitlenumber}
\small%\twocolumn
%% \let\exercise\relax
%%   \let\endexercise\relax
%%   \let\c@exercise\relax 
%%   \newtheoremstyle{exercise}
%%                   {\topsep}{\topsep}{\rmfamily}{}{\bfseries}{)}{~}{}
%% \theoremstyle{exercise}
%% \newtheorem{exercise}{}
}
\renewcommand\chapterstyle{%
  \def\activitystyle{activity-chapter}
  \normalsize
  %\onecolumn
  \def\maketitle{%
    \addtocounter{titlenumber}{1}%
                    {\flushleft\small\sffamily\bfseries\@pretitle\par\vspace{-1.5em}}%
                    {\flushleft\LARGE\sffamily\bfseries\thetitlenumber\hspace{1em}\@title \par }%
                    {\vskip .6em\noindent\textit\theabstract\setcounter{problem}{0}\setcounter{section}{0}}%
                    \par\vspace{2em}
                    \phantomsection\addcontentsline{toc}{section}{\textbf{\thetitlenumber\hspace{1em}\@title}}%
}}
\makeatother







%% \colorlet{textColor}{black}
%% \colorlet{background}{white}
%% \colorlet{penColor}{blue!50!black} % Color of a curve in a plot
%% \colorlet{penColor2}{red!50!black}% Color of a curve in a plot
%% \colorlet{penColor3}{red!50!blue} % Color of a curve in a plot
%% \colorlet{penColor4}{green!50!black} % Color of a curve in a plot
%% \colorlet{penColor5}{orange!80!black} % Color of a curve in a plot
%% \colorlet{penColor6}{yellow!70!black} % Color of a curve in a plot
%% \colorlet{fill1}{penColor!20} % Color of fill in a plot
%% \colorlet{fill2}{penColor2!20} % Color of fill in a plot
%% \colorlet{fillp}{fill1} % Color of positive area
%% \colorlet{filln}{penColor2!20} % Color of negative area
%% \colorlet{fill3}{penColor3!20} % Fill
%% \colorlet{fill4}{penColor4!20} % Fill
%% \colorlet{fill5}{penColor5!20} % Fill
%% \colorlet{gridColor}{gray!50} % Color of grid in a plot


\begin{document}
\begin{exercise}

  % TODO: Maybe move this paragraph on the rule of Sarrus to the
  % exercise. --> DONE on 07/17/2024

  Another way to compute the determinant of a $3 \times 3$ matrix to follow
  the \textit{rule of Sarrus}:
  %% Commented because this is not a particularly useful method in this course.
  %% %% Credit for the following image goes to
  %% %% http://tex.stackexchange.com/a/257063
  \begin{image}
    \begin{tikzpicture}
      \matrix (mtrx)  [matrix of math nodes,ampersand replacement=\&,column sep=1em, nodes={text height=1ex,text width=2ex}]
      {
        a_1 \& a_2 \& a_3 \& a_1 \& a_2         \\
        b_1 \& b_2 \& b_3 \& b_1 \& b_2         \\
        c_1 \& c_2 \& c_3 \& c_1 \& c_2         \\
      };
      \matrix (mtrx3)  [matrix of math nodes,ampersand replacement=\&,column sep=1em, nodes={text height=1ex,text width=2ex}] at (5,0)
      {
        a_1 \& a_2 \& a_3 \& a_1 \& a_2         \\
        b_1 \& b_2 \& b_3 \& b_1 \& b_2         \\
        c_1 \& c_2 \& c_3 \& c_1 \& c_2         \\
      };
      \draw[thick] (mtrx-1-1.north) -| (mtrx-3-1.south west)
      -- (mtrx-3-1.south);
      \draw[thick] (mtrx-1-3.north) -| (mtrx-3-3.south east)
      -- (mtrx-3-3.south);
      \draw[ultra thick,red!20!white,->]
      (mtrx3-3-1.center) edge (mtrx3-1-3.center)
      (mtrx3-3-2.center) edge (mtrx3-1-4.center)
      (mtrx3-3-3.center)  --  (mtrx3-1-5.center);
      \draw[draw,ultra thick,blue!20!white,->]
      (mtrx-1-1.center) edge (mtrx-3-3.center)
      (mtrx-1-2.center) edge (mtrx-3-4.center)
      (mtrx-1-3.center)  --  (mtrx-3-5.center);
      \matrix (mtrx2)  [matrix of math nodes,ampersand replacement=\&,column sep=1em, nodes={text height=1ex,text width=2ex}]
      {
        a_1 \& a_2 \& a_3 \& a_1 \& a_2         \\
        b_1 \& b_2 \& b_3 \& b_1 \& b_2         \\
        c_1 \& c_2 \& c_3 \& c_1 \& c_2         \\
      };
      \matrix (mtrx4)  [matrix of math nodes,ampersand replacement=\&,column sep=1em, nodes={text height=1ex,text width=2ex}] at (5,0)
      {
        a_1 \& a_2 \& a_3 \& a_1 \& a_2         \\
        b_1 \& b_2 \& b_3 \& b_1 \& b_2         \\
        c_1 \& c_2 \& c_3 \& c_1 \& c_2         \\
      };
      \draw[thick] (mtrx4-1-1.north) -| (mtrx4-3-1.south west)
      -- (mtrx4-3-1.south);
      \draw[thick] (mtrx4-1-3.north) -| (mtrx4-3-3.south east)
      -- (mtrx4-3-3.south);
      \node at (2.5,-1) {${\color{blue!50!black}a_1b_2c_3}+{\color{blue!50!black}a_2b_3c_1}+{\color{blue!50!black}a_3b_1c_2}-{\color{red!50!black}a_3b_2c_1}-{\color{red!50!black}a_1b_3c_2}-{\color{red!50!black}a_2b_1c_3}$};
      \node at (0,1) {{\color{blue!50!black}positive terms}};
      \node at (5,1) {{\color{red!50!black}negative terms}};
    \end{tikzpicture}
  \end{image}

  The idea is to attach the first two columns of the given matrix to the
  right of the third column, then to add the products of the down-right
  diagonals (blue  arrows), then to subtract the products of the
  up-right diagonals (red arrows).

  Use this method to compute the following determinants. (Recall that
  $|A| = \det(A)$.)

  \begin{enumerate}
  \pdfOnly{\begin{multicols}{2}}
  \item
    $\begin{vmatrix}
      1 & 0 & 4 \\
      0 & 5 & -2 \\
      2 & 3 & 2
    \end{vmatrix}$ \begin{prompt} $= \answer{-24}$ \end{prompt}

  \item
    $\begin{vmatrix}
      0 & 3 & 1 \\
      4 & -5 & 0 \\
      4 & 3 & 2
    \end{vmatrix}$ \begin{prompt} $= \answer{8}$ \end{prompt}

  \item
    $\begin{vmatrix}
      1 & 3 & -1 \\
      2 & 2 & 3 \\
      -3 & 2 & 3
    \end{vmatrix}$ \begin{prompt} $= \answer{-55}$ \end{prompt}

  \item
    $\begin{vmatrix}
      1 & 3 & 4 \\
      2 & 3 & 1 \\
      2 & 2 & 1
    \end{vmatrix}$ \begin{prompt} $= \answer{-7}$ \end{prompt}
  \pdfOnly{\end{multicols}}
  \end{enumerate}
  \textbf{Note.} The rule of Sarrus only works for $3 \times 3$ matrices.

\end{exercise}
\end{document}
