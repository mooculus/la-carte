\setcounter{tocdepth}{1}
\makeatletter
\usepackage[T1]{fontenc}
\usepackage{stix2}
\usepackage{gillius}
\usepackage{multicol}

%% Since font changed -- need to renewcommands again
\renewcommand{\vec}[1]{\mathbf{#1}}
\renewcommand{\hat}{\widehat}



%% Size of the page
\usepackage[margin=.5in, includehead, includefoot, hmargin={1.8in,1.5in}, paperwidth=8.5in, paperheight=8.5in]{geometry}
%\usepackage[margin=.5in, includehead, includefoot, hmargin={.8in,.5in}, paperwidth=11in, paperheight=8.5in]{geometry}


%% Redefine section and subsection
\renewcommand\section{\@startsection {section}{1}{\z@}%
                                   {-3.5ex \@plus -1ex \@minus -.2ex}%
                                   {2.3ex \@plus.2ex}%
                                   {\boldmath\normalfont\Large\bfseries\sffamily}}
\renewcommand\subsection{\@startsection{subsection}{2}{\z@}%
                                     {-3.25ex \@plus -1ex \@minus -.2ex}%
                                     {1.5ex \@plus .2ex}%
                                     {\boldmath\normalfont\large\bfseries\sffamily}}




%% Redefine "example" and "proposition"
\let\example\relax
\let\endexample\relax
\let\proposition\relax
\let\endproposition\relax
%\let\c@example\relax
\newtheoremstyle{example}
{\topsep}{\topsep}{\rmfamily}{}{\bfseries\sffamily}{:}{ }{#1 (\thmnote{#3})}
\theoremstyle{example}
\newtheorem*{example}{Example}
\newtheorem*{proposition}{Proposition}


%% Redefine "question"
\let\question\relax
\let\endquestion\relax
%% Redefine "warning"
\let\warning\relax
\let\endwarning\relax
%% Redefine "definition"
\let\definition\relax
\let\enddefinition\relax
%\let\c@example\relax
\newtheoremstyle{definition}
{\topsep}{\topsep}{\rmfamily}{}{\bfseries\sffamily}{:}{ }{#1}
\theoremstyle{definition}
\newtheorem*{definition}{Definition}
\newtheorem*{warning}{WARNING}
\newtheorem*{question}{Question}



% Define the 'concept' environment with an optional header
\renewenvironment{algorithm}[1][]{%
  \begin{mdframed}[linecolor=black, linewidth=2pt, innertopmargin=5pt, innerbottommargin=5pt, skipabove=12pt, skipbelow=12pt]%
    \noindent\textbf{#1}\normalsize\\%
}{%
  \end{mdframed}%
}




%% Redefines "explanation"
\renewmdenv[
  font=\upshape,
  linewidth=3mm,
  topline=false,
  bottomline=false,
  leftline=true,
  rightline=false, 
  leftmargin=0,
  innertopmargin=0pt,
  innerbottommargin=0pt,
  skipbelow=\baselineskip,
  linecolor=textColor!20!white,
  fontcolor=textColor,
  backgroundcolor=background
]{explanation}%


%% Redefines "selectAll"
\renewenvironment{selectAll}[1][]
{\begin{trivlist}\item[\hskip \labelsep\bfseries Select All Correct Answers:]\hfil\begin{enumerate}}
    {\end{enumerate}\end{trivlist}}


%% Redefine "multipleChice"
\renewenvironment{multipleChoice}[1][]
{% Environment Start Code
\setkeys{multipleChoice}{#1}%
\recordvariable{\mc@id}%
\begin{trivlist}
\item[\hskip \labelsep\bfseries Multiple Choice:]\hfil
\begin{enumerate}
}% Note this means that \item has to be the first line after \begin{multipleChoice}.
{% Environment End Code
\end{enumerate}
\end{trivlist}
}









%% Define exercise collection. 
\newcommand{\exerciseCollection}[2]{
\def\input@path{{#1}}
\activity{#1#2}}


\newcommand{\practicestyle}{
\newgeometry{hmargin={.8in,.5in}}
\let\exercise\relax
\let\endexercise\relax
\let\c@exercise\relax
\let\problem\relax
\let\endproblem\relax
\let\c@problem\relax
\newtheoremstyle{problem}
{\topsep}{\topsep}{\rmfamily}{}{\bfseries\sffamily}{)}{ }{}
\theoremstyle{problem}
\newtheorem{problem}{}
\newtheorem{exercise}[problem]{}
\twocolumn[\section{Exercises for Chapter~\thetitlenumber}\vspace*{-1em}\hrule height .5mm \vspace*{.5em}]
\scriptsize
%% \let\exercise\relax
%%   \let\endexercise\relax
%%   \let\c@exercise\relax 
%%   \newtheoremstyle{exercise}
%%                   {\topsep}{\topsep}{\rmfamily}{}{\bfseries}{)}{~}{}
%% \theoremstyle{exercise}
%% \newtheorem{exercise}{}
}
\renewcommand\chapterstyle{%
\restoregeometry
  \def\activitystyle{activity-chapter}
  \onecolumn
  \normalsize
  %\onecolumn
  \def\maketitle{%
    \addtocounter{titlenumber}{1}%
                    {\flushleft\small\sffamily\bfseries\@pretitle\par\vspace{-1.5em}}%
                    {\flushleft\LARGE\sffamily\bfseries\thetitlenumber\hspace{1em}\@title \par }%
                    {\vskip .6em\noindent\textit\theabstract\setcounter{problem}{0}\setcounter{section}{0}}%
                    \par\vspace{2em}
                    \phantomsection\addcontentsline{toc}{section}{\textbf{\thetitlenumber\hspace{1em}\@title}}%
}}

\onecolumn
\makeatother
