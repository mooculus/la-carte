\documentclass{ximera}
\usepackage{kbordermatrix}

%\newcommand{\dfn}{\textbf}
\renewcommand{\vec}[1]{{\overset{\boldsymbol{\rightharpoonup}}{\mathbf{#1}}}\hspace{0in}}
%% Simple horiz vectors
\renewcommand{\vector}[1]{\left\langle #1\right\rangle}
\newcommand{\arrowvec}[1]{{\overset{\rightharpoonup}{#1}}}
\newcommand{\R}{\mathbb{R}}
\newcommand{\transpose}{\intercal}
\newcommand{\ro}{\texttt{R}}%% row operation
\newcommand{\dotp}{\bullet}%% dot product
\renewcommand{\l}{\ell}
\let\defaultAnswerFormat\answerFormatBoxed
\usetikzlibrary{calc,bending}
\tikzset{>=stealth}


\usepackage{mdframed} % For framing content
%\usepackage{ifthen}   % For conditional statements

% Define the 'concept' environment with an optional header
\newenvironment{concept}[1][]{%
  \begin{mdframed}[linecolor=black, linewidth=2pt, innertopmargin=5pt, innerbottommargin=5pt, skipabove=12pt, skipbelow=12pt]%
    \noindent\large\textbf{#1}\normalsize%
}{%
  \end{mdframed}%
}



%% Define exercise collection. 
\makeatletter
\newcommand{\exerciseCollection}[2]{
\def\input@path{{#1}}
\activity{#1#2}}



\newcommand{\practicestyle}{
  
  \let\exercise\relax
\let\endexercise\relax
\let\c@exercise\relax
\let\problem\relax
\let\endproblem\relax
\let\c@problem\relax 
\newtheoremstyle{problem}
{\topsep}{\topsep}{\rmfamily}{}{\bfseries}{)}{ }{}
\theoremstyle{problem}
\newtheorem{problem}{}
\newtheorem{exercise}[problem]{}
\section{Exercises for Chapter~\thetitlenumber}
\small%\twocolumn
%% \let\exercise\relax
%%   \let\endexercise\relax
%%   \let\c@exercise\relax 
%%   \newtheoremstyle{exercise}
%%                   {\topsep}{\topsep}{\rmfamily}{}{\bfseries}{)}{~}{}
%% \theoremstyle{exercise}
%% \newtheorem{exercise}{}
}
\renewcommand\chapterstyle{%
  \def\activitystyle{activity-chapter}
  \normalsize
  %\onecolumn
  \def\maketitle{%
    \addtocounter{titlenumber}{1}%
                    {\flushleft\small\sffamily\bfseries\@pretitle\par\vspace{-1.5em}}%
                    {\flushleft\LARGE\sffamily\bfseries\thetitlenumber\hspace{1em}\@title \par }%
                    {\vskip .6em\noindent\textit\theabstract\setcounter{problem}{0}\setcounter{section}{0}}%
                    \par\vspace{2em}
                    \phantomsection\addcontentsline{toc}{section}{\textbf{\thetitlenumber\hspace{1em}\@title}}%
}}
\makeatother







%% \colorlet{textColor}{black}
%% \colorlet{background}{white}
%% \colorlet{penColor}{blue!50!black} % Color of a curve in a plot
%% \colorlet{penColor2}{red!50!black}% Color of a curve in a plot
%% \colorlet{penColor3}{red!50!blue} % Color of a curve in a plot
%% \colorlet{penColor4}{green!50!black} % Color of a curve in a plot
%% \colorlet{penColor5}{orange!80!black} % Color of a curve in a plot
%% \colorlet{penColor6}{yellow!70!black} % Color of a curve in a plot
%% \colorlet{fill1}{penColor!20} % Color of fill in a plot
%% \colorlet{fill2}{penColor2!20} % Color of fill in a plot
%% \colorlet{fillp}{fill1} % Color of positive area
%% \colorlet{filln}{penColor2!20} % Color of negative area
%% \colorlet{fill3}{penColor3!20} % Fill
%% \colorlet{fill4}{penColor4!20} % Fill
%% \colorlet{fill5}{penColor5!20} % Fill
%% \colorlet{gridColor}{gray!50} % Color of grid in a plot



%% \author{Parisa Fatheddin}





%% \begin{document}
%% Here we discuss an application of matrix algebra to the field of genetics. Consider two breeds of horses: bay and black. The dominant allele, the gene in chromosome that determines which breed the offspring will be, is assumed to be black denoted as B and the recessive allele is bay denoted as b. Then there are three genotypes: BB, Bb, bb. Genotype BB means the horse is black, Bb means the horse is black (since B is the dominant allele), and Bb means the horse is bay. We can find the probability of the genotype of an offspring based on its parents' genotypes as follows. \\
%% If both parents have genotypes BB then the offspring has to have genotype BB so we can formulate the following probabilities: 
%% \begin{eqnarray*}
%% P\left(\text{offspring's genotype is BB if both parents' genotypes are BB}\right) &=& 1,\\
%% P\left(\text{offspring's genotype is Bb if both parents' genotypes are BB}\right) &=& 0,\\
%% P\left(\text{offspring's genotype is bb if both parents' genotypes are BB}\right) &=& 0.
%% \end{eqnarray*}
%% Thus, we get the column vector summarizing the probabilities above: $\kbordermatrix{ &\\
%%  \text{BB} & 1  \\
%% \text{Bb} &0  \\
%% \text{bb} & 0
%% }$.\\
%% Now we consider all other possibilities and give their probability vectors. \\
%% If both parents have genotypes bb we have the probabilities represented by the column vector: $\kbordermatrix{ &\\
%%  \text{BB} & 0  \\
%% \text{Bb} &0  \\
%% \text{bb} & 1
%% }$.\\
%%  If both parents have genotypes Bb, there is a $50\%$ chance of offspring's genotype becoming Bb and another $50\%$ chance of it having genotype BB or bb giving: $\kbordermatrix{ &\\
%%  \text{BB} & 1/4  \\
%% \text{Bb} &1/2  \\
%% \text{bb} & 1/4  
%% }$.\\
%%  If one parent has BB genotype and another bb genotype, then the only possibility for the offspring's genotype is Bb giving: $\kbordermatrix{ &\\
%%  \text{BB} & 0 \\
%% \text{Bb} &1  \\
%% \text{bb} & 0
%% }$.\\
%% A BB genotype parent with a Bb genotype parent gives $\kbordermatrix{ &\\
%%  \text{BB} & 1/2  \\
%% \text{Bb} &1/2  \\
%% \text{bb} & 0
%% }$.\\ and a bb genotype parent with a Bb genotype parent gives, $\kbordermatrix{ &\\
%%  \text{BB} & 0  \\
%% \text{Bb} &1/2  \\
%% \text{bb} & 1/2
%% }$.\\
%% Putting all the above probability vectors into one matrix we have, \\
%% $ P = \kbordermatrix{ &\textbf{BB-BB} & \textbf{bb-bb} & \textbf{Bb-Bb} & \textbf{BB-bb}& \textbf{BB-Bb}& \textbf{Bb-bb}\\
%%  \text{BB} & 1 & 0 & 1/4 & 0 & 1/2 & 0  \\
%% \text{Bb} & 0 & 0 & 1/2& 1 & 1/2 & 1/2\\
%% \text{bb} & 0 & 1 & 1/4 & 0 & 0& 1/2
%% }$.\\

%% To be able to use matrix multiplication in this setting, let us assume that we know one of the parent's genotype to be BB, then the above matrix reduces to
%% \[A= \kbordermatrix{ &\textbf{BB-BB} & \textbf{BB-Bb} & \textbf{BB-bb}\\ 
%%  \text{BB} &1 & 1/2 & 0 \\
%% \text{Bb} &0 & 1/2 & 1\\
%% \text{bb} & 0 & 0 & 0 
%% }\].
%% We also assume that initially the population has equal numbers of genotypes BB, Bb, and bb giving the initial probabilities,
%% \[
%% x_{0} = \kbordermatrix{ & \\
%% \text{BB} & 1/3\\
%% \text{Bb} & 1/3\\
%% \text{bb} & 1/3},\]\\
%% since the sum of all probabilities has to equal 1. Then we can find the distribution (the fraction of each type in the population) after one generation by multiplying vector $A$ with $x_{0}$:
%% \[
%% Ax_{0} = \begin{pmatrix}1 & 1/2 & 0\\
%% 0 & 1/2 & 1 \\
%% 0 & 0 & 0
%% \end{pmatrix} \begin{pmatrix}  1/3 \\
%% 1/3\\
%% 1/3
%% \end{pmatrix}= \kbordermatrix{ & \\
%% \text{BB}& 1/2\\
%% \text{Bb} & 1/2\\
%% \text{bb} & 0}
%% \]
%% To find the probabilities after the second generation, we have, 
%% \[
%% A(Ax_{0}) = \begin{pmatrix} 1 & 1/2& 0\\
%% 0 & 1/2 & 1 \\
%% 0 & 0 & 0 \end{pmatrix} \begin{pmatrix} 1/2\\
%% 1/2\\
%% 0
%% \end{pmatrix} = \begin{pmatrix} 3/4\\
%% 1/4\\
%% 0 \end{pmatrix} \kbordermatrix{ & \\
%% \text{BB} & 3/(2^2)\\
%% \text{Bb} & 1/(2^2)\\
%% \text{bb} & 0 },\]
%% and after the third generation we get,
%% \[
%% A^{3}x_{0}= \begin{pmatrix} 1 & 1/2& 0\\
%% 0 & 1/2 & 1\\
%% 0 & 0 &0
%% \end{pmatrix} \begin{pmatrix} 3/4\\
%% 1/4\\
%% 0 
%% \end{pmatrix} = \begin{pmatrix} 7/8\\
%% 1/8\\
%% 0\end{pmatrix} = \kbordermatrix{ &\\
%% \text{BB} & 7/(2^3)\\
%% \text{Bb} & 1/(2^3)\\
%% \text{bb} & 0}\].
%% From the above we can conclude that after the n$^{\text{th}}$ generation, we will have,
%% \[
%% A^{n}x_{0} = \begin{pmatrix} \frac{2^{n}-1}{2^{n}}\\
%% \frac{1}{2^n}\\
%% 0
%% \end{pmatrix}
%% = \kbordermatrix{ & \\
%% \text{BB} & 1- \frac{1}{2^{n}}\\
%% \text{Bb} & \frac{1}{2^{n}}\\
%% \text{bb} & 0}.\]
%% Thus, letting $n$ become very large, we have $1/(2^{n}) \rightarrow 0$ so after a long time and many generations, the probability distribution becomes 
%% \[\kbordermatrix{ & \\
%% \text{BB}& 1\\
%% \text{Bb} &0\\
%% \text{bb} &0
%% },\]
%% making BB genotype take over the population. 























\end{document} 
