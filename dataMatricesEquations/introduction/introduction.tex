\documentclass{ximera}


\title{Introduction}

\begin{document}
\begin{abstract}
  This is a concrete, nongeometric introduction to linear alegbra.
\end{abstract}
\maketitle

This is a brief course in linear algebra with an emphasis on the
connections between data, matrices, and equations. It is suitable as
either an introduction, or a refresher. The basic concepts of linear
algebra are introduced through concrete examples (mostly nongeometric)
culminating in the concept and computation of inverses of matrices.


It was authored by Fatheddin, Kim and Snapp early in the summer of
2024. It is licensed under a Creative Commons CC-BY-NC-SA 4.0 license. 



All typos, errors, flubs, etc. can be reported on GitHub as an issue at:
\begin{center}
\url{https://github.com/mooculus/la-carte/issues}
\end{center}


\section*{For the student}


Reading mathematics is \textbf{not} the same as reading a novel. To
read mathematics you need:
\begin{enumerate}
\item A pen.
\item Plenty of blank paper.
\item A willingness to write things down.
\end{enumerate}
As you read mathematics, you must work alongside the text itself. You
must write down each expression, and \textbf{think} about what you are
doing. Sometimes, we will tell you ``write with me,'' and that means
we think it would help your understanding to \textit{write as you
  read.}  You should work examples and fill in the details. This is
not an easy task; it is in fact \textbf{hard} work. However,
\textbf{mathematics is not a passive endeavor}. You, the reader, must \textbf{become a
  doer of mathematics}.


To encourage you to work through this course, we provide an online experience found at
\begin{center}
  \url{https://ximera.osu.edu/la-carte/dataMatricesEquations}
\end{center}

In this online, interactive text you will find many questions and
answer-blanks for you to fill-in. These are to \textbf{help you check
  your understanding}. If you don't get the answer correct, no
worries! You get as many tries as you like. However, if you get stuck,
\textbf{please reach out to a human}.


\end{document}
