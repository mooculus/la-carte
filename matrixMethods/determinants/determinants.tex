\documentclass{ximera}

\newcommand{\dfn}{\textbf}
\renewcommand{\vec}[1]{{\overset{\boldsymbol{\rightharpoonup}}{\mathbf{#1}}}\hspace{0in}}
%% Simple horiz vectors
\renewcommand{\vector}[1]{\left\langle #1\right\rangle}
\newcommand{\arrowvec}[1]{{\overset{\rightharpoonup}{#1}}}
\newcommand{\R}{\mathbb{R}}
\newcommand{\transpose}{\intercal}
\newcommand{\ro}{\texttt{R}}%% row operation
\newcommand{\dotp}{\bullet}%% dot product

\usetikzlibrary{calc,bending}
\tikzset{>=stealth}


\usepackage{mdframed} % For framing content
%\usepackage{ifthen}   % For conditional statements

% Define the 'concept' environment with an optional header
\newenvironment{concept}[1][]{%
  \begin{mdframed}[linecolor=black, linewidth=2pt, innertopmargin=5pt, innerbottommargin=5pt, skipabove=12pt, skipbelow=12pt]%
    \noindent\large\textbf{#1}\normalsize%
}{%
  \end{mdframed}%
}











%% \colorlet{textColor}{black}
%% \colorlet{background}{white}
%% \colorlet{penColor}{blue!50!black} % Color of a curve in a plot
%% \colorlet{penColor2}{red!50!black}% Color of a curve in a plot
%% \colorlet{penColor3}{red!50!blue} % Color of a curve in a plot
%% \colorlet{penColor4}{green!50!black} % Color of a curve in a plot
%% \colorlet{penColor5}{orange!80!black} % Color of a curve in a plot
%% \colorlet{penColor6}{yellow!70!black} % Color of a curve in a plot
%% \colorlet{fill1}{penColor!20} % Color of fill in a plot
%% \colorlet{fill2}{penColor2!20} % Color of fill in a plot
%% \colorlet{fillp}{fill1} % Color of positive area
%% \colorlet{filln}{penColor2!20} % Color of negative area
%% \colorlet{fill3}{penColor3!20} % Fill
%% \colorlet{fill4}{penColor4!20} % Fill
%% \colorlet{fill5}{penColor5!20} % Fill
%% \colorlet{gridColor}{gray!50} % Color of grid in a plot


\author{Bart Snapp \and Tae Eun Kim}



\title{Determinants}


\begin{document}
\begin{abstract}
  Determinants are numbers that give information about matrices.
\end{abstract}
\maketitle

% \begin{quote}
%   It is difficult to imagine a more fundamental single scalar to
%   associate with a square matrix, and experience has long demonstrated
%   its ability to contribute unique insight on both theoretical and
%   applied levels.

%   \hfill ---\link[D.\ Carlson and others]{https://a.co/d/cw805aA}
% \end{quote}

\begin{quote}
  The determinant of a square matrix is a single number. That number
  contains an amazing amount of information about the matrix.

  \hfill ---\link[G.\ Strang]{https://a.co/d/eE2OVbS}
\end{quote}

The determinant is a function of square matrices in the sense that it
takes a square matrix as input and yields a number as output. Viewing
a square matrix as a stack of row vectors (or a concatenation of
column vectors), the determinant may also be viewed as a function of
several input variables producing a scalar output. Regardless of the
perspective adopted, the determinant is a tool which extracts a
representative number that conveys a special kind of information about a
given matrix or the set of its constituent vectors.

%% TODO: COME BACK AND REWIRTE THIS PARAGRAPH
The chapter begins mildly with the determinant of the simple
$2 \times 2$ and $3 \times 3$ matrices and study its various properties both in
algebraic and geometric terms. These properties are then shown to
extend to higher dimensions. Different methods of calculation the
determinant of general $n \times n$ matrices are introduced. The connection
between the row operations studied in a previous chapter is presented
afterward. The chapter concludes by showing how to interpret the
determinant of a matrix, both when it is viewed as a data and as a
transformation.

\section{The determinants of small matrices}
Here we introduce the determinants of small matrices, in particular, $2 \times 2$ and $3 \times 3$ matrices.

\subsection{$2 \times 2$ matrices}
\begin{definition}
  The \dfn{determinant} of a $2 \times 2$ matrix $\begin{pmatrix} a & b \\
    c & d \end{pmatrix}$ is given by
  \[
    \det
    \begin{pmatrix}
      a & b\\
      c & d
    \end{pmatrix}
    =
    \begin{vmatrix}
      a & b\\
      c & d
    \end{vmatrix}
    = ad - bc.
  \]
\end{definition}

% Typically, when one computes the determinant of a $2\times 2$ matrix,
% we think of the terms as follows.
In the definition above, notice that we introduced two ways to denote
this number, $\det A$ and $|A|$.

In words, the determinant of a $2 \times 2$ matrix is the difference between
the product of diagonal terms and the product of anti-diagonal terms.
\begin{image}[0.75in]
  \begin{tikzpicture}
    \matrix (mtrx)  [matrix of math nodes,ampersand replacement=\&,%column sep=.1em,
    nodes={text height=1ex,text width=2ex}]
    {
      a \& b \\
      c \& d \\
    };
    \draw[thick] (mtrx-1-1.north) -| (mtrx-2-1.south west)
    -- (mtrx-2-1.south);
    \draw[thick] (mtrx-1-2.north) -| (mtrx-2-2.south east)
    -- (mtrx-2-2.south);
    \draw[ultra thick,red!20!white,->]
    (mtrx-1-2.center) -- (mtrx-2-1.center);
    \draw[draw,ultra thick,blue!20!white,->]
    (mtrx-1-1.center)  --  (mtrx-2-2.center);
    \matrix (mtrx2)  [matrix of math nodes,ampersand replacement=\&,nodes={text height=1ex,text width=2ex}]
    {
      a \& b\\
      c \& d\\
    };
    \node at (0,-.7) {${\color{blue!50!black}ad}-{\color{red!50!black}bc}$};
  \end{tikzpicture}
\end{image}


\begin{question}
  Let
  \[
    A =                         % base matrix
    \begin{pmatrix}
      1 & 3 \\ -2 & 2
    \end{pmatrix}, \quad
    B =                         % rows swapped from the base
    \begin{pmatrix}
      -2 & 2 \\ 1 & 3
    \end{pmatrix}, \quad
    C =                         % 2nd row is multiple of 1st
    \begin{pmatrix}
      1 & 3 \\ 3 & 9
    \end{pmatrix}, \quad
    D =                         % triangular
    \begin{pmatrix}
      1 & 3 \\ 0 & 2
    \end{pmatrix}
  \]
  Compute the following determinants.
  \begin{enumerate}
  \item $\det(A) = \answer{8}$
  \item $\det(B) = \answer{-8}$
  \item $\det(C) = \answer{0}$
  \item $\det(D) = \answer{2}$
  \end{enumerate}
\end{question}

% Why would anyone ever be interested in this? Well to start, given nonzero vectors

%% TK: Have we introduced the angle bracket notation for vectors?

To gain a geometric insight behind this definition, consider the
parallelogram spanned by two vectors
\[
  \vec{v} = \langle a, b \rangle
  \quad\text{and}\quad
  \vec{w} = \langle c, d \rangle,
\]
which correspond to the two rows of the matrix in the definition.
% Here we assume that the two vectors are not parallel so that they
% indeed span a parallelogram as shown below.

\begin{image}
  \begin{tikzpicture}
    \draw[fill=fill1!50!white,draw=none]
    (0, 0)         % starting point
    -- ++(5,1)     % move along this vector
    -- ++(2,3)     % then along this vector
    -- ++(-5,-1)   % then back along that vector
    -- cycle;      % and back to where you started
    \draw[->,ultra thick,penColor] (0,0) -- (5,1);
    \draw[->,ultra thick,penColor2] (0,0) -- (2,3);
    \draw[->,ultra thick,dashed,penColor2] (5,1) -- (7,4);
    \draw[->,ultra thick,dashed,penColor] (2,3) -- (7,4);
    \node[above,penColor] at (2.5,.5) {$\vec{v}$}; %% <a,b>
    \node[below right,penColor2] at (1,1.5) {$\vec{w}$}; %% <c,d>
    \draw[->] (.5,.1) arc[radius=.5cm,start angle=11.3,end angle=56.3];
    \node[above right] at (.4,.2) {$\theta$};
  \end{tikzpicture}
\end{image}

The determinant gives the \textit{signed} area of this parallelogram,
where the sign of the area is determined by the orientation of the two
vectors $\vec{v}$ and $\vec{w}$. If the second vector $\vec{w}$ lies
counterclockwise from the first vector $\vec{v}$ as illustrated above,
then the sign is positive. If $\vec{w}$ lies clockwise from $\vec{v}$,
then the sign is negative. If two vectors are parallel so that one
lies neither counterclockwise nor clockwise from the other, the
determinant is zero.

To understand how the determinant is related to the area of the
parallelogram, make a rectangle around the parallelogram as below.

\begin{image}
  \begin{tikzpicture}
    \draw[fill=fill1!50!white,draw=none]
    (0, 0)         % starting point
    -- ++(5,1)     % move along this vector
    -- ++(2,3)     % then along this vector
    -- ++(-5,-1)   % then back along that vector
    -- cycle;      % and back to where you started

    \draw[fill=fill5] (5, 0) -- (7,0) -- (7,1) -- (5,1) -- cycle;
    \draw[fill=fill5] (0, 3) -- (2,3) -- (2,4) -- (0,4) -- cycle;

    \draw[fill=fill3] (0, 0) -- (5,0) -- (5,1) -- cycle;
    \draw[fill=fill3] (2,3) -- (7,4) -- (2,4) -- cycle;

    \draw[fill=fill4] (0, 0) -- (2,3) -- (0,3) -- cycle;
    \draw[fill=fill4] (5, 1) -- (7,1) -- (7,4) -- cycle;

    \draw[->,ultra thick,penColor] (0,0) -- (5,1);
    \draw[->,ultra thick,penColor2] (0,0) -- (2,3);
    \draw[->,ultra thick,dashed,penColor2] (5,1) -- (7,4);
    \draw[->,ultra thick,dashed,penColor] (2,3) -- (7,4);

    \draw[decoration={brace,mirror,raise=.1cm},decorate,thin] (0,0)--(5,0);
    \node[below] at (2.5,-.2) {$a$};

    \draw[decoration={brace,mirror,raise=.1cm},decorate,thin] (7,4)--(2,4);
    \node[above] at (4.5,4.2) {$a$};

    \draw[decoration={brace,mirror,raise=.1cm},decorate,thin] (7,0)--(7,1);
    \node[right] at (7.2,.5) {$b$};

    \draw[decoration={brace,mirror,raise=.1cm},decorate,thin] (0,4)--(0,3);
    \node[left] at (-.2,3.5) {$b$};

    \draw[decoration={brace,mirror,raise=.1cm},decorate,thin] (7,1)--(7,4);
    \node[right] at (7.2,2.5) {$d$};

    \draw[decoration={brace,mirror,raise=.1cm},decorate,thin] (0,3)--(0,0);
    \node[left] at (-.2,1.5) {$d$};

    \draw[decoration={brace,mirror,raise=.1cm},decorate,thin] (0,3)--(0,0);
    \node[left] at (-.2,1.5) {$d$};

    \draw[decoration={brace,raise=.1cm},decorate,thin] (0,4)--(2,4);
    \node[above] at (1,4.2) {$c$};

    \draw[decoration={brace,mirror,raise=.1cm},decorate,thin] (5,0)--(7,0);
    \node[below] at (6,-.2) {$c$};

    \node[above,penColor] at (2.5,.5) {$\vec{v}$}; %% <a,b>
    \node[below right,penColor2] at (1,1.5) {$\vec{w}$}; %% <c,d>
  \end{tikzpicture}
\end{image}

Now, we see that the area of the parallelogram is
\[
  \text{Area of rectangle} - \text{Area of other regions}
\]
and this is
\begin{align*}
  (a+c)(b+d) - \left(cd + ab + 2bc \right)&= ab + ad + bc + cd - cd - ab-2bc\\
                                          &=  ad - bc\\
                                          &= \det
                                            \begin{pmatrix}
                                              a & c\\
                                              b & d
                                            \end{pmatrix}.
\end{align*}

\begin{question}
  Let $A = (5,1)$ and $B = (1,4)$ be two points in the plane as shown below. Find
  the area of the triangle $\triangle OAB$ using determinant.
  \begin{image}[0.6\textwidth]
    \begin{tikzpicture}
      % shade the region first
      \fill[fill=fill1!50!white,draw=none]
      (0, 0) -- ++(5,1) -- ++(-4,3) -- cycle;

      % then draw grids
      \draw[step=1cm,lightgray,very thin] (-0.5,-0.5) grid (6.5,5.5);
      \draw[->,thin] (-0.5,0) -- (6.5,0) node[right] {$x$};
      \draw[->,thin] (0,-0.5) -- (0,5.5) node[above] {$y$};

      % then draw triangle and label vertices
      \draw[penColor,ultra thick] (0, 0) -- ++(5,1) -- ++(-4,3) -- cycle;
      \node[below left] at (0,0) {$O = (0,0)$};
      \node[right,fill=white] at (5,1) {$A = (5,1)$};
      \node[above,fill=white] at (1,4) {$B = (1,4)$};
    \end{tikzpicture}
  \end{image}
\end{question}

\subsection{$3 \times 3$ matrices}
Moving on to the next dimension:
\begin{definition}
  The \dfn{determinant} of a $3 \times 3$ matrix is given by
  \begin{align*}
    \det
    \begin{pmatrix}
      a_1 &  a_2 & a_3 \\
      b_1 &  b_2 & b_3 \\
      c_1 &  c_2 & c_3
    \end{pmatrix}
    & =
      a_1
      \begin{vmatrix}
        b_2 & b_3 \\
        c_2 & c_3
      \end{vmatrix}
      - a_2
      \begin{vmatrix}
        b_1 & b_3 \\
        c_1 & c_3
      \end{vmatrix}
      + a_3
      \begin{vmatrix}
        b_1 & b_2 \\
        c_1 & c_2
      \end{vmatrix} \\
          & = a_1(b_2c_3 - b_3c_2) - a_2(b_1c_3 - b_3c_1) + a_3(b_1c_2 - b_2c_1).
  \end{align*}
\end{definition}

Note that the determinant of a $3 \times 3$ matrix is related to the
determinants of three $2 \times 2$ submatrices, each of which is obtained
from the original matrix by removing the first row and a column at a
time.

\begin{center}
  \begin{tikzpicture}
    \node at (0,0) {$\det
      \begin{pmatrix}
        a_1 & a_2 & a_3\\
        b_1 & b_2 & b_3\\
        c_1 & c_2 & c_3
      \end{pmatrix}
      = a_1
      \begin{vmatrix}
        b_2 & b_3\\
        c_2 & c_3
      \end{vmatrix}
      -a_2
      \begin{vmatrix}
        b_1 & b_3\\
        c_1 & c_3
      \end{vmatrix}
      +a_3
      \begin{vmatrix}
        b_1 & b_2\\
        c_1 & c_2
      \end{vmatrix}$};

    \draw[ultra thick,->,gray] (-.27,-1.3) -- (-.27,-.7);
    \draw[ultra thick,->,gray] (1.76,-1.3) -- (1.76,-.7);
    \draw[ultra thick,->,gray] (3.8,-1.3) -- (3.8,-.7);

    \node at (-.27,-2) {\scalebox{.7}{
        $\begin{pmatrix}
          a_1 & \bullet & \bullet\\
          \bullet & b_2 & b_3\\
          \bullet & c_2 & c_3
        \end{pmatrix}$}};

    \node at (1.76,-2) {\scalebox{.7}{
        $\begin{pmatrix}
          \bullet & a_2 & \bullet\\
          b_1 & \bullet & b_3\\
          c_1 & \bullet & c_3
        \end{pmatrix}$}};

    \node at (3.8,-2) {\scalebox{.7}{
        $\begin{pmatrix}
          \bullet & \bullet & a_3\\
          b_1 & b_2 & \bullet\\
          c_1 & c_2 & \bullet
        \end{pmatrix}$}};
  \end{tikzpicture}
\end{center}
Note also the alternation of the signs from term to term. We will
revisit this feature later.

\begin{question}
  Compute the following determinant.
  \[
    \det
    \begin{pmatrix}
      1 & 4 & 7\\
      2 & 5 & 8\\
      3 & 6 & 9\\
    \end{pmatrix}
    \begin{prompt}
      = \answer{0}
    \end{prompt}
  \]
\end{question}

Recall that the determinant of a $2 \times 2$ matrix gives the signed area
of the parallelogram spanned by its two rows. A similar geometric
interpretation holds for the determinant of a $3 \times 3$ matrix, namely,
it measures the \textit{signed volume} of the parallelipiped spanned
by its three rows. The illustration below was drawn assuming that the
three rows are not co-planar, meaning not all three vectors are on the
same plane.

\begin{image}[.75\textwidth]
  \begin{tikzpicture}
    % u = <5,1>, v = <3,2.5>, w = <-1,3>

    % visible outlines
    \draw (0,0) -- ++(5,1) -- ++(-1,4) -- ++(-5,-1) -- cycle; % front
    \draw (5,1) -- ++(3,2.5) -- ++(-1,4) -- ++(-3,-2.5);     % right
    \draw (-1,4) -- ++(3,2.5) -- ++(5,1);                   % top

    % invisible dashed lines
    \draw[dashed] (0,0) -- ++(3,2.5) -- ++(5,1);
    \draw[dashed] (3,2.5) -- ++(-1,4);

    % arrows
    \draw[->,ultra thick,penColor] (0,0) -- ++(5,1);    % u
    \draw[->,ultra thick,penColor2] (0,0) -- ++(3,2.5); % v
    \draw[->,ultra thick,penColor4] (0,0) -- ++(-1,4);  % w

    % labels
    \node[penColor, below] at (5,.5) {$\vec{u} = \langle a_1,a_2,a_3 \rangle$};
    \node[penColor2,right] at (2.3,1.7) {$\vec{v} = \langle b_1,b_2,b_3 \rangle$};
    \node[penColor4,left] at (-1,3.5) {$\vec{w} = \langle c_1,c_2,c_3 \rangle$};
  \end{tikzpicture}
\end{image}

The sign of the volume is determined by the \textit{right-hand
  rule}. The rule dictates that the sign of the volume is positive if,
when pointing your index finger of your right hand in the direction of
the first row vector $\vec{u}$ and your middle finger in the direction
of the second row vector $\vec{v}$, your thumb points in the general
direction of the third vector $\vec{w}$. The figure above adheres to
the right-hand rule (confirm it!), so the determinant yields the
positive volume of the shape.

Usual approaches to compute the volume of the parallelipiped from the
coordinates of verticies would require evaluation of square roots and
trigonometric functions. The determinant formula finds the same volume
only using simple arithmetic operations. This is a remarkable way to
calculate the volume!


\section{Determinants in higher dimensions}
The determinant function is defined not just for small
$2 \times 2$ or $3 \times 3$ matrices, but for square matrices of any
size. There are a couple of different ways to compute the determinant:
\begin{itemize}
\item using the cofactor expansion formula, a.k.a., the Laplace expansion;
\item using the Leibniz formula;
\item using row reductions.
\end{itemize}
In this section we will introduce the cofactor expansion which is a
generalization of the $3 \times 3$ formula already seen. We will then see
how determinants can be computed using row reductions. Unlike the
other approaches, this does not provide a concrete formula to
calculate determinants out of the elements of a given matrix but,
rather, is based on fundamental properties that the determinant
function satisfies. The Leibniz formula is based on mathematical
objects called permutation. It will not be presented here, and we
refer interested readers to other linear algebra textbooks such as
[FILL IN].

\subsection{The cofactor expansion}
Recall the $3 \times 3$ determinant formula:
\begin{center}
  \begin{tikzpicture}
    \node at (0,0) {$\det
      \begin{pmatrix}
        a_1 & a_2 & a_3\\
        b_1 & b_2 & b_3\\
        c_1 & c_2 & c_3
      \end{pmatrix}
      = a_1
      \begin{vmatrix}
        b_2 & b_3\\
        c_2 & c_3
      \end{vmatrix}
      -a_2
      \begin{vmatrix}
        b_1 & b_3\\
        c_1 & c_3
      \end{vmatrix}
      +a_3
      \begin{vmatrix}
        b_1 & b_2\\
        c_1 & c_2
      \end{vmatrix}$};

    \draw[ultra thick,->,gray] (-.27,-1.3) -- (-.27,-.7);
    \draw[ultra thick,->,gray] (1.76,-1.3) -- (1.76,-.7);
    \draw[ultra thick,->,gray] (3.8,-1.3) -- (3.8,-.7);

    \node at (-.27,-2) {\scalebox{.7}{
        $\begin{pmatrix}
          a_1 & \bullet & \bullet\\
          \bullet & b_2 & b_3\\
          \bullet & c_2 & c_3
        \end{pmatrix}$}};

    \node at (1.76,-2) {\scalebox{.7}{
        $\begin{pmatrix}
          \bullet & a_2 & \bullet\\
          b_1 & \bullet & b_3\\
          c_1 & \bullet & c_3
        \end{pmatrix}$}};

    \node at (3.8,-2) {\scalebox{.7}{
        $\begin{pmatrix}
          \bullet & \bullet & a_3\\
          b_1 & b_2 & \bullet\\
          c_1 & c_2 & \bullet
        \end{pmatrix}$}};
  \end{tikzpicture}
\end{center}
This is an example of \textit{recursive} formulas in the sense that
the determinant of a $3 \times 3$ matrix is computed in terms of the
determinants of smaller submatrices. Note that this requires one
already knows how to compute the $2 \times 2$ determinants. The cofactor
expansion formula is the generalization of this idea to all
dimensions.

To kickstart the recursion process, we need a starting point. Viewing
a scalar as an $1 \times 1$ matrix, we first establish that
\[
  \det \left( a \right) = a.
\]
We then introduce a couple of convenient notation.

\begin{definition}
  Let $n$ be a positive integer and let $A$ be an $n \times n$
  matrix.
  \begin{itemize}
  \item The $(i,j)$ submatrix of $A$, denoted $A_{ij}$, is the $(n-1) \times
    (n-1)$ matrix obtained by removing the $i$th row and the $j$th
    column of $A$.
  \item The $(i,j)$ cofactor of $A$, denoted $C_{ij}$, is defined by
    \[
      C_{ij} = (-1)^{i+j} \det(A_{ij}).
    \]
  \end{itemize}
\end{definition}

For example, for a $3 \times 3$ matrix
\[
  A =
  \begin{pmatrix}
    a_{11} & a_{12} & a_{13} \\
    a_{21} & a_{22} & a_{23} \\
    a_{31} & a_{32} & a_{33}
  \end{pmatrix},
\]
three of the nine possible cofactors are
\begin{align*}
  C_{11}
  & = (-1)^{1+1} \det A_{11}
    = +
    \begin{vmatrix}
      a_{22} & a_{23} \\ a_{32} & a_{33}
    \end{vmatrix}, \\
  C_{12}
  & = (-1)^{1+2} \det A_{12}
    = -
    \begin{vmatrix}
      a_{21} & a_{23} \\ a_{31} & a_{33}
    \end{vmatrix}, \\
  C_{13}
  & = (-1)^{1+3} \det A_{13}
    = +
    \begin{vmatrix}
      a_{21} & a_{22} \\ a_{31} & a_{32}
    \end{vmatrix}.
\end{align*}
These are the three determinants that appear in the previous
definition of the $3 \times 3$ determinant formula, which is now succintly
written as
\[
  \det A = a_{11} C_{11} + a_{12} C_{12} + a_{13} C_{13}.
\]
With the introduction of clever and proper notation, the $3 \times 3$
determinant formula is now written with order and elegance.

Note that the factor $(-1)^{i+j}$ in the definition of the cofactor
takes care of the sign alternation. The sign pattern generated by this
factor displays the following ``checkerboard'' pattern:
\[
  \begin{pmatrix}
    + & - & + \\
    - & + & - \\
    + & - & +
  \end{pmatrix}
\]

We now present as a theorem the cofactor expansion formula for the
determinant of general square matrices.
\begin{theorem}[General cofactor expansion]
  Let $n$ be a positive integer and let $A = (a_{ij})$ be an
  $n \times n$ matrix. Then
  \begin{enumerate}
  \item for any fixed $i \in \{1, 2, \ldots, n\}$,
    \begin{align*}
      \det A
      & = \sum_{j=1}^n a_{ij} C_{ij}
        \tag{cofactor expansion along $i$th row} \\
      & = a_{i1} C_{i1} + a_{i2} C_{i2} + \cdots + a_{in} C_{in};
    \end{align*}
  \item for any fixed $j \in \{1, 2, \ldots, n\}$,
    \begin{align*}
      \det A
      & = \sum_{i=1}^n a_{ij} C_{ij}
        \tag{cofactor expansion along $j$th column} \\
      & = a_{1j} C_{1j} + a_{2j} C_{2j} + \cdots + a_{nj} C_{nj}.
    \end{align*}
  \end{enumerate}
\end{theorem}


% %
% Incidentally, the determinant can also be computed using the
% entries of the first column as coefficients. Below is the illustration
% of this idea for a $3 \times 3$ matrix:

% \begin{center}
%   \begin{tikzpicture}
%     \node at (0,0) {$\det
%       \begin{pmatrix}
%         a_1 & a_2 & a_3\\
%         b_1 & b_2 & b_3\\
%         c_1 & c_2 & c_3
%       \end{pmatrix}
%       = a_1
%       \begin{vmatrix}
%         b_2 & b_3\\
%         c_2 & c_3
%       \end{vmatrix}
%       -b_1
%       \begin{vmatrix}
%         a_2 & a_3 \\
%         c_2 & c_3
%       \end{vmatrix}
%       +c_1
%       \begin{vmatrix}
%         a_2 & a_3 \\
%         b_2 & b_3
%       \end{vmatrix}$};

%     \draw[ultra thick,->,gray] (-.27,-1.3) -- (-.27,-.7);
%     \draw[ultra thick,->,gray] (1.76,-1.3) -- (1.76,-.7);
%     \draw[ultra thick,->,gray] (3.8,-1.3) -- (3.8,-.7);

%     \node at (-.27,-2) {\scalebox{.7}{
%         $\begin{pmatrix}
%           a_1 & \bullet & \bullet\\
%           \bullet & b_2 & b_3\\
%           \bullet & c_2 & c_3
%         \end{pmatrix}$}};

%     \node at (1.76,-2) {\scalebox{.7}{
%         $\begin{pmatrix}
%           \bullet & a_2 & a_3\\
%           b_1 & \bullet & \bullet\\
%           \bullet & c_2 & c_3
%         \end{pmatrix}$}};

%     \node at (3.8,-2) {\scalebox{.7}{
%         $\begin{pmatrix}
%           \bullet & a_2 & a_3\\
%           \bullet & b_2 & b_3\\
%           c_1 & \bullet & \bullet
%         \end{pmatrix}$}};
%   \end{tikzpicture}
% \end{center}
% Using the cofactor notation, this is succinctly written as
% \[
%   \det A = a_1 C_{11} + b_1 C_{21} + c_1 C_{31}.
% \]

% As a matter of fact, the cofactor expansion works along any row or any
% column, not just the first row or the first column. No matter which
% row or column is used for expansion, the cofactor ensures that the
% determinants of submatrices are given alternating signs. In
% particular, the sign pattern generated by the factor $(-1)^{i+j}$
% displays the following ``checkerboard'' pattern:
% \[
%   \begin{pmatrix}
%     + & - & + \\
%     - & + & - \\
%     + & - & +
%   \end{pmatrix}
% \]

With this theorem, we see that the $3 \times 3$ determinant formula
previously presented is merely the cofactor expansion along the first
row of a matrix. The two parts of the theorem tells us that we can
expand along any row or any column of the given matrix and still
obtain the same number, the determinant.

For instance, to calculate the determinant via expansion along the
second row, we have
\begin{center}
  \begin{tikzpicture}
    \node at (0,0) {$\det
      \begin{pmatrix}
        a_{11} & a_{12} & a_{13} \\
        a_{21} & a_{22} & a_{23} \\
        a_{31} & a_{32} & a_{33}
      \end{pmatrix}
      = -a_{21}
      \begin{vmatrix}
        a_{12} & a_{13} \\ a_{32} & a_{33}
      \end{vmatrix}
      + a_{22}
      \begin{vmatrix}
        a_{11} & a_{13} \\ a_{31} & a_{33}
      \end{vmatrix}
      - a_{23}
      \begin{vmatrix}
        a_{11} & a_{12} \\ a_{31} & a_{32}
      \end{vmatrix}$};

    \draw[ultra thick,->,gray] (-.27,-1.3) -- (-.27,-.7);
    \draw[ultra thick,->,gray] (2.3,-1.3) -- (2.3,-.7);
    \draw[ultra thick,->,gray] (4.8,-1.3) -- (4.8,-.7);

    \node at (-.27,-2) {\scalebox{.7}{
        $\begin{pmatrix}
          \bullet & a_{12} & a_{13} \\
          a_{21} & \bullet & \bullet \\
          \bullet & a_{32} & a_{33}
        \end{pmatrix}$}};

    \node at (2.3,-2) {\scalebox{.7}{
        $\begin{pmatrix}
          a_{11} & \bullet & a_{13} \\
          \bullet & a_{22} & \bullet \\
          a_{31} & \bullet & a_{33}
        \end{pmatrix}$}};

    \node at (4.8,-2) {\scalebox{.7}{
        $\begin{pmatrix}
          a_{11} & a_{12} & \bullet \\
          \bullet & \bullet & a_{23} \\
          a_{31} & a_{32} & \bullet
        \end{pmatrix}$}};
  \end{tikzpicture}
\end{center}

Having multiple alternate options, we may at times compute
determinants efficiently by choosing to expand along a convenient row
or column. Most notably, if a row or a column contains many zeros,
then we can exploit the structure to drastically simplify calculation.

\begin{example}[$3 \times 3$ cofactor expansion]
  In calculating
  \[
    \det
    \begin{pmatrix}
      1 & 2 & 5 \\
      0 & 0 & 3 \\
      6 & 8 & 4
    \end{pmatrix},
  \]
  it is convenient to use the cofactor expansion along the second row:
  \[
    - 0
    \begin{vmatrix}
      2 & 5 \\ 8 & 4
    \end{vmatrix}
    + 0
    \begin{vmatrix}
      1 & 5 \\ 6 & 4
    \end{vmatrix}
    - 3
    \begin{vmatrix}
      1 & 2 \\ 6 & 8
    \end{vmatrix}
    = -3 (1 \cdot 8 - 2 \cdot 6) = -3 (-4) = 12.
  \]
  Because of the two zeros in the second row, only one $2 \times 2$
  determinant needs to be computed.
\end{example}

\begin{question}
  Calculate
  \[
    \det
    \begin{pmatrix}
      1 & 2 & 3 & 3 & 8 \\
      4 & 5 & 0 & 4 & 2\\
      0 & 0 & 0 & 0 & 1 \\
      7 & 8 & 0 & 5 & 6 \\
      0 & 2 & 0 & 0 & 9
    \end{pmatrix} = \answer{48}.
  \]
  \textit{Hint.} At each step, expand along the row or the column with
  the most number of zeros.
  \begin{prompt}
    \answer{48}
  \end{prompt}
\end{question}

% Moved the following paragraph on the rule of Sarrus to the exercise section.
%
% Another way to carry out this computation to follow the \textit{rule
% of Sarrus}:
% %% Commented because this is not a particularly useful method in this course.
% %% %% Credit for the following image goes to
% %% %% http://tex.stackexchange.com/a/257063
% \begin{image}
%   \begin{tikzpicture}
%     \matrix (mtrx)  [matrix of math nodes,ampersand replacement=\&,column sep=1em, nodes={text height=1ex,text width=2ex}]
%     {
%       a_1 \& a_2 \& a_3 \& a_1 \& a_2         \\
%       b_1 \& b_2 \& b_3 \& b_1 \& b_2         \\
%       c_1 \& c_2 \& c_3 \& c_1 \& c_2         \\
%     };
%     \matrix (mtrx3)  [matrix of math nodes,ampersand replacement=\&,column sep=1em, nodes={text height=1ex,text width=2ex}] at (5,0)
%     {
%       a_1 \& a_2 \& a_3 \& a_1 \& a_2         \\
%       b_1 \& b_2 \& b_3 \& b_1 \& b_2         \\
%       c_1 \& c_2 \& c_3 \& c_1 \& c_2         \\
%     };
%     \draw[thick] (mtrx-1-1.north) -| (mtrx-3-1.south west)
%     -- (mtrx-3-1.south);
%     \draw[thick] (mtrx-1-3.north) -| (mtrx-3-3.south east)
%     -- (mtrx-3-3.south);
%     \draw[ultra thick,red!20!white,->]
%     (mtrx3-3-1.center) edge (mtrx3-1-3.center)
%     (mtrx3-3-2.center) edge (mtrx3-1-4.center)
%     (mtrx3-3-3.center)  --  (mtrx3-1-5.center);
%     \draw[draw,ultra thick,blue!20!white,->]
%     (mtrx-1-1.center) edge (mtrx-3-3.center)
%     (mtrx-1-2.center) edge (mtrx-3-4.center)
%     (mtrx-1-3.center)  --  (mtrx-3-5.center);
%     \matrix (mtrx2)  [matrix of math nodes,ampersand replacement=\&,column sep=1em, nodes={text height=1ex,text width=2ex}]
%     {
%       a_1 \& a_2 \& a_3 \& a_1 \& a_2         \\
%       b_1 \& b_2 \& b_3 \& b_1 \& b_2         \\
%       c_1 \& c_2 \& c_3 \& c_1 \& c_2         \\
%     };
%     \matrix (mtrx4)  [matrix of math nodes,ampersand replacement=\&,column sep=1em, nodes={text height=1ex,text width=2ex}] at (5,0)
%     {
%       a_1 \& a_2 \& a_3 \& a_1 \& a_2         \\
%       b_1 \& b_2 \& b_3 \& b_1 \& b_2         \\
%       c_1 \& c_2 \& c_3 \& c_1 \& c_2         \\
%     };
%     \draw[thick] (mtrx4-1-1.north) -| (mtrx4-3-1.south west)
%     -- (mtrx4-3-1.south);
%     \draw[thick] (mtrx4-1-3.north) -| (mtrx4-3-3.south east)
%     -- (mtrx4-3-3.south);
%     \node at (2.5,-1) {${\color{blue!50!black}a_1b_2c_3}+{\color{blue!50!black}a_2b_3c_1}+{\color{blue!50!black}a_3b_1c_2}-{\color{red!50!black}a_3b_2c_1}-{\color{red!50!black}a_1b_3c_2}-{\color{red!50!black}a_2b_1c_3}$};
%     \node at (0,1) {{\color{blue!50!black}positive terms}};
%     \node at (5,1) {{\color{red!50!black}negative terms}};
%   \end{tikzpicture}
% \end{image}

% The idea is to attach the first two columns of the given matrix to the
% right of the third column, then to add the products of the down-right
% diagonals (blue  arrows), then to subtract the products of the
% up-right diagonals (red arrows).

% \begin{remark}
%   The rule of Sarrus only works for $3 \times 3$ matrices.
% \end{remark}

% - but the drawback (inefficiency) -- cofactor expanion requires about
% n! operations, show how large 10! or 15! are.

% TODO: Motivate the subsection
% - as an alternate efficient method to calculate determinants in
% higher dimensions, we focus on key properties of determinants which
% generalizes the notion of area or volume in higher dimensions
% - mention that it may be advantageous to view the determinant as a function of
% multiple (row) vectors instead of square matrices
% - present fundamental properties of the determinants in any
% dimensions, but illustrates them in 2-D





\subsection{Properties-oriented definition}

Another way to define the determinant function is by specifying the
properties that it satisfies, instead of providing a formula such as
the cofactor expansion.

\begin{definition}
  % Let $n$ be a positive integer. The \dfn{determinant} is a scalar
  % function of an $n \times n$ square matrix
  % \[
  %   \det: \mathbb{R}^{n \times n} \to \mathbb{R}
  % \]
  The \dfn{determinant} is a scalar function of a square matrix
  satisfying the following properties:
  \begin{enumerate}
  \item the determinant of the identity matrix is 1;
  \item the determinant changes signs when two rows are interchanged;
  \item the determinant is a linear function of each row while other
    rows are held fixed.
  \end{enumerate}
\end{definition}

The properties in the definition can be considered as characterization
of the notion of volume for higher dimensions, which is consistent
with the geometric interpretation of $2 \times 2$ and $3 \times 3$ determinants
previously presented, if one agrees to view the notion of area as a
2-D analogue of the volume.

It turns out that there is one and only function that satisfies the
properties, hence it makes sense to give it a name, the
determinant. That is, the definition above is unambiguous and
well-defined. However, we will not present the technical proof of the
existence and the uniqueness of the determinant here. Instead, we will
focus on understanding each of the properties and their connections to
row operations.

We will begin by confirming the three properties indeed hold true for
$2 \times 2$ matrices based on the previously given formula
\[
  \det
  \begin{pmatrix}
    a & b \\ c & d
  \end{pmatrix}
  = ad - bc
\]
and its geometric interpretation as the signed area of the
parallelogram spanned by its two rows.

\paragraph{Property (a): determinant of identity.}
The first property is readily confirmed for $I_2$ using the
$2 \times 2$ determinant formula:
\begin{align*}
  \det
  \begin{pmatrix}
    1 & 0 \\ 0 & 1
  \end{pmatrix}
  & = \text{(product of diagonals)} - \text{(product of
    anti-diagonals)} \\
      & = 1 \cdot 1 - 0 \cdot 0 = 1.
\end{align*}

Geometrically, $\det I_2$ computes the signed area of the
parallelogram spanned by $\vec{e}_1 = \langle 1, 0 \rangle$ and
$\vec{e}_2 = \langle 0, 1 \rangle$, which is the unit square with area
1. Note that the second vector $\vec{e}_2$ (facing due
north) lies counterclockwise from the first vector
$\vec{e}_1$ (facing due east), so the area is indeed
endowed with the positive sign.

\begin{image}[0.5\textwidth]
  \begin{tikzpicture}
    \draw[fill=fill1!50!white,draw=none]
    (0, 0)                      % starting point
    -- ++(5,0)                   % move along this vector
    -- ++(0,5)                   % then along this vector
    -- ++(-5,0)                  % then back along that vector
    -- cycle;                    % and back to where you started
    \draw[->,ultra thick,penColor] (0,0) -- (5,0);
    \draw[->,ultra thick,penColor2] (0,0) -- (0,5);
    \draw[->,ultra thick,dashed,penColor2] (5,0) -- (5,5);
    \draw[->,ultra thick,dashed,penColor] (0,5) -- (5,5);
    \node[above,penColor] at (2.5,-1) {$\vec{e}_1 = \langle 1, 0 \rangle$};
    \node[left,penColor2] at (-.25,2.5) {$\vec{e}_2 = \langle 0,1 \rangle$};
    \draw[->,thick] (.5,0) -- (.5,.5) -- (0,.5);
    \node[above right] at (.5,.5) {$\theta = \pi/2$};
    \node at (2.5,2.5) {$\det\begin{pmatrix} 1 & 0 \\ 0 & 1 \end{pmatrix} = 1$};
  \end{tikzpicture}
\end{image}


\paragraph{Property (b): sign reversal.}
The second property says that the determinant changes signs when two
rows are interchanged. One can check this algebraically for the
$2 \times 2$ case as follows:
\[
  \det
  \begin{pmatrix}
    c & d \\
    a & b
  \end{pmatrix}
  = bc - ad = -(ad - bc) =
  - \det
  \begin{pmatrix}
    a & b \\
    c & d
  \end{pmatrix}.
\]

Geometrically speaking, interchanging two rows of a $2 \times 2$ matrix
boils down to reversing the orientation of the parallelogram spanned
by its two row vectors, which has the very effect of changing the sign
of the area, while keeping the magnitude unchanged.

\begin{image}[.98\textwidth]
  \begin{tikzpicture}
    \draw[fill=fill1!50!white,draw=none]
    (0, 0)         % starting point
    -- ++(5,1)     % move along this vector
    -- ++(2,3)     % then along this vector
    -- ++(-5,-1)   % then back along that vector
    -- cycle;      % and back to where you started
    \draw[->,ultra thick,penColor] (0,0) -- (5,1);
    \draw[->,ultra thick,penColor2] (0,0) -- (2,3);
    \draw[->,ultra thick,dashed,penColor2] (5,1) -- (7,4);
    \draw[->,ultra thick,dashed,penColor] (2,3) -- (7,4);
    \node[below,penColor] at (4,0.5) {$\vec{v} = \langle a,b \rangle$}; %% <a,b>
    \node[left,penColor2] at (1.2,2.4) {$\vec{w} = \langle c,d \rangle$}; %% <c,d>
    \draw[->,thick] (.98,0.2) arc[radius=1cm,start angle=11.3,end angle=56.3];
    \node[above right] at (.4,.2) {$\theta$};
    \node at (3.5, 2) {$\det\begin{pmatrix} a & b \\ c & d \end{pmatrix} > 0$};
    \node at (3.5, -1) {counterclockwise orientation};
  \end{tikzpicture}
  \hfill
  \begin{tikzpicture}
    \draw[fill=fill1!50!white,draw=none]
    (0, 0)         % starting point
    -- ++(5,1)     % move along this vector
    -- ++(2,3)     % then along this vector
    -- ++(-5,-1)   % then back along that vector
    -- cycle;      % and back to where you started
    \draw[->,ultra thick,penColor2] (0,0) -- (5,1);
    \draw[->,ultra thick,penColor] (0,0) -- (2,3);
    \draw[->,ultra thick,dashed,penColor] (5,1) -- (7,4);
    \draw[->,ultra thick,dashed,penColor2] (2,3) -- (7,4);
    \node[below,penColor2] at (4,0.5) {$\vec{v} = \langle a,b \rangle$}; %% <a,b>
    \node[left,penColor] at (1.2,2.4) {$\vec{w} = \langle c,d \rangle$}; %% <c,d>
    \draw[->] (.55,.83) arc[radius=1cm,end angle=11.3,start angle=56.3];
    \node[above right] at (.4,.2) {$\theta$};
    \node at (3.5, 2) {$\det\begin{pmatrix} c & d \\ a & b \end{pmatrix} < 0$};
    \node at (3.5, -1) {clockwise orientation};
  \end{tikzpicture}
\end{image}

%% TODO: Say more? Leave it out? Come back and decide.
An interesting and useful consequence of the sign reversal property is
that the determinant of a matrix with two identical rows is zero.

\paragraph{Property (c): linearity in each row.}
We illustrate the last property again in the $2 \times 2$ case. In
particular, we show that the determinant is linear in the first row
while the second row is held fixed. This amounts to showing:
\begin{itemize}
\item multiplying the first row by a scalar $k$, the determinant is
  multiplied by $k$;
\item adding a row vector to the first row, the two determinants are added.
\end{itemize}

Both can be confirmed algebraically using the $2 \times 2$ determinant
formula:
\[
  \det
  \begin{pmatrix}
    k a & k b \\
    c & d
  \end{pmatrix}
  = (k a)d - (k b)c
  = k (ad - bc)
  = k \det
  \begin{pmatrix}
    a & b \\
    c & d
  \end{pmatrix}
\]
and
\begin{align*}
  \det
  \begin{pmatrix}
    a_1 + a_2 & b_1 + b_2 \\
    c & d
  \end{pmatrix}
  & = (a_1 + a_2) d - (b_1 + b_2) c \\
              & = (a_1 d - b_1 c) + (a_2 d - b_2 c) \\
              & = \det
                \begin{pmatrix}
                  a_1 & b_1 \\
                  c & d
                \end{pmatrix}
                + \det
                \begin{pmatrix}
                  a_2 & b_2 \\
                  c & d
                \end{pmatrix}.
\end{align*}

The geometric illustration of this last property can be quite
satisfying. If the first row is multiplied by $k$, the area is also
scaled by the factor of $k$.

%% TODO: Can this be illustrated better?
\begin{image}[0.75\textwidth]
  \begin{tikzpicture}
    % shade
    \draw[fill=fill1!50!white,draw=none] % p-span(kv,w)
    (0, 0) -- ++(6,0) -- ++(2,3) -- ++(-6,0) -- cycle;
    \draw[fill=fill1!80!white,draw=none] % p-span(u,v)
    (0, 0) -- ++(2,0) -- ++(2,3) -- ++(-2,0) -- cycle;

    % vectors (arrows)
    \draw[->,ultra thick,penColor2] (0,0) -- (2,3); % w
    \draw[->,ultra thick,penColor4] (0,0) -- (6,0); % kv
    \draw[->,ultra thick,penColor] (0,0) -- (2,0);  % v

    % parallel mirrors
    \draw[->,ultra thick,dashed,penColor2] (6,0) -- ++(2,3);
    \draw[->,ultra thick,dashed,penColor4] (2,3) -- ++(6,0);
    \draw[->,ultra thick,dashed,penColor] (2,3) -- ++(2,0);

    % auxilliary lines
    \draw[-,dashed,penColor2] (2,0) -- (4,3);
    \draw[-,dashed,penColor2] (4,0) -- (6,3);

    % labels
    \node[below,penColor] at (1,-0.1) {$\vec{v}$};
    \node[below,penColor4] at (5,-0.1) {$k\vec{v}$};
    \node[left,penColor2] at (1.2,2.5) {$\vec{w}$};
  \end{tikzpicture}
\end{image}

If a new row $\vec{v}_2 = \langle a_2, b_2 \rangle$ is added to the first row
$\vec{v}_1 = \langle a_1, b_1 \rangle$, while the second row
$\vec{w} = \langle c,d \rangle$ is held fixed, the areas of two parallelograms
outlined by solid lines add up to the area of the parallelogram
outlined by dotted lines.
\begin{image}[0.75\textwidth]
  \begin{tikzpicture}
    % shade
    \draw[fill=fill1!50!white,draw=none] % det(3u,v)
    (0,0) -- ++(4,0) -- ++(-1,3) -- ++(-4,0) -- cycle;
    \draw[fill=fill1!80!white,draw=none] % det(u,v)
    (4,0) -- ++(2,2) -- ++(-1,3) -- ++(-2,-2) -- cycle;

    % vectors (arrows)
    \draw[->,ultra thick,penColor] (0,0) -- (4,0);
    \draw[->,ultra thick,penColor] (4,0) -- (6,2);
    \draw[->,ultra thick,penColor2] (0,0) -- (-1,3);

    \draw[-,ultra thick,penColor] (-1,3) -- (3,3);
    \draw[-,ultra thick,penColor] (3,3) -- (5,5);
    \draw[-,ultra thick,penColor2] (4,0) -- (3,3);
    \draw[-,ultra thick,penColor2] (6,2) -- (5,5);

    % auxilliary lines
    \draw[->,ultra thick,dashed,penColor4] (0,0) -- (6,2);
    \draw[->,ultra thick,dashed,penColor4] (-1,3) -- (5,5);

    % labels
    \node[below,penColor] at (3,-0.1) {$\vec{v}_1 = \langle a_1, b_1 \rangle$};
    \node[right,penColor] at (5,0.8) {$\vec{v}_2 = \langle a_2, b_2 \rangle$};
    \node[right,penColor4] at (6,2.2) {$\vec{v}_1 + \vec{v}_2$};
    \node[left,penColor2] at (-1,2) {$\vec{w} = \langle c,d \rangle$};
  \end{tikzpicture}
\end{image}

We close this section by listing additional properties of the
determinant that can be derived from the three defining properties.

%% TODO: Should we prove these properties? Can this be motivated better?
\begin{theorem}[Additional properties of determinant]
  Let $A, B$ be $n \times n$ matrices.
  \begin{enumerate}
  \item If two rows of $A$ are equal, then $\det (A) = 0$.
  \item Adding a multiple of one row to another row leaves $\det (A)$
    unchanged.
  \item If $A$ has a row of zeroes, then $\det (A) = 0$.
  \item If $A$ is triangular, then $\det (A)$ is the product of its
    diagonal entries.
  \item $A$ is invertible if and only if $\det (A) \neq 0$.
  \item $\det (AB) = \det(A) \det(B)$
  \item $\det (A^\transpose) = \det(A)$
  \end{enumerate}
\end{theorem}

Because of the last property, all the properties of the determinant
pertaining to rows of a matrix holds true for columns as well. For
example, if two columns of $A$ are equal, then $\det(A) = 0$.


\subsection{Computing determinants using row operations}

The previous definition tells us about various properties that the
determinant satisfies. Is it any useful in actually calculating the
determinant of a given matrix? We will find out, but before we do so,
we have a short list of properties that are most relevant in the
upcoming calculations.

\begin{concept}
  Here we summarize the effects of elementary row operations on
  determinants.
  \begin{itemize}
  \item If two rows of $A$ are interchanged to produce $B$, then
    $\det(B) = -\det(A)$. That is,
    \[
      A \xrightarrow{R_i \leftrightarrow R_j} B
      \quad\Longrightarrow\quad
      \det(B) = -\det(A).
    \]
  \item If one row of $A$ is multiplied by $k$ to produce $B$, then
    $\det(B) = k \det(A)$. That is,
    \[
      A \xrightarrow{R_i \to k R_i} B
      \quad\Longrightarrow\quad
      \det(B) = k\det(A).
    \]
  \item If a multiple of one row of $A$ is added to another row to
    produce $B$, then $\det(B) = \det(A)$. That is,
    \[
      A \xrightarrow{R_i \to R_i + c R_j} B
      \quad\Longrightarrow\quad
      \det(B) = \det(A).
    \]
  \end{itemize}
\end{concept}

Our strategy to compute $\det(A)$ is to reduce $A$ to a row echelon
form and use the property that the determinant of a triangular matrix
is the product of the diagonal terms. Note that we do not need to
reduce $A$ all the way to the reduced row echelon form because
calculating the determinant of the triangular matrix is already
a trivial task.

\begin{example}[Determinant via row operations]
  Compute $\det(A)$, where
  \[
    A =
    \begin{pmatrix}
      1 & 0 & 0 \\
      0 & 0 & -3 \\
      6 & 2 & 0
    \end{pmatrix}.
  \]
  %% TODO: Think about an alternate presentation/explanation of the
  %% solution. One idea is to have two columns where one records row
  %% operations while the other keeps track of the change in the
  %% determinant side by side.
  \begin{explanation}
    The row replacement $R_3 \to R_3 - 6R_1$ does not change the
    determinant:
    \[
      \det(A) =
      \begin{vmatrix}
        1 & 0 & 0 \\
        0 & 0 & -3 \\
        6 & 2 & 0
      \end{vmatrix}
      =
      \begin{vmatrix}
        1 & 0 & 0 \\
        0 & 0 & -3 \\
        0 & 2 & 0
      \end{vmatrix}.
    \]
    The row interchange $R_2 \leftrightarrow R_3$ reverses the sign of the
    determinant:
    \[
      \det(A) = -
      \begin{vmatrix}
        1 & 0 & 0 \\
        0 & 2 & 0 \\
        0 & 0 & -3
      \end{vmatrix}
      = - 1 \cdot 2 \cdot (-3) = 6.
    \]
  \end{explanation}
\end{example}

At times, we don't need to go all the way through.
\begin{example}[Row of zeros]
  Compute $\det(A)$, where
  \[
    A =
    \begin{pmatrix}
      1 & 5 & 4 \\
      3 & 15 & 12 \\
      12 & 72 & 23
    \end{pmatrix}.
  \]
  \begin{explanation}
    The row replacement $R_2 \to R_2 - 3R_1$ does not change the
    determinant, and it introduces a row of all zeros:
    \[
      \det(A) =
      \begin{vmatrix}
        1 & 5 & 2 \\
        3 & 15 & 12 \\
        12 & 72 & 23
      \end{vmatrix}
      =
      \begin{vmatrix}
        1 & 5 & 2 \\
        0 & 0 & 0 \\
        12 & 72 & 23
      \end{vmatrix}.
    \]
    If you remember the following property from a previous theorem
    \begin{center}
      If $A$ has a row of zeros, then $\det(A) =0$.
    \end{center}
    then you know that the answer is 0 and call it a day.

    However, even if you do not remember it, the answer is only a
    couple of steps away. Carrying out the row replacement
    $R_3 \to R_3 - 12 R_1$ does not change the determinant:
    \[
      \det(A) =
      \begin{vmatrix}
        1 & 5 & 4 \\
        0 & 0 & 0 \\
        0 & 12 & -1
      \end{vmatrix}.
    \]
    Interchanging Row 2 and Row 3 reverses the sign of the
    determinant, but we note that this leaves us with an upper
    triangular matrix one of whose diagonal entries is zero, hence
    $\det(A) = 0$.
    \[
      \det(A) = -
      \begin{vmatrix}
        1 & 5 & 4 \\
        0 & 12 & -1 \\
        0 & 0 & 0
      \end{vmatrix} = -1 \cdot 12 \cdot 0 = 0.
    \]
  \end{explanation}
\end{example}

%% TODO: Maybe compare the values of n! and n^3 for a concrete n value
%% (n = 15 or 20) to give a feel for the difference.
\begin{remark}
  Computing determinants via row reduction is a much more efficient
  way than to use the cofactor expansion (or the Leibniz
  formula). Calculating the determinant of an $n \times n$ matrix using the
  cofactor expansion requires about $n!$ operations ($+$, $-$, $\times$, and $\div$) while the
  row-reduction methods takes about $2/3\, n^3$ operations.

\end{remark}


%% TODO: Add a couple more examples?

%% TODO: Add the following sections on interpretation of (computed) determinants
% \section{When matrices store data}
% \section{When matrices transform data}




\end{document}


%% \section{Connections to row operations and row echelon form}
%% Swapping Rows: If you swap two rows of a matrix, the determinant of the matrix changes sign. That is, if you swap rows ii and jj in a matrix AA, the determinant of the new matrix A′A′ is det⁡(A′)=−det⁡(A)det(A′)=−det(A).

%% Multiplying a Row by a Scalar: If you multiply a row by a non-zero scalar cc, the determinant of the matrix is multiplied by cc. That is, if you multiply row ii by cc in a matrix AA, the determinant of the new matrix A′A′ is det⁡(A′)=c⋅det⁡(A)det(A′)=c⋅det(A).

%% Adding a Multiple of One Row to Another: If you add a multiple of
%% one row to another row, the determinant of the matrix remains
%% unchanged. That is, if you replace row ii with
%% % rowi+c⋅rowjrowi​+c\cdot rowj​
%% in a matrix $A$, the determinant of the new
%% matrix $A'$ is $\det⁡(A')=det⁡(A)det(A')=det(A)$.
