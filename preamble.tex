\usepackage{ccicons}
\usepackage{multicol}
\newif\ifcolor %% for color cover
\colorfalse    %% add \colortrue before cover art


\newcommand{\dfn}{\textbf}
%\renewcommand{\vec}[1]{{\overset{\boldsymbol{\rightharpoonup}}{\mathbf{#1}}}\hspace{0in}}
\renewcommand{\vec}[1]{\mathbf{#1}}
%% Simple horiz vectors
%\renewcommand{\vector}[1]{\left\langle #1\right\rangle}
\newcommand{\arrowvec}[1]{{\overset{\rightharpoonup}{#1}}}
\newcommand{\R}{\mathbb{R}}
%\newcommand{\transpose}{\intercal}
\newcommand{\transpose}{\top}
\newcommand{\dotp}{\bullet}%% dot product
\renewcommand{\l}{\ell}
\DeclareMathOperator{\Span}{Span}
\DeclareMathOperator{\Null}{Null}
\DeclareMathOperator{\row}{row}
\DeclareMathOperator{\col}{col}
\renewcommand{\hat}{\widehat}
\newcommand{\N}{\mathbb N}



\let\defaultAnswerFormat\answerFormatBoxed
\usetikzlibrary{calc,bending}
\tikzset{>=stealth}
\usetikzlibrary{matrix} %% for fancy matrix stuff


\usepackage{mdframed} % For framing content
%\usepackage{ifthen}   % For conditional statements

% Define the 'concept' environment with an optional header
\newenvironment{concept}[1][]{%
  \begin{mdframed}[linecolor=black, linewidth=2pt, innertopmargin=5pt, innerbottommargin=5pt, skipabove=12pt, skipbelow=12pt]%
    \noindent\large\textbf{#1}\normalsize%
}{%
  \end{mdframed}%
}



%% https://texblog.net/latex-archive/maths/amsmath-matrix/
%% allows \begin{pmatrix}[cc|c]
\makeatletter
\renewcommand*\env@matrix[1][*\c@MaxMatrixCols c]{%
  \hskip -\arraycolsep
  \let\@ifnextchar\new@ifnextchar
  \array{#1}}
\makeatother




\colorlet{textColor}{black}
\colorlet{background}{white}
\colorlet{penColor}{blue!50!black} % Color of a curve in a plot
\colorlet{penColor2}{red!50!black}% Color of a curve in a plot
\colorlet{penColor3}{red!50!blue} % Color of a curve in a plot
\colorlet{penColor4}{green!50!black} % Color of a curve in a plot
\colorlet{penColor5}{orange!80!black} % Color of a curve in a plot
\colorlet{penColor6}{yellow!70!black} % Color of a curve in a plot
\colorlet{fill1}{penColor!20} % Color of fill in a plot
\colorlet{fill2}{penColor2!20} % Color of fill in a plot
\colorlet{fillp}{fill1} % Color of positive area
\colorlet{filln}{penColor2!20} % Color of negative area
\colorlet{fill3}{penColor3!20} % Fill
\colorlet{fill4}{penColor4!20} % Fill
\colorlet{fill5}{penColor5!20} % Fill
\colorlet{gridColor}{gray!50} % Color of grid in a plot
