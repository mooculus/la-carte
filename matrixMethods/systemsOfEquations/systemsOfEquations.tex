\documentclass{ximera}

\newcommand{\dfn}{\textbf}
\renewcommand{\vec}[1]{{\overset{\boldsymbol{\rightharpoonup}}{\mathbf{#1}}}\hspace{0in}}
%% Simple horiz vectors
\renewcommand{\vector}[1]{\left\langle #1\right\rangle}
\newcommand{\arrowvec}[1]{{\overset{\rightharpoonup}{#1}}}
\newcommand{\R}{\mathbb{R}}
\newcommand{\transpose}{\intercal}
\newcommand{\ro}{\texttt{R}}%% row operation
\newcommand{\dotp}{\bullet}%% dot product

\usetikzlibrary{calc,bending}
\tikzset{>=stealth}


\usepackage{mdframed} % For framing content
%\usepackage{ifthen}   % For conditional statements

% Define the 'concept' environment with an optional header
\newenvironment{concept}[1][]{%
  \begin{mdframed}[linecolor=black, linewidth=2pt, innertopmargin=5pt, innerbottommargin=5pt, skipabove=12pt, skipbelow=12pt]%
    \noindent\large\textbf{#1}\normalsize%
}{%
  \end{mdframed}%
}











%% \colorlet{textColor}{black}
%% \colorlet{background}{white}
%% \colorlet{penColor}{blue!50!black} % Color of a curve in a plot
%% \colorlet{penColor2}{red!50!black}% Color of a curve in a plot
%% \colorlet{penColor3}{red!50!blue} % Color of a curve in a plot
%% \colorlet{penColor4}{green!50!black} % Color of a curve in a plot
%% \colorlet{penColor5}{orange!80!black} % Color of a curve in a plot
%% \colorlet{penColor6}{yellow!70!black} % Color of a curve in a plot
%% \colorlet{fill1}{penColor!20} % Color of fill in a plot
%% \colorlet{fill2}{penColor2!20} % Color of fill in a plot
%% \colorlet{fillp}{fill1} % Color of positive area
%% \colorlet{filln}{penColor2!20} % Color of negative area
%% \colorlet{fill3}{penColor3!20} % Fill
%% \colorlet{fill4}{penColor4!20} % Fill
%% \colorlet{fill5}{penColor5!20} % Fill
%% \colorlet{gridColor}{gray!50} % Color of grid in a plot


%% Jim Hefferon’s Linear Algebra. (CC-BY-NC-SA)
%% Anna Davis and Paul Zachlin: https://github.com/annadavismath/LinearAlgebraV2
%% https://ximera.osu.edu/linearalgebrav3/LinearAlgebraInteractiveIntro/SYS-0030/main

\title{Matrices and systems of linear equations}


\begin{document}
\begin{abstract}
  We use matrics to solve systems of linear equations
\end{abstract}
\maketitle

\section{Row operations on matrices}


Any time we have a system of equations we represent it with a matrix:
\[
\begin{array}{ccccccc}
      3x & -&y&+&z&= &0 \\
     2x& +&y&+&2z&=&2\\
     x& +&4y&-&2z&=&11
\end{array}
\qquad\Rightarrow\qquad
\left(\begin{array}{ccc|c}
  3 & -1 & 1 & 0 \\
  2 &  1 & 2 & 2 \\
  1 &  4 & 2 & 11 
\end{array}\right)
\]
We're going $\ro_1\to  \ro_2$

TEACH ROW OPERATIONS WHILE REDUCING TO ROW ECHECLON FORM



\section{Row echelon form}

At this point, we can write many matrices that all represent systems
of equations that have a common solution. Some `forms' of these
matrices are more convient than others. One form of matrix is called
\dfn{row echelon form} or \dfn{triangular form}. An \textit{echelon}
is a military term meaning a formation (of soldiers, vechicals, and so
on) that makes a ``stair step'' shape. All of the matrices below are
in row echelon form:
\begin{tikzpicture}
\node at (0,0) {$\begin{pmatrix}
 ~1~&2&-1&-5&0&2\\0&0&3&1&2&0\\0&0&0&0&1&0
  \end{pmatrix}$};
\draw (-1.7,.7) -- (-1.7,.2) -- (-.6,.2) -- (-.6, -.2) -- (1,-.2) -- (1,-.7);
\end{tikzpicture}
BADBAD -- IDENTIFY STAIR STEP SHAPE
\[
\begin{bmatrix}
 \fbox{1}&2&-1&-5&0&2\\0&0&\fbox{3}&1&2&0\\0&0&0&0&\fbox{1}&0
\end{bmatrix}
\quad
\text{and}
\quad\begin{bmatrix}
 \fbox{4}&2\\0&\fbox{1}\\0&0
\end{bmatrix}
\]
The so-called leading entries (as viewed from left-to-right) of each
matrix are boxed.


\begin{tikzpicture}
\node at (0,0) {$\begin{pmatrix}
 \fbox{1}&2&-1&-5&0&2\\0&0&\fbox{3}&1&2&0\\0&0&0&0&\fbox{1}&0
  \end{pmatrix}$};
\draw (-1.9,.7) -- (-1.9,.2) -- (-.6,.2) -- (-.6, -.2) -- (1,-.2) -- (1,-.7);
\end{tikzpicture}



\section{Reduced row echelon form}






\end{document}
