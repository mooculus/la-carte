\documentclass{ximera}

\newcommand{\dfn}{\textbf}
\renewcommand{\vec}[1]{{\overset{\boldsymbol{\rightharpoonup}}{\mathbf{#1}}}\hspace{0in}}
%% Simple horiz vectors
\renewcommand{\vector}[1]{\left\langle #1\right\rangle}
\newcommand{\arrowvec}[1]{{\overset{\rightharpoonup}{#1}}}
\newcommand{\R}{\mathbb{R}}
\newcommand{\transpose}{\intercal}
\newcommand{\ro}{\texttt{R}}%% row operation
\newcommand{\dotp}{\bullet}%% dot product

\usetikzlibrary{calc,bending}
\tikzset{>=stealth}


\usepackage{mdframed} % For framing content
%\usepackage{ifthen}   % For conditional statements

% Define the 'concept' environment with an optional header
\newenvironment{concept}[1][]{%
  \begin{mdframed}[linecolor=black, linewidth=2pt, innertopmargin=5pt, innerbottommargin=5pt, skipabove=12pt, skipbelow=12pt]%
    \noindent\large\textbf{#1}\normalsize%
}{%
  \end{mdframed}%
}











%% \colorlet{textColor}{black}
%% \colorlet{background}{white}
%% \colorlet{penColor}{blue!50!black} % Color of a curve in a plot
%% \colorlet{penColor2}{red!50!black}% Color of a curve in a plot
%% \colorlet{penColor3}{red!50!blue} % Color of a curve in a plot
%% \colorlet{penColor4}{green!50!black} % Color of a curve in a plot
%% \colorlet{penColor5}{orange!80!black} % Color of a curve in a plot
%% \colorlet{penColor6}{yellow!70!black} % Color of a curve in a plot
%% \colorlet{fill1}{penColor!20} % Color of fill in a plot
%% \colorlet{fill2}{penColor2!20} % Color of fill in a plot
%% \colorlet{fillp}{fill1} % Color of positive area
%% \colorlet{filln}{penColor2!20} % Color of negative area
%% \colorlet{fill3}{penColor3!20} % Fill
%% \colorlet{fill4}{penColor4!20} % Fill
%% \colorlet{fill5}{penColor5!20} % Fill
%% \colorlet{gridColor}{gray!50} % Color of grid in a plot



\author{Parisa Fatheddin}



\begin{document}



\begin{exercise}
  Below we have a system of equations:
\begin{align*}
\frac{5}{3}x + \frac{7}{4}y &= 2\\
(4a)x + \frac{2}{3}z &= 1\\
\frac{3}{7} y + bz &= 11
\end{align*}
Determine the values of $a,b$ for which the given system has:
\begin{enumerate}
\item No solutions.
\item Infinitely many solutions.
\end{enumerate}
In either case, write ``DNE'' if there are no such values.
\begin{prompt}
  \begin{enumerate}
    %% TK: The answers below do not seem correct to me. What I have
    %% found is that if a = -10/(147*b) for any nonzero b \neq 1030/147,
    %% the system has no solutions. Please review.
  \item  There are no solutions:
    \[
    a = \left(\answer{-15/21}\right)k,\quad  b= \left(\answer{2/21}\right)k,
    \]
    where $k$ is any number.

    %% TK: The system permits infinitely many solutions
    %% for a = -1/103, b = 1030/147. Please review.
  \item There are infinitely many solutions:
    \[
    a = \answer[format=string]{DNE},\quad b= \answer[format=string]{DNE}.
    \]
\begin{hint}
Writing down the system in matrix form and transforming the entries in the first and third rows to integers gives,
\[
\left(\begin{array}{ccc|c}
  \answer{20} &  \answer{21} & \answer{0} &\answer{24} \\
  4a &  \answer{0} & \answer{2/3} & 1\\
  \answer{0} &  \answer{3} & \answer{7b} & \answer{77}
\end{array}\right)
\]
Then carry out $\frac{R_{1}}{20} \to R_1$ and
$-4a R_{1} + R_{2} \to R_2$. Finally, solve for $a$ and $b$.
\end{hint}
  \end{enumerate}
\end{prompt}
\end{exercise}

\end{document}

%% TK: Below is a short Maple script I wrote to check my computation
%% above.

% with(LinearAlgebra):
% A := Matrix([[5/3, 7/4, 0, 2], [0, 3/7, b, 11], [4*a, 0, 2/3, 1]]):
% A := GaussianElimination(A);
%
% # Condition to have infinitely many solutions:
% solve({A[3,3]=0, A[3,4]=0}, {a,b});
%
% # Conditions to have no solutions:
% solve({A[3,3]=0, A[3,4]<>0}, {a,b});
%
% # This last result should not be taken naively; it has to be
% interpreted more carefully. From the condition A[3,3]=0, we first note
% that a and b must be nonzero. Then we solve for a = -10/(147*b), which
% is justified because b is nonzero. Then we substitute the expression
% for a just obtained in the "non-equation" A[3,4]<>0, which simplifies
% to b<>1030/147.
%
% a_soln := solve(A[3,3]=0, a);
% b_soln := solve(subs(a=a_soln, A[3,4]<>0), b)[1];
