\documentclass{ximera}

\author{Tae Eun Kim}

\newcommand{\dfn}{\textbf}
\renewcommand{\vec}[1]{{\overset{\boldsymbol{\rightharpoonup}}{\mathbf{#1}}}\hspace{0in}}
%% Simple horiz vectors
\renewcommand{\vector}[1]{\left\langle #1\right\rangle}
\newcommand{\arrowvec}[1]{{\overset{\rightharpoonup}{#1}}}
\newcommand{\R}{\mathbb{R}}
\newcommand{\transpose}{\intercal}
\newcommand{\ro}{\texttt{R}}%% row operation
\newcommand{\dotp}{\bullet}%% dot product

\usetikzlibrary{calc,bending}
\tikzset{>=stealth}


\usepackage{mdframed} % For framing content
%\usepackage{ifthen}   % For conditional statements

% Define the 'concept' environment with an optional header
\newenvironment{concept}[1][]{%
  \begin{mdframed}[linecolor=black, linewidth=2pt, innertopmargin=5pt, innerbottommargin=5pt, skipabove=12pt, skipbelow=12pt]%
    \noindent\large\textbf{#1}\normalsize%
}{%
  \end{mdframed}%
}











%% \colorlet{textColor}{black}
%% \colorlet{background}{white}
%% \colorlet{penColor}{blue!50!black} % Color of a curve in a plot
%% \colorlet{penColor2}{red!50!black}% Color of a curve in a plot
%% \colorlet{penColor3}{red!50!blue} % Color of a curve in a plot
%% \colorlet{penColor4}{green!50!black} % Color of a curve in a plot
%% \colorlet{penColor5}{orange!80!black} % Color of a curve in a plot
%% \colorlet{penColor6}{yellow!70!black} % Color of a curve in a plot
%% \colorlet{fill1}{penColor!20} % Color of fill in a plot
%% \colorlet{fill2}{penColor2!20} % Color of fill in a plot
%% \colorlet{fillp}{fill1} % Color of positive area
%% \colorlet{filln}{penColor2!20} % Color of negative area
%% \colorlet{fill3}{penColor3!20} % Fill
%% \colorlet{fill4}{penColor4!20} % Fill
%% \colorlet{fill5}{penColor5!20} % Fill
%% \colorlet{gridColor}{gray!50} % Color of grid in a plot


\begin{document}
\begin{exercise}
  The exam scores (in percentage) of 35 students are encoded in a
  $3 \times 35$ matrix $S$
  \[
    S =
    \begin{pmatrix}
      |&|&&| \\
      \vec{s}_1 & \vec{s}_2 & \cdots & \vec{s}_{35} \\
      |&|&&|
    \end{pmatrix}
  \]
  where the $j$th column $\vec{s}_j$ contains the data of the $j$-th
  student:
  \[
    \vec{s}_j =
    \begin{pmatrix}
      s_{1,j}\\ s_{2,j} \\ s_{3,j}
    \end{pmatrix}
    \qquad
    \begin{array}{l}
      \text{Midterm 1}\\
      \text{Midterm 2}\\
      \text{Final Exam}
    \end{array}
  \]

  Assume that the exam weights are encoded in a column
  vector $\vec{w}$:
  \[
    \vec{w} =
    \begin{pmatrix}
      w_1\\ w_2\\ w_3
    \end{pmatrix}
    \qquad
    \begin{array}{l}
      \text{weight of Midterm 1}\\
      \text{weight of Midterm 2}\\
      \text{weight of Final Exam}
    \end{array}
  \]
  with $w_1 + w_2 + w_3 = 1$. Which of the following expressions
  correctly calculates the weighted averages of all 35 students?
  \begin{multipleChoice}
    \pdfOnly{\begin{multicols}{4}}
    \choice{$S \vec{w}$}
    \choice{$S \vec{w}^\transpose$}
    \choice{$\vec{w} S$}
    \choice[correct]{$\vec{w}^\transpose S$}
    \pdfOnly{\end{multicols}}
  \end{multipleChoice}
  \begin{hint}
    % Recall that a matrix-matrix product makes sense only when the
    % number of columns of the first matrix equals the number of rows of
    % the second matrix.
    The weighted average of the $j$th student is computed by computing
    the dot product of the two vectors $\vec{s}_j$ and $\vec{w}$.
  \end{hint}
  \begin{prompt}
    The outcome of the calculation is a
    \wordChoice{\choice[correct]{row} \choice{column}} vector
    containing 35 weighted averages.
  \end{prompt}
\end{exercise}
\end{document}
