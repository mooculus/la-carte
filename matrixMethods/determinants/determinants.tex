\documentclass{ximera}

\newcommand{\dfn}{\textbf}
\renewcommand{\vec}[1]{{\overset{\boldsymbol{\rightharpoonup}}{\mathbf{#1}}}\hspace{0in}}
%% Simple horiz vectors
\renewcommand{\vector}[1]{\left\langle #1\right\rangle}
\newcommand{\arrowvec}[1]{{\overset{\rightharpoonup}{#1}}}
\newcommand{\R}{\mathbb{R}}
\newcommand{\transpose}{\intercal}
\newcommand{\ro}{\texttt{R}}%% row operation
\newcommand{\dotp}{\bullet}%% dot product

\usetikzlibrary{calc,bending}
\tikzset{>=stealth}


\usepackage{mdframed} % For framing content
%\usepackage{ifthen}   % For conditional statements

% Define the 'concept' environment with an optional header
\newenvironment{concept}[1][]{%
  \begin{mdframed}[linecolor=black, linewidth=2pt, innertopmargin=5pt, innerbottommargin=5pt, skipabove=12pt, skipbelow=12pt]%
    \noindent\large\textbf{#1}\normalsize%
}{%
  \end{mdframed}%
}











%% \colorlet{textColor}{black}
%% \colorlet{background}{white}
%% \colorlet{penColor}{blue!50!black} % Color of a curve in a plot
%% \colorlet{penColor2}{red!50!black}% Color of a curve in a plot
%% \colorlet{penColor3}{red!50!blue} % Color of a curve in a plot
%% \colorlet{penColor4}{green!50!black} % Color of a curve in a plot
%% \colorlet{penColor5}{orange!80!black} % Color of a curve in a plot
%% \colorlet{penColor6}{yellow!70!black} % Color of a curve in a plot
%% \colorlet{fill1}{penColor!20} % Color of fill in a plot
%% \colorlet{fill2}{penColor2!20} % Color of fill in a plot
%% \colorlet{fillp}{fill1} % Color of positive area
%% \colorlet{filln}{penColor2!20} % Color of negative area
%% \colorlet{fill3}{penColor3!20} % Fill
%% \colorlet{fill4}{penColor4!20} % Fill
%% \colorlet{fill5}{penColor5!20} % Fill
%% \colorlet{gridColor}{gray!50} % Color of grid in a plot



\author{Bart Snapp}



\title{Determinants}


\begin{document}
\begin{abstract}
  Determinants are numbers that give information about matrices.
\end{abstract}
\maketitle







\section{Basic computations}

\begin{definition}
  Given a $2\times2$ matrix, the \dfn{determinant} is given by
  \[
  \det
  \begin{pmatrix}
    a & b\\
    c & d
  \end{pmatrix}
  =
  \begin{vmatrix}
    a & b\\
    c & d
  \end{vmatrix}
  = ad -bc.
  \]
\end{definition}
Why would anyone ever be interested in this? Well to start, given nonzero vectors
\[
\vec{v} = (a,b)\quad\text{and}\quad\vec{w}=(c,d)
\]
we can form a parallelogram as below.
\begin{image}
\begin{tikzpicture}
  \draw[fill=fill1!50!white,draw=none]
  (0, 0)         % starting point
  -- ++(5,1)     % move along this vector
  -- ++(2,3)     % then along this vector
  -- ++(-5,-1)   % then back along that vector
  -- cycle;      % and back to where you started
  \draw[->,ultra thick,penColor] (0,0) -- (5,1);
  \draw[->,ultra thick,penColor2] (0,0) -- (2,3);
  \draw[->,ultra thick,dashed,penColor2] (5,1) -- (7,4);
  \draw[->,ultra thick,dashed,penColor] (2,3) -- (7,4);
  \node[above,penColor] at (2.5,.5) {$\vec{v}$}; %% <a,b>
  \node[below right,penColor2] at (1,1.5) {$\vec{w}$}; %% <c,d>
  \draw[->] (.5,.1) arc[radius=.5cm,start angle=11.3,end angle=56.3];
  \node[above right] at (.4,.2) {$\theta$}; 
\end{tikzpicture}
\end{image}
The determinant gives the (signed) area of this parallelogram, where
the sign of the area is given by the sign of the angle $\theta$ (drawn
counterclockwise) between $\vec{v}$ and $\vec{w}$. To understand why
this is true, make a rectangle around the parallelogram as below.
\begin{image}
\begin{tikzpicture}
  \draw[fill=fill1!50!white,draw=none]
  (0, 0)         % starting point
  -- ++(5,1)     % move along this vector
  -- ++(2,3)     % then along this vector
  -- ++(-5,-1)   % then back along that vector
  -- cycle;      % and back to where you started

  \draw[fill=fill5] (5, 0) -- (7,0) -- (7,1) -- (5,1) -- cycle;
  \draw[fill=fill5] (0, 3) -- (2,3) -- (2,4) -- (0,4) -- cycle;

  \draw[fill=fill3] (0, 0) -- (5,0) -- (5,1) -- cycle;
  \draw[fill=fill3] (2,3) -- (7,4) -- (2,4) -- cycle;

  \draw[fill=fill4] (0, 0) -- (2,3) -- (0,3) -- cycle;
  \draw[fill=fill4] (5, 1) -- (7,1) -- (7,4) -- cycle;
  
  \draw[->,ultra thick,penColor] (0,0) -- (5,1);
  \draw[->,ultra thick,penColor2] (0,0) -- (2,3);
  \draw[->,ultra thick,dashed,penColor2] (5,1) -- (7,4);
  \draw[->,ultra thick,dashed,penColor] (2,3) -- (7,4);

  \draw[decoration={brace,mirror,raise=.1cm},decorate,thin] (0,0)--(5,0);
  \node[below] at (2.5,-.2) {$a$};

  \draw[decoration={brace,mirror,raise=.1cm},decorate,thin] (7,4)--(2,4);
  \node[above] at (4.5,4.2) {$a$};

  \draw[decoration={brace,mirror,raise=.1cm},decorate,thin] (7,0)--(7,1);
  \node[right] at (7.2,.5) {$b$};

  \draw[decoration={brace,mirror,raise=.1cm},decorate,thin] (0,4)--(0,3);
  \node[left] at (-.2,3.5) {$b$};

  \draw[decoration={brace,mirror,raise=.1cm},decorate,thin] (7,1)--(7,4);
  \node[right] at (7.2,2.5) {$d$};

  \draw[decoration={brace,mirror,raise=.1cm},decorate,thin] (0,3)--(0,0);
  \node[left] at (-.2,1.5) {$d$};

  \draw[decoration={brace,mirror,raise=.1cm},decorate,thin] (0,3)--(0,0);
  \node[left] at (-.2,1.5) {$d$};

  \draw[decoration={brace,raise=.1cm},decorate,thin] (0,4)--(2,4);
  \node[above] at (1,4.2) {$c$};

  \draw[decoration={brace,mirror,raise=.1cm},decorate,thin] (5,0)--(7,0);
  \node[below] at (6,-.2) {$c$};
  
  \node[above,penColor] at (2.5,.5) {$\vec{v}$}; %% <a,b>
  \node[below right,penColor2] at (1,1.5) {$\vec{w}$}; %% <c,d>
\end{tikzpicture}
\end{image}
Now, we see that the area of the parallelogram is
\[
\text{Area of rectangle} - \text{Area of other regions}
\]
and this is
\begin{align*}
  (a+c)(b+d) - \left(cd + ab + 2bc \right)&= ab + ad + bc + cd - cd - ab-2bc\\
  &=  ad - bc\\
  &= \det
  \begin{pmatrix}
    a & b\\
    c & d
  \end{pmatrix}.
\end{align*}

Typically, when one computes the determinant of a $2\times 2$ matrix,
we think of the terms as follows.
\begin{image}[1in]
  \begin{tikzpicture}
    \matrix (mtrx)  [matrix of math nodes,ampersand replacement=\&,%column sep=.1em,
      nodes={text height=1ex,text width=2ex}]
            {
              a \& b \\
              c \& d \\
            };
            \draw[thick] (mtrx-1-1.north) -| (mtrx-2-1.south west)
            -- (mtrx-2-1.south);
            \draw[thick] (mtrx-1-2.north) -| (mtrx-2-2.south east)
            -- (mtrx-2-2.south);
            \draw[ultra thick,red!20!white,->]
            (mtrx-1-2.center) -- (mtrx-2-1.center);
            \draw[draw,ultra thick,blue!20!white,->]
            (mtrx-1-1.center)  --  (mtrx-2-2.center);
            \matrix (mtrx2)  [matrix of math nodes,ampersand replacement=\&,nodes={text height=1ex,text width=2ex}]
            {
              a \& b\\
              c \& d\\
            };
            \node at (0,-.7) {${\color{blue!50!black}ad}-{\color{red!50!black}bc}$};
  \end{tikzpicture}
\end{image}


\begin{definition}
  Given a $3\times 3$ matrix, the \dfn{determinant} is denoted by the following.
  \[
  \det\begin{pmatrix}
  a_1 &  a_2 & a_3 \\
  b_1 &  b_2 & b_3 \\
  c_1 &  c_2 & c_3
  \end{pmatrix}
  =
  \begin{vmatrix}
    a_1 &  a_2 & a_3 \\
    b_1 &  b_2 & b_3 \\
    c_1 &  c_2 & c_3
  \end{vmatrix}
  \]
  The determinant can be computed using the following pattern.
\begin{image}[3in]
\begin{tikzpicture}
\node at (0,0) {$\det
\begin{pmatrix}
  a_1 & a_2 & a_3\\
  b_1 & b_2 & b_3\\
  c_1 & c_2 & c_3
\end{pmatrix}
= a_1
\begin{vmatrix}
  b_2 & b_3\\
  c_2 & c_3
\end{vmatrix}
-a_2
\begin{vmatrix}
  b_1 & b_3\\
  c_1 & c_3
\end{vmatrix}
+a_3
\begin{vmatrix}
  b_1 & b_2\\
  c_1 & c_2
\end{vmatrix}$};

\draw[ultra thick,->,gray] (-.27,-1.3) -- (-.27,-.7);
\draw[ultra thick,->,gray] (1.76,-1.3) -- (1.76,-.7);
\draw[ultra thick,->,gray] (3.8,-1.3) -- (3.8,-.7);

\node at (-.27,-2) {\scalebox{.7}{
    $\begin{pmatrix}
      a_1 & \bullet & \bullet\\
      \bullet & b_2 & b_3\\
      \bullet & c_2 & c_3
    \end{pmatrix}$}};

\node at (1.76,-2) {\scalebox{.7}{
    $\begin{pmatrix}
      \bullet & a_2 & \bullet\\
      b_1 & \bullet & b_3\\
      c_1 & \bullet & c_3
    \end{pmatrix}$}};

\node at (3.8,-2) {\scalebox{.7}{
    $\begin{pmatrix}
      \bullet & \bullet & a_3\\
      b_1 & b_2 & \bullet\\
      c_1 & c_2 & \bullet
    \end{pmatrix}$}};
\end{tikzpicture}
\end{image}
This equals
\[
 a_1(b_2c_3-b_3c_2)- a_2(b_1c_3-b_3c_1)+ a_3(b_1c_2-b_2c_1).
\]
\end{definition}

Another way to conceptualize this computation is the following:
%% Commented because this is not a particularly useful method in this course.
%% %% Credit for the following image goes to
%% %% http://tex.stackexchange.com/a/257063
\begin{image}
  \begin{tikzpicture}
    \matrix (mtrx)  [matrix of math nodes,ampersand replacement=\&,column sep=1em, nodes={text height=1ex,text width=2ex}]
            {
              a_1 \& a_2 \& a_3 \& a_1 \& a_2         \\
              b_1 \& b_2 \& b_3 \& b_1 \& b_2         \\
              c_1 \& c_2 \& c_3 \& c_1 \& c_2         \\
            };
            \matrix (mtrx3)  [matrix of math nodes,ampersand replacement=\&,column sep=1em, nodes={text height=1ex,text width=2ex}] at (5,0)
            { 
              a_1 \& a_2 \& a_3 \& a_1 \& a_2         \\
              b_1 \& b_2 \& b_3 \& b_1 \& b_2         \\
              c_1 \& c_2 \& c_3 \& c_1 \& c_2         \\
            };
            \draw[thick] (mtrx-1-1.north) -| (mtrx-3-1.south west)
            -- (mtrx-3-1.south);
            \draw[thick] (mtrx-1-3.north) -| (mtrx-3-3.south east)
            -- (mtrx-3-3.south);
            \draw[ultra thick,red!20!white,->]
            (mtrx3-3-1.center) edge (mtrx3-1-3.center)
            (mtrx3-3-2.center) edge (mtrx3-1-4.center)
            (mtrx3-3-3.center)  --  (mtrx3-1-5.center);
            \draw[draw,ultra thick,blue!20!white,->]
            (mtrx-1-1.center) edge (mtrx-3-3.center)
            (mtrx-1-2.center) edge (mtrx-3-4.center)
            (mtrx-1-3.center)  --  (mtrx-3-5.center);
            \matrix (mtrx2)  [matrix of math nodes,ampersand replacement=\&,column sep=1em, nodes={text height=1ex,text width=2ex}]
            { 
              a_1 \& a_2 \& a_3 \& a_1 \& a_2         \\
              b_1 \& b_2 \& b_3 \& b_1 \& b_2         \\
              c_1 \& c_2 \& c_3 \& c_1 \& c_2         \\
            };
            \matrix (mtrx4)  [matrix of math nodes,ampersand replacement=\&,column sep=1em, nodes={text height=1ex,text width=2ex}] at (5,0)
            { 
              a_1 \& a_2 \& a_3 \& a_1 \& a_2         \\
              b_1 \& b_2 \& b_3 \& b_1 \& b_2         \\
              c_1 \& c_2 \& c_3 \& c_1 \& c_2         \\
            };
            \draw[thick] (mtrx4-1-1.north) -| (mtrx4-3-1.south west)
            -- (mtrx4-3-1.south);
            \draw[thick] (mtrx4-1-3.north) -| (mtrx4-3-3.south east)
            -- (mtrx4-3-3.south);
            \node at (2.5,-1) {${\color{blue!50!black}a_1b_2c_3}+{\color{blue!50!black}a_2b_3c_1}+{\color{blue!50!black}a_3b_1c_2}-{\color{red!50!black}a_3b_2c_1}-{\color{red!50!black}a_1b_3c_2}-{\color{red!50!black}a_2b_1c_3}$};
            \node at (0,1) {{\color{blue!50!black}positive terms}};
            \node at (5,1) {{\color{red!50!black}negative terms}};
  \end{tikzpicture}
\end{image}

\begin{question}
  Compute.
  \[
  \det
  \begin{pmatrix} 
    1 & 4 & 7\\
    2 & 5 & 8\\
    3 & 6 & 9\\
  \end{pmatrix}
  \begin{prompt}
    = \answer{0}
  \end{prompt}
  \]
\end{question}



\section{Connections to row operations and row echelon form}

    %% Swapping Rows: If you swap two rows of a matrix, the determinant of the matrix changes sign. That is, if you swap rows ii and jj in a matrix AA, the determinant of the new matrix A′A′ is det⁡(A′)=−det⁡(A)det(A′)=−det(A).

    %% Multiplying a Row by a Scalar: If you multiply a row by a non-zero scalar cc, the determinant of the matrix is multiplied by cc. That is, if you multiply row ii by cc in a matrix AA, the determinant of the new matrix A′A′ is det⁡(A′)=c⋅det⁡(A)det(A′)=c⋅det(A).

    %% Adding a Multiple of One Row to Another: If you add a multiple of
    %% one row to another row, the determinant of the matrix remains
    %% unchanged. That is, if you replace row ii with
    %% %rowi+c⋅rowjrowi​+c\cdot rowj​
    %% in a matrix $A$, the determinant of the new
    %% matrix $A'$ is $\det⁡(A')=det⁡(A)det(A')=det(A)$.






\section{When matrices store data}



\section{When matrices transform data}





\end{document}
