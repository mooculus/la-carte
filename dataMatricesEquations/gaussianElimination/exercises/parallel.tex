\documentclass{ximera}

\newcommand{\dfn}{\textbf}
\renewcommand{\vec}[1]{{\overset{\boldsymbol{\rightharpoonup}}{\mathbf{#1}}}\hspace{0in}}
%% Simple horiz vectors
\renewcommand{\vector}[1]{\left\langle #1\right\rangle}
\newcommand{\arrowvec}[1]{{\overset{\rightharpoonup}{#1}}}
\newcommand{\R}{\mathbb{R}}
\newcommand{\transpose}{\intercal}
\newcommand{\ro}{\texttt{R}}%% row operation
\newcommand{\dotp}{\bullet}%% dot product
\renewcommand{\l}{\ell}
\let\defaultAnswerFormat\answerFormatBoxed
\usetikzlibrary{calc,bending}
\tikzset{>=stealth}


\usepackage{mdframed} % For framing content
%\usepackage{ifthen}   % For conditional statements

% Define the 'concept' environment with an optional header
\newenvironment{concept}[1][]{%
  \begin{mdframed}[linecolor=black, linewidth=2pt, innertopmargin=5pt, innerbottommargin=5pt, skipabove=12pt, skipbelow=12pt]%
    \noindent\large\textbf{#1}\normalsize%
}{%
  \end{mdframed}%
}



%% Define exercise collection. 
\makeatletter
\newcommand{\exerciseCollection}[2]{
\def\input@path{{#1}}
\activity{#1#2}}



\newcommand{\practicestyle}{
  
  \let\exercise\relax
\let\endexercise\relax
\let\c@exercise\relax
\let\problem\relax
\let\endproblem\relax
\let\c@problem\relax 
\newtheoremstyle{problem}
{\topsep}{\topsep}{\rmfamily}{}{\bfseries}{)}{ }{}
\theoremstyle{problem}
\newtheorem{problem}{}
\newtheorem{exercise}[problem]{}
\section{Exercises for Chapter~\thetitlenumber}
\small%\twocolumn
%% \let\exercise\relax
%%   \let\endexercise\relax
%%   \let\c@exercise\relax 
%%   \newtheoremstyle{exercise}
%%                   {\topsep}{\topsep}{\rmfamily}{}{\bfseries}{)}{~}{}
%% \theoremstyle{exercise}
%% \newtheorem{exercise}{}
}
\renewcommand\chapterstyle{%
  \def\activitystyle{activity-chapter}
  \normalsize
  %\onecolumn
  \def\maketitle{%
    \addtocounter{titlenumber}{1}%
                    {\flushleft\small\sffamily\bfseries\@pretitle\par\vspace{-1.5em}}%
                    {\flushleft\LARGE\sffamily\bfseries\thetitlenumber\hspace{1em}\@title \par }%
                    {\vskip .6em\noindent\textit\theabstract\setcounter{problem}{0}\setcounter{section}{0}}%
                    \par\vspace{2em}
                    \phantomsection\addcontentsline{toc}{section}{\textbf{\thetitlenumber\hspace{1em}\@title}}%
}}
\makeatother







%% \colorlet{textColor}{black}
%% \colorlet{background}{white}
%% \colorlet{penColor}{blue!50!black} % Color of a curve in a plot
%% \colorlet{penColor2}{red!50!black}% Color of a curve in a plot
%% \colorlet{penColor3}{red!50!blue} % Color of a curve in a plot
%% \colorlet{penColor4}{green!50!black} % Color of a curve in a plot
%% \colorlet{penColor5}{orange!80!black} % Color of a curve in a plot
%% \colorlet{penColor6}{yellow!70!black} % Color of a curve in a plot
%% \colorlet{fill1}{penColor!20} % Color of fill in a plot
%% \colorlet{fill2}{penColor2!20} % Color of fill in a plot
%% \colorlet{fillp}{fill1} % Color of positive area
%% \colorlet{filln}{penColor2!20} % Color of negative area
%% \colorlet{fill3}{penColor3!20} % Fill
%% \colorlet{fill4}{penColor4!20} % Fill
%% \colorlet{fill5}{penColor5!20} % Fill
%% \colorlet{gridColor}{gray!50} % Color of grid in a plot



\author{Parisa Fatheddin}



\begin{document}



\begin{exercise}
  Consider the following system equations:
  \begin{align*}
    2x-3y &= 8\\
    -2x +3y &= \frac{15}{4}
  \end{align*}
  Each equation in the system represents a line in the $xy$-plane.

  \begin{enumerate}
  \item What is the slope of the line represented by each equation?
    \begin{prompt}
      \[
        m_{1} = \answer{2/3}, \quad m_{2} = \answer{2/3}.
      \]
    \end{prompt}
  \item Are the two lines represented by the system parallel or perpendicular?
    \begin{prompt}
      \begin{multipleChoice}
        \choice[correct]{parallel}
        \choice{perpendicular}
        \choice{neither}
      \end{multipleChoice}
    \end{prompt}
  \item Does the system have one unique solution, no solution, or infinitely many solutions?
    \begin{prompt}
      \begin{multipleChoice}
        \choice{one unique solution}
        \choice[correct]{no solution}
        \choice{infinitely many}
      \end{multipleChoice}
    \end{prompt}
  \end{enumerate}
\end{exercise}
\end{document}


% a. What is the slope of each equation? \\

% \begin{prompt}
%   $m_{1} = \answer{2/3}, \hspace{.1cm} m_{2} = \answer{2/3}$
% \end{prompt}

% b. Are the two equations parallel or perpendicular? \\

% \begin{prompt}
%   \begin{multipleChoice}
%     \choice[correct]{parallel}
%     \choice{perpendicular}
%     \choice{neither}
%   \end{multipleChoice}
% \end{prompt}

% c. Does the system have one unique solution, no solution, or infinitely many solutions?

% \begin{prompt}
%   \begin{multipleChoice}
%     \choice{one unique solution}
%     \choice[correct]{no solution}
%     \choice{infinitely many}
%   \end{multipleChoice}
