\documentclass{ximera}

\newcommand{\dfn}{\textbf}
\renewcommand{\vec}[1]{{\overset{\boldsymbol{\rightharpoonup}}{\mathbf{#1}}}\hspace{0in}}
%% Simple horiz vectors
\renewcommand{\vector}[1]{\left\langle #1\right\rangle}
\newcommand{\arrowvec}[1]{{\overset{\rightharpoonup}{#1}}}
\newcommand{\R}{\mathbb{R}}
\newcommand{\transpose}{\intercal}
\newcommand{\ro}{\texttt{R}}%% row operation
\newcommand{\dotp}{\bullet}%% dot product

\usetikzlibrary{calc,bending}
\tikzset{>=stealth}


\usepackage{mdframed} % For framing content
%\usepackage{ifthen}   % For conditional statements

% Define the 'concept' environment with an optional header
\newenvironment{concept}[1][]{%
  \begin{mdframed}[linecolor=black, linewidth=2pt, innertopmargin=5pt, innerbottommargin=5pt, skipabove=12pt, skipbelow=12pt]%
    \noindent\large\textbf{#1}\normalsize%
}{%
  \end{mdframed}%
}











%% \colorlet{textColor}{black}
%% \colorlet{background}{white}
%% \colorlet{penColor}{blue!50!black} % Color of a curve in a plot
%% \colorlet{penColor2}{red!50!black}% Color of a curve in a plot
%% \colorlet{penColor3}{red!50!blue} % Color of a curve in a plot
%% \colorlet{penColor4}{green!50!black} % Color of a curve in a plot
%% \colorlet{penColor5}{orange!80!black} % Color of a curve in a plot
%% \colorlet{penColor6}{yellow!70!black} % Color of a curve in a plot
%% \colorlet{fill1}{penColor!20} % Color of fill in a plot
%% \colorlet{fill2}{penColor2!20} % Color of fill in a plot
%% \colorlet{fillp}{fill1} % Color of positive area
%% \colorlet{filln}{penColor2!20} % Color of negative area
%% \colorlet{fill3}{penColor3!20} % Fill
%% \colorlet{fill4}{penColor4!20} % Fill
%% \colorlet{fill5}{penColor5!20} % Fill
%% \colorlet{gridColor}{gray!50} % Color of grid in a plot



\author{Bart Snapp}
%%OpenAI. (2023). ChatGPT (Mar 14 version) [Large language model]. https://chat.openai.com/chat


\title{Vectors, scalars, and matrices}


\begin{document}
\begin{abstract}
  A concrete introduction to vectors and matrices, with an informal
  view towards vector spaces and linear transformations.
\end{abstract}
\maketitle


\section{Vectors and scalars}

In mathematics, we study (among other things) numbers. In many cases,
these numbers represent real-world data. Sometimes it ``makes sense''
to add data and sometimes it doesn't.
\paragraph{When it does not make sense to add data}
\begin{description}
\item[Personal Identification Numbers] Summing things like social
  security numbers, phone numbers, or zip codes doesn't provide
  meaningful information.
\item[Temperatures] While you can mathematically add temperatures
  together, doing so usually doesn't provide useful information.
\item[Time] Adding specific points in time, like dates or hours of the
  day, usually doesn't yield meaningful results. 
\item[Geographic Coordinates] Adding the latitude and longitude of two
  locations doesn't give you a location that has any real-world
  significance.
\end{description}
For some of the categories above, the \textit{average} can be
meaningful; or perhaps if each quantity is thought of as
`displacement' or `duration,' summing might be meaningful. However,
without additional (pun intended!) stipulations, summing the types of
quantities above is not meaningful. On the other hand, there's lots of
times that it makes sense to add data:
\paragraph{When it makes sense to add data}
\begin{description}
\item[Population Counts] Adding the populations of different regions,
  cities, or countries to find the total population of a larger
  area.
\item[Financial Transactions] Summing daily sales to find total
  monthly sales, or adding up all expenses to find total costs. These
  are real numbers that represent actual amounts of money, and their
  total gives meaningful information about financial status or
  performance.
\item[Time Spent on Tasks] Adding the duration of time spent on
  various activities on day gives a total time spent. We all have busy
  lives, and time is a very precious commodity. Hence, it is good to
  understand how we spend our time.
\item[Distances] If you walk $3$ miles in the morning and another $2$
  miles in the evening, adding these distances gives a meaningful
  total of $5$ miles walked in a day.
\end{description}
Often when it make sense to \textit{add} quantities, it make sense to
\textit{scale} them as well. For example with population, you could
ask for the population of a city with $3$ times the population, or
half the population. Similar \textit{scaling} examples exist for all
the examples above where it make sense to add data. Regular old
numbers used to scale vectors are called \textit{scalars}.


\begin{definition}
  We say data represents \dfn{vectors} $\vec{v}$ and $\vec{w}$ if
  \[
  s\vec{v}\quad\text{and}\quad s\vec{w}
  \]
  are also vectors (encode `meaningful' data in the same way that
  $\vec{v}$ and $\vec{w}$ do) and for all \textbf{scalars} $s$ (where $s$ is
  usually a real number),
  \[
  s(\vec{v} + \vec{w}) = s\vec{v}+ s\vec{w}.
  \]
\end{definition}
This might all sound very abstract. Let's give some very concrete
examples, based on the examples above. Before we start, we need to
give some notation for vectors.

\paragraph{Ways data can be stored in a vector}
\begin{description}
\item[As an ordered tuple] An ordered pair is just a pair of numbers
  $(a,b)$ delineated by parenthesis with the entries separated by a
  comma. It's called ``ordered'' because $(a,b) \ne (b,a)$. An \textit{ordered
  tuple} is just a list of an arbitrary, but fixed, number of elements
  that is ordered like an ordered pair. As an example, we could
  represent the \link[demographic information]{https://worldpopulationreview.com/states/states-by-race} for Ohio as an vector
  represented by an ordered tuple:
  \[
  \vec{p}_{\texttt{OH}} = (\underset{\text{White}}{9394878},\underset{\text{Black}}{1442655},\underset{\text{American Indian}}{20442},\underset{\text{Asian}}{268527},\underset{\text{Hawaiian}}{3907},\underset{\text{Other}}{544866})
  \]
\item[As a row vector] A row vector is a lot like an ordered tuple,
  except we do not use a comma to separate entries. Instead, we use a
  space.  Suppose a bookstore sells a variety of books in a month, say
  199 Science Fiction, 249 Fantasy, 141 Mystery, 304 Romance, 251
  Historical. We can express this data as a row vector:
  \[
  \vec{s} = \begin{pmatrix}199 & 249 & 141 & 304 & 251 \end{pmatrix} 
  \]
\item[As a column vector] A column vector is just like a row vector,
  except is it written vertically:
  \[
  \vec{w}_{\texttt{TR}} = \begin{pmatrix}
  0.5\\ 4 \\ 0 \\ 1\end{pmatrix}
  \]
  Above, we could suppose that $\vec{w}_{\texttt{TR}}$ represents the
  fact that someone spends $0.5$ hours on emails, $4$ hours in class,
  $0$ hours on projects, and $1$ hour exercising on Tuesdays and
  Thursdays.
\end{description}


\begin{definition}
  To switch a row vector into a column vector and vice versa, we use the \dfn{transpose} operation:
  \[
  \begin{pmatrix} 1 &  2 & 3 \end{pmatrix}^\transpose =
  \begin{pmatrix} 1 \\ 2 \\ 3 \end{pmatrix}
  \quad\text{and}\quad
  \begin{pmatrix} 1 \\ 2 \\ 3 \end{pmatrix}^\transpose =
  \begin{pmatrix} 1 &  2 & 3 \end{pmatrix}
  \]
\end{definition}

Once we have data encoded as vectors, we can describe some general
information about them. In particular, we can think about their
dimension and their components.

\begin{definition}
The \dfn{dimension} of a vector is the number of entries. Each
individual entry of a vector is called a \dfn{component}.
\end{definition}

\begin{question}
  What is the dimension of the vector $\vec{p}_{\texttt{OH}}$?
\end{question}


\begin{question}
  What is the dimension of the vector 
  \[
  \vector{3,4,1,-4}?
  \]
  \begin{prompt}
  \[
  \text{Dimension} = \answer{4}
  \]
  \end{prompt}
  \begin{question}
    What are the components of the vector $\vector{1,4,-3}$?
    \begin{prompt}
      \begin{itemize}
      \item The $x$-component is $\answer{1}$.
      \item The $y$-component is $\answer{4}$.
      \item The $z$-component is $\answer{-3}$.
      \end{itemize}
    \end{prompt}
  \end{question}
\end{question}



Onto the examples!

\begin{example}[Population Counts] %https://worldpopulationreview.com/states/states-by-race
  The Midwest of the United States consists of $12$ states. We can
  express the \link[demographics]{https://worldpopulationreview.com/states/states-by-race} of each state as a vector represented by an
  ordered tuple. The ordered tuple for Ohio looks like:
  \[
  \vec{p}_{\texttt{OH}} = (\underset{\text{White}}{9394878},\underset{\text{Black}}{1442655},\underset{\text{American Indian}}{20442},\underset{\text{Asian}}{268527},\underset{\text{Hawaiian}}{3907},\underset{\text{Other}}{544866}).
  \]
  The ordered tuples for each state in the Midwest looks like:
\begin{align*}
  \vec{p}_{\texttt{IA}} &= (2806418,117035,10538,79296,3941,132783)\\
  \vec{p}_{\texttt{IL}} &= (8874067,1796660,33972,709567,5196,1296702)\\
  \vec{p}_{\texttt{IN}} &= (5510354,631923,14030,158705,2205,379676)\\
  \vec{p}_{\texttt{KA}} &= (2416165,165837,22278,87093,2344,218902)\\
  \vec{p}_{\texttt{MI}} &= (7735902,1360149,50035,316844,3117,507860)\\
  \vec{p}_{\texttt{MN}} &= (4572149,359817,54558,275242,2201,336199)\\
  \vec{p}_{\texttt{MO}} &= (4978046,698043,24274,123810,8887,291100)\\
  \vec{p}_{\texttt{ND}} &= (651470,23959,39165,11979,1004,32817)\\
  \vec{p}_{\texttt{NE}} &= (1641256,91896,16875,47944,1235,124620)\\
  \vec{p}_{\texttt{OH}} &= (9394878,1442655,20442,268527,3907,544866)\\
  \vec{p}_{\texttt{SD}} &= (735228,18836,74975,12413,544,37340)\\
  \vec{p}_{\texttt{WI}} &= (4895065,367889,48674,163396,2672,329279)
\end{align*}
\begin{enumerate}
\item What are the combined demographics of the states Michigan, Ohio,
  and Indiana?
\item Suppose that the annual percentage growth rate of Ohio is
  currently $0.1\%$. Assuming this is even across all demographics,
  what might the population data look like for Ohio in 2025?
\end{enumerate}
\begin{solution}
  We'll use the properties of vectors to solve this problem.
  \begin{enumerate}
  \item To find the combined demographics of Michigan, Ohio, and
    Indiana we compute
    \[
    \vec{p}_{\texttt{MI}} + \vec{p}_{\texttt{OH}} + \vec{p}_{\texttt{IN}}
    \]
  \item To find the population growth of each demographic, we use the scalar, $1.001$ to find
    \[
    1.001 \vec{p}_{\texttt{OH}} = 
    \]
  \end{enumerate}
\end{solution}
\end{example}



\begin{example}[Financial Transactions]
  A bookstore sells a variety of books. We'll express the number of
  each type of book sold in one month as a row vector
  \[
  \vec{s} = \begin{pmatrix}199 & 249 & 141 & 304 & 251 \end{pmatrix} 
  \]
  with the entries representing the categories: Science Fiction,
  Fantasy, Mystery, Romance, Historical in that order.  If each book
  costs the bookstore $6$ dollars from the wholesaler, and each book sells
  for $9$ dollars, what is the profit for each category?
  \begin{solution}
    Using the properties of vectors, we wish to compute
    \[
    9\vec{s}-6\vec{s} = 
    \]
  \end{solution}
  
\end{example}



\begin{example}[Time Spent on Tasks]
  On Mondays, Wednesdays, and Fridays, suppose you spend you spend $1$
  hour on emails, $2$ hours in class, and $3$ hours working on a
  project. On Tuesdays and Thursdays, you spend $1$ hour on emails,
  $4$ hours in class, and $1$ hour exercising. We can express each day
  as a column vector:
  \[
 \vec{w}_{\texttt{MWF}} =
  \]
\end{example}




\section{Operations on vectors}


We can add vectors of the same dimension together by component-wise
addition. Here, it is useful to write vectors vertically.
\[
\begin{bmatrix}
  a\\
  b\\
  c
\end{bmatrix}
+
\begin{bmatrix}
  d\\
  e\\
  f
\end{bmatrix}
=
\begin{bmatrix}
  a+d\\
  b+e\\
  c+f
\end{bmatrix}.
\]

\begin{question}
  \[
  \vector{1,2,3}+ \vector{-1,2,2} =
  \vector{\answer{0},\answer{4},\answer{5}}
  \]
\end{question}

Now, let us investigate the geometry of addition of vectors. Let
$\vec{v} = \vector{1,2}$ and $\vec{w} = \vector{3,1}$.  If we
place the tail of the vector $\vec{w}$ at the tip of the vector
$\vec{v}$, like this:
\begin{image}
  \begin{tikzpicture}
    \begin{axis}[
        xmin=-1,xmax=5,ymin=-1,ymax=4,
            axis lines=center,
            %ticks=none,
            unit vector ratio*=1 1 1,
            xlabel=$x$, ylabel=$y$,
            ytick={-2,-1,...,7},
	    %yticklabels={$0.5$,$1$,$1.5$,$2$},
	    xtick={-2,-1,...,10},
	    %xticklabels={$0.5$,$1$,$1.5$,$2$},
	    grid = major,
            every axis y label/.style={at=(current axis.above origin),anchor=south},
            every axis x label/.style={at=(current axis.right of origin),anchor=west},
          ]
          \addplot[very thick,penColor,->] plot coordinates {(1,2) (4,3)};
          \addplot[very thick,penColor2,->] plot coordinates {(0,0) (1,2)};
          \addplot[ultra thick,penColor3,->] plot coordinates {(0,0) (4,3)};

           \node[left] at (axis cs:.5, 1) [penColor2] {$\vec{v}$};
           \node[above] at (axis cs:2.5, 2.5 ) [penColor] {$\vec{w}$};
           \node[below right] at (axis cs:2, 1.5 ) [penColor3] {$\vec{v}+\vec{w}$};
    \end{axis}
\end{tikzpicture}
\end{image}
or like this:
\begin{image}
  \begin{tikzpicture}
    \begin{axis}[
        xmin=-1,xmax=5,ymin=-1,ymax=4,
            axis lines=center,
            %ticks=none,
            unit vector ratio*=1 1 1,
            xlabel=$x$, ylabel=$y$,
            ytick={-2,-1,...,7},
	    %yticklabels={$0.5$,$1$,$1.5$,$2$},
	    xtick={-2,-1,...,10},
	    %xticklabels={$0.5$,$1$,$1.5$,$2$},
	    grid = major,
            every axis y label/.style={at=(current axis.above origin),anchor=south},
            every axis x label/.style={at=(current axis.right of origin),anchor=west},
          ]
          \addplot[very thick,penColor2,->] plot coordinates {(3,1) (4,3)};
          \addplot[very thick,penColor3,->] plot coordinates {(0,0) (4,3)};
          \addplot[ultra thick,penColor,->] plot coordinates {(0,0) (3,1)};

           \node[right] at (axis cs:3.5, 2) [penColor2] {$\vec{v}$};
           \node[below] at (axis cs:1.5, .5 ) [penColor] {$\vec{w}$};
           \node[above left] at (axis cs:2, 1.5 ) [penColor3] {$\vec{v}+\vec{w}$};
    \end{axis}
\end{tikzpicture}
\end{image}
then the sum $\vec{v}+\vec{w}$ connects the tail of $\vec{v}$ to the
tip of $\vec{w}$. In fact, you can think of the sum of two vectors as
being the diagonal of the parallelogram formed by the two vectors.
\begin{image}
  \begin{tikzpicture}
    \begin{axis}[
        xmin=-1,xmax=5,ymin=-1,ymax=4,
            axis lines=center,
            %ticks=none,
            unit vector ratio*=1 1 1,
            xlabel=$x$, ylabel=$y$,
            ytick={-2,-1,...,7},
	    %yticklabels={$0.5$,$1$,$1.5$,$2$},
	    xtick={-2,-1,...,10},
	    %xticklabels={$0.5$,$1$,$1.5$,$2$},
	    grid = major,
            every axis y label/.style={at=(current axis.above origin),anchor=south},
            every axis x label/.style={at=(current axis.right of origin),anchor=west},
          ]
          \addplot[very thick,penColor,->] plot coordinates {(1,2) (4,3)};
          \addplot[very thick,penColor2,->] plot coordinates {(0,0) (1,2)};
          \addplot[ultra thick,penColor3,->] plot coordinates {(0,0) (4,3)};

          \node[left] at (axis cs:.5, 1) [penColor2] {$\vec{v}$};
          \node[above] at (axis cs:2.5, 2.5 ) [penColor] {$\vec{w}$};
          \node[below right] at (axis cs:1.9, 1.6 ) [penColor3] {$\vec{v}+\vec{w}$};
          
          \addplot[very thick,penColor2,->] plot coordinates {(3,1) (4,3)};
          \addplot[very thick,penColor3,->] plot coordinates {(0,0) (4,3)};
          \addplot[ultra thick,penColor,->] plot coordinates {(0,0) (3,1)};
          
          \node[right] at (axis cs:3.5, 2) [penColor2] {$\vec{v}$};
          \node[below] at (axis cs:1.5, .5 ) [penColor] {$\vec{w}$};
          %\node[above left] at (axis cs:2, 1.5 ) [penColor3] {$\vec{v}+\vec{w}$};
    \end{axis}
\end{tikzpicture}
\end{image}



Hence,
\[
\vec{v}+\vec{w} = \vector{4 , 3}.
\]
\begin{question}
  Consider the following diagram.
  \begin{image}
  \begin{tikzpicture}
    \begin{axis}[
        xmin=0,xmax=5,ymin=-1,ymax=4,
        axis lines=center,
            %ticks=none,
            unit vector ratio*=1 1 1,
            xlabel=$x$, ylabel=$y$,
            ytick={-2,-1,...,7},
	    %yticklabels={$0.5$,$1$,$1.5$,$2$},
	    xtick={-2,-1,...,10},
	    %xticklabels={$0.5$,$1$,$1.5$,$2$},
	    grid = major,
            every axis y label/.style={at=(current axis.above origin),anchor=south},
            every axis x label/.style={at=(current axis.right of origin),anchor=west},
          ]
          \addplot[very thick,penColor,->] plot coordinates {(2,3) (1,0)};
          \addplot[very thick,penColor2,->] plot coordinates {(4,1) (1,0)};
          \addplot[very thick,penColor3,->] plot coordinates {(4,1) (2,3)};
          
          \node[left] at (axis cs:1.5, 1.5 ) [penColor] {$\vec{a}$};
          \node[below] at (axis cs:2.5, .5) [penColor2] {$\vec{b}$};
          \node[above right] at (axis cs:3, 2 ) [penColor3] {$\vec{c}$};
    \end{axis}
  \end{tikzpicture}
  \end{image}
  Which equation is represented by the diagram above?
  \begin{multipleChoice}
    \choice{$\vec{a} + \vec{b} = \vec{c}$}
    \choice[correct]{$\vec{a} + \vec{c} = \vec{b}$}
    \choice{$\vec{b} + \vec{c} = \vec{a}$}
  \end{multipleChoice}
  \begin{feedback}[correct]
    Notice that we also could have drawn the diagram above like this.
      \begin{image}
        \begin{tikzpicture}
          \begin{axis}[
              xmin=0,xmax=5,ymin=-1,ymax=4,
              axis lines=center,
              %ticks=none,
              unit vector ratio*=1 1 1,
              xlabel=$x$, ylabel=$y$,
              ytick={-2,-1,...,7},
	      %yticklabels={$0.5$,$1$,$1.5$,$2$},
	      xtick={-2,-1,...,10},
	      %xticklabels={$0.5$,$1$,$1.5$,$2$},
	      grid = major,
              every axis y label/.style={at=(current axis.above origin),anchor=south},
              every axis x label/.style={at=(current axis.right of origin),anchor=west},
            ]
            \addplot[very thick,penColor,->] plot coordinates {(4,3) (3,0)};
            \addplot[very thick,penColor2,->] plot coordinates {(4,3) (1,2)};
            \addplot[very thick,penColor3,->] plot coordinates {(3,0) (1,2)};
          
            \node[right] at (axis cs:3.5, 1.5 ) [penColor] {$\vec{a}$};
            \node[above] at (axis cs:2.5, 2.5) [penColor2] {$\vec{b}$};
            \node[below left] at (axis cs:2, 1 ) [penColor3] {$\vec{c}$};
          \end{axis}
        \end{tikzpicture}
      \end{image}
  \end{feedback}
\end{question}



We can also multiply vectors by a \dfn{scalar} (a number), by
multiplying each component by the scalar:

\begin{question}
  \[
  4\cdot \vector{2 , 4 , 0 , 1} = \vector{\answer{8}, \answer{16} , \answer{0} , \answer{4}}
  \]	
\end{question}

\begin{question}
  True or False: Multiplying a vector by a nonzero scalar will not
  change the direction of the vector.
  \begin{prompt}
  \begin{multipleChoice}
    \choice{true}
    \choice[correct]{false}
  \end{multipleChoice}
  \end{prompt}
  \begin{feedback}
    Multiplying a vector by a positive scalar $s$ will not change the
    direction of the vector.
    \begin{image}
      \begin{tikzpicture}
	 \begin{axis}[
            xmin=-1,xmax=7,ymin=-1,ymax=3,
            clip=false,
            axis lines=center,
            %ticks=none,
            unit vector ratio*=1 1 1,
            xlabel=$x$, ylabel=$y$,
            ytick={-1,0,...,3},
	    xtick={-1,0,...,7},
	    grid = major,
            every axis y label/.style={at=(current axis.above origin),anchor=south},
            every axis x label/.style={at=(current axis.right of origin),anchor=west},
          ]
          \addplot[ultra thick,penColor,->] plot coordinates {(0,0) (3,1)};
          \addplot[very thick,penColor2,->] plot coordinates {(0,0) (6,2)};
          \node[above] at (axis cs:1.5, .5) [penColor] {$\vec{v}$};
          \node[below] at (axis cs:4, 1.25) [penColor2] {$s\cdot\vec{v}$};

         \end{axis}
       \end{tikzpicture}
    \end{image}
    However, if we multiply a vector by a \textit{negative} scalar $-s$, 
    then the direction will change.
    \begin{image}
      \begin{tikzpicture}
	 \begin{axis}[
            xmin=-7,xmax=4,ymin=-3,ymax=2,
            clip=false,
            axis lines=center,
            %ticks=none,
            unit vector ratio*=1 1 1,
            xlabel=$x$, ylabel=$y$,
            ytick={-3,-2,...,2},
	    xtick={-7,-6,...,4},
	    grid = major,
            every axis y label/.style={at=(current axis.above origin),anchor=south},
            every axis x label/.style={at=(current axis.right of origin),anchor=west},
          ]
          \addplot[very thick,penColor,->] plot coordinates {(0,0) (3,1)};
          \addplot[very thick,penColor2,->] plot coordinates {(0,0) (-6,-2)};
          \node[above] at (axis cs:1.5, .5) [penColor] {$\vec{v}$};
          \node[below] at (axis cs:-3, -1) [penColor2] {$-s\cdot\vec{v}$};

         \end{axis}
       \end{tikzpicture}
    \end{image}
  \end{feedback}
\end{question}

Thinking about how the magnitude of a vector changes when we multiply by a scalar 
reveals why scalars are called \textit{scalars}.

\begin{onlineOnly}
  You can use this interactive to see how scalars affect vectors.
  \begin{center}
    \geogebra{WYNdzcGP}{800}{600}%https://www.geogebra.org/m/WYNdzcGP
  \end{center}
\end{onlineOnly}

\begin{question}
  Consider a vector
  \[
  \vec{v} = \vector{a,b,c}.
  \]
  What is the magnitude of $\vec{v}$?
  \begin{prompt}
    \[
    |\vec{v}| = \answer{\sqrt{a^2+b^2+c^2}}
    \]
  \end{prompt}
  \begin{question}
    What is the magnitude of $6\cdot\vec{v}$?
    \begin{prompt}
      \[
      |6\cdot\vec{v}| = \answer{6}\cdot\sqrt{a^2+b^2+c^2}
      \]
    \end{prompt}
    \begin{question}
    What is the magnitude of $-6\cdot\vec{v}$?
    \begin{prompt}
      \[
      |-6\cdot\vec{v}| = \answer{6}\cdot\sqrt{a^2+b^2+c^2}
      \]
    \end{prompt}
    \begin{feedback}
      If $s$ is a positive constant, and $\vec{v}$ is a vector, then
      vector $s\cdot\vec{v}$ points in the same direction as
      $\vec{v}$, but its length is scaled by a factor of $s$.  If $s$
      is negative, then $s\cdot\vec{v}$ points in the opposite
      direction of $\vec{v}$, and its length is scaled by a factor of
      $|s|$.
    \end{feedback}
  \end{question}
  \end{question}
\end{question}


\section{Matrices store and transform data}


\subsection{Matrices store data}

Extract a new data set from a matrix using basis vectors

\subsection{Matrices transform data}

Scale/normalize population etc.


Geometry


\end{document}
