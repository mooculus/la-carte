\documentclass{ximera}

\newcommand{\dfn}{\textbf}
\renewcommand{\vec}[1]{{\overset{\boldsymbol{\rightharpoonup}}{\mathbf{#1}}}\hspace{0in}}
%% Simple horiz vectors
\renewcommand{\vector}[1]{\left\langle #1\right\rangle}
\newcommand{\arrowvec}[1]{{\overset{\rightharpoonup}{#1}}}
\newcommand{\R}{\mathbb{R}}
\newcommand{\transpose}{\intercal}
\newcommand{\ro}{\texttt{R}}%% row operation
\newcommand{\dotp}{\bullet}%% dot product

\usetikzlibrary{calc,bending}
\tikzset{>=stealth}


\usepackage{mdframed} % For framing content
%\usepackage{ifthen}   % For conditional statements

% Define the 'concept' environment with an optional header
\newenvironment{concept}[1][]{%
  \begin{mdframed}[linecolor=black, linewidth=2pt, innertopmargin=5pt, innerbottommargin=5pt, skipabove=12pt, skipbelow=12pt]%
    \noindent\large\textbf{#1}\normalsize%
}{%
  \end{mdframed}%
}











%% \colorlet{textColor}{black}
%% \colorlet{background}{white}
%% \colorlet{penColor}{blue!50!black} % Color of a curve in a plot
%% \colorlet{penColor2}{red!50!black}% Color of a curve in a plot
%% \colorlet{penColor3}{red!50!blue} % Color of a curve in a plot
%% \colorlet{penColor4}{green!50!black} % Color of a curve in a plot
%% \colorlet{penColor5}{orange!80!black} % Color of a curve in a plot
%% \colorlet{penColor6}{yellow!70!black} % Color of a curve in a plot
%% \colorlet{fill1}{penColor!20} % Color of fill in a plot
%% \colorlet{fill2}{penColor2!20} % Color of fill in a plot
%% \colorlet{fillp}{fill1} % Color of positive area
%% \colorlet{filln}{penColor2!20} % Color of negative area
%% \colorlet{fill3}{penColor3!20} % Fill
%% \colorlet{fill4}{penColor4!20} % Fill
%% \colorlet{fill5}{penColor5!20} % Fill
%% \colorlet{gridColor}{gray!50} % Color of grid in a plot



\author{Bart Snapp}



\title{Inverses of matrices}


\begin{document}
\begin{abstract}
  We view matrices as functions and compute their inverse
\end{abstract}
\maketitle

\section{Matrices are functions}

\begin{definition}
  A \textbf{linear function} is a function $L:\R^n\to\R^m$ such that
  \[
  L(a\vec{x}+b\vec{y}) = aL(\vec{x}) + bL(\vec{y})
  \]
  for all $a,b\in\R$ and $\vec{x},\vec{y}\in \R^n$.
\end{definition}

\begin{question}
  Now consider this function:
  \begin{align*}
    \vec{F}: \R^6 &\to \R^4\\
    \begin{pmatrix}
      x_1\\
      x_2\\
      x_3\\
      x_4\\
      x_5\\
      x_6
    \end{pmatrix}
    &\mapsto
        \begin{pmatrix}
      x_1+1\\
      x_2+1\\
      x_3+1\\
      x_4+1
    \end{pmatrix}
  \end{align*}
  Starting from the definition of a linear function, \textbf{is $\vec{F}$ a
  linear function? Why or why not?}
\end{question}



\section{Inverses of matrices}

\end{document}
